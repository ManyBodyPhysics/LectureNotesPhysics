\title{Motivation and overarching aims}
\author{Morten Hjorth-Jensen, Maria Paola Lombardo, and Ubirajara van Kolck}
\institute{Morten Hjorth-Jensen  \at Department of Physics and Astronomy and National Superconducting Cyclotron Laboratory, Michigan State University, East Lansing, Michigan, USA and Department of Physics, University of Oslo, Oslo, Norway, \email{hjensen@msu.edu}, \and Maria Paola Lombardo \at Name and address of institution(s), \email{mariapaola.lombardo@lnf.infn.it}, \and Ubirajara van Kolck \at Name and address of institution(s), \email{vankolck@ipno.in2p3.fr}}
\maketitle

\abstract{Our presentation}


Nuclear physics has recently experienced several discoveries and
technological advances that address the fundamental questions of the
field, in particular how nuclei emerge from the strong dynamics
of quantum chromodynamics (QCD).
Many of these advances have been made possible by significant
investments in frontier research facilities worldwide over the last
two decades. Some of these discoveries are the detection of perhaps
the most exotic state of matter, the quark-gluon plasma, which is
believed to have existed in the very first moments of the Universe
(refs).  Recent experiments have validated the standard solar model
and established that neutrinos have mass (refs). High-precision
measurements of the quark structure of the nucleon are challenging
existing theoretical understanding.  Nuclear physicists have started
to explore a completely unknown landscape of nuclei with extreme
neutron-to-proton ratios using radioactive and short-lived ions,
including rare and very neutron-rich isotopes.  These experiments push
us towards the extremes of nuclear stability.  Moreover, these rare
nuclei lie at the heart of nucleosynthesis processes in the universe
and are therefore an important component in the puzzle of matter
generation in the universe.

A firm experimental and theoretical understanding of nuclear stability
in terms of the basic constituents is a huge intellectual endeavor.
Experiments indicate that developing a comprehensive description of
all nuclei and their reactions requires theoretical and experimental
investigations of rare isotopes with unusual neutron-to-proton ratios
that are very different from their stable counterparts.  These rare
nuclei are difficult to produce and study experimentally since they
can have extremely short lifetimes. Theoretical approaches to these
nuclei involve solving the nuclear many-body problem.

Accompanying these developments, a qualitative change has swept the
nuclear theory landscape thanks to a combination of techniques that is
allowing, for the first time, th direct connection between QCD and
nuclear structure. This transformation has been brought by a dramatic
improvement in the capability of numerical calculations both in QCD,
via lattice simulations, and in the nuclear many-body problem, via "ab
initio" methods for the diagonilization of non-relativistic
Hamiltonians. Simultaneously, the framework of effective field
theories builts a bridge between the two numerical approaches,
allowing to convert the results of lattice QCD into input to ab initio
methods.

Now, algorithmic and computational advances hold promise for
breakthroughs in predictive power including proper error estimates,
enhancing the already strong links between theory and experiment.
These advances include better ab initio many-body methods as well as a
better understanding of the underlying effective degrees of freedom
and the respective forces at play.  And obviously better numerical
algorithms as well as developments in high-performance computing.
This will provide us with important new insights about the stability
of nuclear matter and allow us to relate these novel understandings to
the underlying laws of motion, the corresponding forces and the
pertinent fundamental building blocks.

Important issues such as whether we can explain from first-principle
methods the existence of magic numbers and their vanishing as we add
more and more nucleons, how the binding energy of neutron-rich nuclei
behaves, or the radii, neutron skins, and many many other probes that
extract information about many-body correlations as nuclei evolve
towards their limits of stability --- these are all fundamental
questions which, combined with recent experimental and theoretical
advances, will allow us to advance our basic knowledge about the
limits of stability of matter, and, hopefully, help us in gaining a
better understanding of visible matter.

It is within this framework the present texts finds its rationale.
This text collects and synthesizes seven series of lectures on the
nuclear many-body problem, starting from our present understanding of
the underlying forces with a presentation of recent advances within
the field of lattice QCD, via effective field theories to central
many-body methods like Monte Carlo, coupled-cluster, and large-scale
diagonalization methods.  The applications span from our smallest
components, quarks and gluons as the mediators of the strong force to
the computation of the equation of state for infinite nuclear matter
and neutron star matter.  The lectures provide a proper exposition of
the underlying theoretical and algorithmic approaches as well as
strong ties to the numerical implementation of the exposed methods.
The lectures propose exercises, often providing a proper link to
actual numerical software The latter will enable the reader to build
upon these and develop his/her own insights about these methods, as
well as using these codes for developing his/her own programs for
tackling complicated many-body problems.






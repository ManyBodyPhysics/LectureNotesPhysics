\title{From few to many nucleons and methods for nuclear reactions}
\author{Giuseppina Orlandini}
\institute{Giuseppina Orlandini \at Name of institution and address, \email{name@email.address}}
\maketitle
\abstract{Each chapter should be preceded by an abstract (10--15 lines long) that summarizes the content. The abstract will appear \textit{online} at \url{www.SpringerLink.com} and be available with unrestricted access. This allows unregistered users to read the abstract as a teaser for the complete chapter. As a general rule the abstracts will not appear in the printed version of your book unless it is the style of your particular book or that of the series to which your book belongs.\newline\indent
Please use the 'starred' version of the new Springer \texttt{abstract} command for typesetting the text of the online abstracts (cf. source file of this chapter template \texttt{abstract}) and include them with the source files of your manuscript. Use the plain \texttt{abstract} command if the abstract is also to appear in the printed version of the book.}
%\maketile


\section{The Nuclear few- and many-body problem}
\section{Methods for bound states based on the variational principle I:The No Core Shell Model (NCSM)}
\section{Methods for bound states based on the variational principle II:The Hyperspherical Harmonics (HH) method}
\section{Methods for reactions involving continuum states I:Perturbation induced reactions and integral transforms}
\section{Methods for reactions involving continuum states II:The continuum state problem reduced to a bound state problem}

\begin{acknowledgement}
If you want to include acknowledgments of assistance and the like at the end of an individual chapter please use the \verb|acknowledgement| environment -- it will automatically render Springer's preferred layout.
\end{acknowledgement}
%




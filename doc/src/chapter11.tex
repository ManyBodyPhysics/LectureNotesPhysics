\title{Self-consistent Green's function approaches}
\author{Carlo~Barbieri and Arianna Carbone}
\institute{
Carlo Barbieri  \at 1 Department of Physics, University of Surrey, Guildford GU2 7XH, UK \email{C.Barbieri@surrey.ac.uk},
\and Arianna Carbone   \at 2 Institut f\"ur Kernphysik, Technische Universit\"at Darmstadt, 64289 Darmstadt, Germany, and 
\\ ExtreMe Matter Institute EMMI, GSI Helmholtzzentrum f\"ur Schwerionenforschung GmbH, 64291 Darmstadt, Germany, \email{arianna@theorie.ikp.physik.tu-darmstadt.de}
}
\maketitle
\abstract{
We present the fundamental techniques and working equations of many-body Green's function theory for calculating ground state properties and the spectral strength.  Green's function methods closely relate to other polynomial scaling approaches discussed in chapters~8 and ~10. However, here we aim directly at a global view of the many-fermion structure.
  We  derive the working equations for calculating many-body propagators, using both the Algebraic Diagrammatic Construction technique and the self-consistent formalism at finite temperature.  Their implementation is discussed, as well as the the inclusion of three-nucleon interactions.
 %The third order ADC approach [ADC(3)] is the method of choice for handling finite nuclei and can also be applied to  infinite systems by discretising the single particle Hilbert space in momentum coordinates. 
 %As a related numerical project, we describe the construction of a complete numerical code for infinite neutron and nuclear matter.
%We then focus on calculations matter at finite temperature which do not require a discretisation of the Hilbert space. 
The self-consistency feature is essential to guarantee thermodynamic consistency.
The  paring and neutron matter models introduced in previous chapters are solved and compared with the other methods in this book.}

\allowdisplaybreaks[1]


\section{Introduction}

Ab-intio methods that present polynomial scaling with the number of particles have proven highly useful in reaching
finite systems of rather large sizes up to medium mass nuclei~\cite{ch11_Soma2014s2n,ch11_Binder2014ccSn132,ch11_Hergert2014Ni} and even infinite matter~\cite{ch11_Frick2003,ch11_Baardsen2013ccNM,ch11_Carbone2014}. Most approaches of this type that are discussed in previous chapters aim at the direct evaluation of the ground state energy of the system, where several other quantities of interests can be addressed in a second stage through the equation of motion and particle removal or attachment techniques.
%
Here, we will follow a different route and focus on gaining from the start a global  view of the spectral structure of a system of fermions. Our approach will be that of calculating directly the self-energy (also know as {\em mass operator}), which describes the 
complete response of a particle embedded in the true ground state of the system. This not only provides an effective in medium interaction for the nucleon, but it is also the optical potential for elastic scattering and it yields the spectral information relative to the attachment and removal of a particle.   Once the  one body Green's function has been obtained, the total energy of the system is calculated, as the final step, by means of appropriate sum rules~\cite{ch11_dickhoff2004}.

Two main approaches have become standard choices for calculations of Green's functions in nuclear many-body theory.
%
The Algebraic Diagrammatic Construction (ADC) method, that was originally devised for quantum chemistry applications~\cite{ch11_Schirmer1982ADC2,ch11_Schirmer1983ADCn}, has proven to be optimal for discrete bases, as it is normally necessary to exploit for finite nuclei. However, this can also be applied to fermion gases in a box with periodic boundary conditions, which  simplifies the analysis even more thanks to translational invariance. We will focus on the case on infinite neutron matter and provide an example of a working numerical code.
%
ADC($n$) methods are part of a larger class of approaches based on intermediate-state representations (ISRs) to which also the equation-of-motion coupled cluster belongs~\cite{ch11_Mertins1996ISR1,ch11_Mertins1996ISR2}.
%
The other method consists in solving directly the nucleon-nucleon ladder scattering matrix for dressed particles in the medium, which can be done effectively in a finite temperature formalism~\cite{ch11_Frick2004PhD,ch11_Rios2007PhD}. Hence, this makes possible to study thermodynamic properties of the infinite and liquid matter. For these studies to be reliable, it is mandatory to ensure the satisfaction of fundamental conservation laws and to maintain thermodynamic consistency in the infinite size limit. We show here how to achieve this by preforming fully {\em self-consistent} calculations of the Green's function. In this context, `self-consistency' means that the information on the ground state and excitations of the systems are not obtained from a reference state but taken directly from the true correlated wave function. To achieve this, the computed spectral function is fed back into the working equations and calculations are repeated until a consistency between input and output is obtained.  This approach 
is referred to as self-consistent Green's function (SCGF) method and it is always implemented, partially or in full, for nuclear structure applications.

Very recent advances in computational applications concern the extension of SCGF theory to the Gorkov-Nambu formalism for the breaking on particle number symmetry~\cite{ch11_VdSluys1993,ch11_Soma2011GkvI,ch11_Idini2012}. This allows to  treat pairing systematically in systems with degenerate reference states and, therefore, calculate directly open-shell nuclei. As a result, this developments have  opened the possibility of  studying large set of semi-magic nuclei that were previously beyond the reach of ab-initio theory. We will not discuss the Gorkov-SCGF approach here, but we will focus on the fundamental features of the standard approaches instead. The interested reader is referred to recent literature on the topic~\cite{ch11_Soma2011GkvI,ch11_Soma2013rc,ch11_Soma2014Lanc}.

In the process of discussing the relevant working equations  of SCGF theory, we will also deal with applications to the same pairing model and the neutron matter with the Minnesota potential already discussed in chapters~8 and~9.  Together with presenting the most important steps for their numerical implementations, this book provides two examples of working codes in FORTRAN and C++ that can solve these models.  Results for the self-energy and spectral functions  should serve to gain a deeper understanding of the many-body physics that is embedded in the SCGF method.
 In discussing this, we will also present a final benchmark among the converged results from other methods discussed in this book: coupled cluster (chapter~8),  Monte Carlo (chapter~9),  in-medium similarity renormalization group (chapter~10),  and SCGF~(this chapter).

%We introduce the concept of many-particle Green's function in Section~\ref{sec:scgf_defs} and discuss the relation between the one-body spectral function and experimental data. This will also allow us to cover the main sum rules needed to extract the total binding energy of the system and any one-body observable.
 %This will also allow us to cover the main sum rules needed to extract the total binding energy of the system and any one-body observable.
 %The ADC($n$) method is explained in Sec.~\ref{sec:scgf_adc}, where all working equations up to third order are derived.  We demonstrate how to apply this in a calculation of infinite matter in  Sec.~\ref{sec:scgf_comp} by using a simplified two-nucleon interaction. This section will give insight on how to construct the corresponding C++ code included with this book.
%The finite temperature formalism is then introduced in Sec.~\ref{sec:scgf_finiteT} together with working equation used as standard in the nuclear physics literature.
%
%Further details of the implementation of chiral 3NF in the finite temperature formalism are given in  Appendix~\ref{app:scgf_3NF}, while a summary of the Feynman rules for the general case of three-nucleon interactions are given in Appendix~\ref{app:Feyn_rules}.


\section{Many-body Green's function theory}
\label{sec:scgf_defs}

%%%We introduce the concept of many-particle Green's function in Section~\ref{sec:scgf_defs} and discuss the relation between the one-body spectral function and experimental data. This will also allow us to cover the main sum rules needed to extract the total binding energy of the system and any one-body observable.
 

This chapter will focus on many-body {\em  Green's functions}, which are also referred to as {\em propagators}. These are defined  in the 
second quantization formalism by assuming the knowledge of the the ground state $\vert\Psi^A_0\rangle$ of a target system of $A$ nucleons, which
is taken to be a vacuum of excitations.  The one-body Green's function (or propagator) is then defined as~\cite{ch11_FetterWalecka,ch11_Dickhoff2008}:
\begin{equation}
 i\hbar \; g_{\alpha\beta}(t - t') =  \langle\Psi^A_0\vert  \pazocal{T} [ a_\alpha(t)   a^\dagger_\beta(t') ]  \vert\Psi^A_0\rangle \; ,
 \label{eq:g1Time}
\end{equation}
where $\pazocal{T}$ is the time ordering operator, $a^\dagger_\alpha(t)$~($a_\alpha(t)$) are the creation (annihilation) operators in Heisenberg picture,
and  greek indices $\alpha$,~$\beta$,~... label a complete single particle basis that defines our  model space. These can be states in momentum or coordinate space or any discrete set of single particle basis. Note that $ g(t - t')$ depends only on the time difference $t-t'$ due to time translation invariance.
For $t>t'$, Eq.~\eqref{eq:g1Time} gives the probability amplitude to add a particle to $\vert\Psi^A_0\rangle$ in state $\beta$  at time $t'$ and then to let it propagate to state $\alpha$ at a later time $t$. Vice versa, for $t<t'$ a particle is removed from state $\alpha$ at $t$ and added to $\beta$ at $t'$.

In spite of being the simplest type of propagator, the one-body Green's function does contain a wealth of information regarding single particle behavior inside the many-body system, one-body observables, the total binding energy, and even elastic nucleon-nucleus scattering.
%
The propagation of a particle or a hole excitation correspond to the time evolution of an intermediate many-body system with $A+1$ or $A-1$ particles.
One can better  understand the physics information included in Eq.~\eqref{eq:g1Time} from considering the eigenstates $\vert\Psi^{A+1}_n\rangle$, $\vert\Psi^{A-1}_k\rangle$ and eigenvalues $E^{A+1}_n$, $E^{A-1}_k$ of these intermediate systems. By expanding on these eigenstates and Fourier transforming from time to frequency, one arrives at the Lehmann representation of the one-body Green's function~\cite{ch11_Lehmann1954}:
\begin{eqnarray}
 g_{\alpha\beta}(\omega) ~&=&~ \int d\tau \; e^{i\omega\tau} g_{\alpha\beta}(\tau)
 \nonumber \\
 &=&~
 \sum_n  \frac{ 
          \langle\Psi^A_0\vert  	a_\alpha   \vert\Psi^{A+1}_n\rangle
          \langle\Psi^{A+1}_n\vert  a^\dagger_\beta  \vert\Psi^A_0\rangle
              }{\hbar\omega - (E^{A+1}_n - E^A_0) + {\rm i} \eta }
 ~+~ \sum_k \frac{
 		  \langle\Psi^A_0\vert  	a^\dagger_\beta   \vert\Psi^{A-1}_k\rangle
          \langle\Psi^{A-1}_k\vert  a_\alpha  \vert\Psi^A_0\rangle
              }{\hbar\omega - (E^A_0 - E^{A-1}_k) - {\rm i} \eta } \; ,
 \nonumber \\
  &\equiv&~ \sum_n \frac{(\pazocal{X}^{n}_\alpha)^*  \pazocal{X}^{n}_\beta}{\hbar\omega  - \varepsilon^+_{n} + i\eta} 
        ~+~ \sum_k \frac{\pazocal{Y}^{k}_\alpha  (\pazocal{Y}^{k}_\beta)^*}{\hbar\omega  - \varepsilon^-_{k} - i\eta}  \; ,
\label{eq:g1Leh}
\end{eqnarray}
where the operators $a^\dagger_\alpha$ and~$a_\alpha$ are now in Sch\"ordinger picture.
%
For the rest of this chapter (with the only exception of Appendix 1) we will work in dimensionless $\hbar=c=1$ units to avoid carrying over unnecessary  $\hbar$ terms.
%
From Eq.~\eqref{eq:g1Leh}, we see that the poles of the Green's function,
\hbox{$\varepsilon_n^{+}\equiv(E^{A+1}_n - E^A_0)$} and  \hbox{$\varepsilon_k^{-}\equiv(E^A_0 - E^{A-1}_k)$},
are one-nucleon addition and removal energies, respectively.  Note that these are generically referred to in the literature as ``separation'' or ``quasiparticle'' energies although the first naming should normally refer to transitions involving only ($A\pm1$)-nucleon ground states.  We will use the second convention in the following, unless the two naming are strictly equivalent.
%
In the last line of Eq.~\eqref{eq:g1Leh} we have also introduced short notations for the spectroscopic amplitudes associated with the addition
($\pazocal{X}^{n}_\alpha \equiv \langle\Psi^{A+1}_n\vert  a^\dagger_\alpha  \vert\Psi^A_0\rangle$) and the removal
($\pazocal{Y}^{k}_\alpha \equiv \langle\Psi^{A-1}_k\vert  a_\alpha  \vert\Psi^A_0\rangle$)  of a particle to and from the initial ground state~$\vert\Psi^A_0\rangle$.
We will use the latin letter  $n$ to label one-particle excitations and to distinguish them from one-hole states that are indicated by $k$ instead.
 This compact form will simplify deriving the working formalism in the following sections. 


\begin{figure}[t]
\begin{center}
\includegraphics[width=0.42\textwidth]{Chapter11-figures/fig11_1_a.pdf}   \hspace{0.08\textwidth} 
\includegraphics[width=0.42\textwidth]{Chapter11-figures/fig11_1_b.pdf}
\caption{Diagrammatic representations of the Dyson equation. The diagram on the left represents Eq.~\eqref{eq:Dyson_a}, while its conjugate equation~\eqref{eq:Dyson_b} is shown to the right. Single lines with an
arrow represent the unperturbed propagator $g^{(0)}_{\alpha\beta}(\omega)$ and  double lines are the fully
dressed propagator $g_{\alpha\beta}(\omega)$ of Eq.~\eqref{eq:g1Leh}.  Both equations, when expanded in terms
of $g^{(0)}_{\alpha\beta}(\omega)$, give rise to the same expansion for the correlated propagator. } 
\label{fig:DysonEq}
\end{center}
\end{figure}

The one-body Green's function~\eqref{eq:g1Leh} is completely determined by solving the Dyson equation:
\begin{subequations}
\label{eq:Dyson}
\begin{eqnarray}
  \label{eq:Dyson_a}
  g_{\alpha\beta}(\omega)&=&g^{(0)}_{\alpha\beta}(\omega) ~+~ \sum_{\gamma\delta} \; g^{(0)}_{\alpha\gamma}(\omega) \, \Sigma_{\gamma\delta}^{\star}(\omega) \, g_{\delta\beta}(\omega) 
   \\   \label{eq:Dyson_b}
  &=&g^{(0)}_{\alpha\beta}(\omega) ~+~ \sum_{\gamma\delta} \; g_{\alpha\gamma}(\omega) \, \Sigma_{\gamma\delta}^{\star}(\omega) \, g^{(0)}_{\delta\beta}(\omega)  \; ,
\end{eqnarray}
\end{subequations}
where we have put in evidence that there exists two different conjugate forms of this equation, corresponding to the first and second line. 
In Eqs.~\eqref{eq:Dyson},  the unperturbed propagator $g^{(0)}_{\alpha\beta}(\omega)$ is the initial reference state (usually a mean-field or Hartree-Fock state), while $g_{\alpha\beta}(\omega)$ is called the {\em correlated} or {\em dressed} propagator. The quantity $\Sigma_{\gamma\delta}^{\star}(\omega)$ is the {\em irreducible self-energy}
and it is often refereed to as the {\em mass operator}.  This operator plays a central role in the GF formalism and can be interpreted as the 
non-local and energy-dependent potential that each fermion feels due to the interactions with the medium.  For frequencies $\omega>0$,  the solution of Eqs.~\eqref{eq:Dyson} yields a continuum spectrum with $E^{A+1}_n > E^A_0$ and the state $\vert\Psi^{A+1}_n\rangle$  describes the elastic scattering of the additional nucleon off the $\vert\Psi^A_0\rangle$ target. It can be show that $\Sigma^{\star}(\omega)$ is an exact optical potential for scattering of a particle  from the many-body target~\cite{ch11_MahauxSartor91,ch11_Capuzzi1996,ch11_Cederbaum2001}.
 %
The Dyson equation is nonlinear in its solution, $g(\omega)$, and thus it corresponds to an all-orders resummation of diagrams involving the self-energy.
 %
 The Feynman diagrams corresponding to both forms of the Dyson equation are shown in Fig.~\ref{fig:DysonEq}.
In both cases, by recursively substituting the exact Green's function (indicated by double lines) that appears on the right hand side with the whole equation, one finds equal expansions in terms of the unperturbed $g^{(0)}(\omega)$ and the irreducible self-energy. The solution of Eqs.~\eqref{eq:Dyson} is referred to as {\em dressed} propagators since it formally results by `dressing' the free particle by repeated interactions with the system ($\Sigma^{\star}(\omega)$).

 
 A full knowledge of the self-energy  $\Sigma^{\star}(\omega)$ (see Eqs.~(\ref{eq:Dyson})) would yield the exact solution for $g(\omega)$ but in practice this has to be approximated somehow.
 Standard perturbation theory,  expands $\Sigma^{\star}(\omega)$ in a series of terms that depend on the interactions and on the unperturbed propagator $g^{(0)}(\omega)$.  However, it is also possible to rearrange the 
perturbative expansion in diagrams that depend only on the exact dressed propagator itself (that is, $\Sigma^{\star}=\Sigma^{\star}[g(\omega)]$).  Since any propagator in this diagrammatic expansion is already dressed, one only need to consider a smaller set of contributions---the so-called  {\em skeleton} diagrams.
These are diagrams that  do not explicitly include any self-energy insertion, as these are already generated by Eqs.~\eqref{eq:Dyson}. 
We will discuss these aspects in more detail in Sec.~\ref{sec:pertexp}.
%
For the present discussion, we only need to be aware that the functional dependence of $\Sigma^{\star}[g(\omega)]$  requires an iterative procedure in which  $\Sigma^{\star}(\omega)$ and Eqs.~\eqref{eq:Dyson} are calculated several times until they converge to a unique solution.
 %
This approach defines the SCGF method and it is particularly important since it can be shown that full self-consistency guarantees to exactly satisfy fundamental symmetries and conservations laws~\cite{ch11_Baym1961,ch11_Baym1962}.
In practical applications, and especially in finite systems, this scheme may not be achievable exactly and self-consistency is implemented only partially for the most important contributions. Normally this is still sufficient to obtain highly accurate results.
 %
 We will present suitable approximation schemes to calculate the self-energy in the following sections. In particular, we will focus on the ADC($n$) method   that can be applied with discretized bases in finite and infinite systems in Secs.~\ref{sec:scgf_adc} and~\ref{sec:scgf_comp}. The case of extended systems at finite temperature is discussed in Sec.~\ref{sec:scgf_finiteT}.   Before going into the actual approximation schemes, we need to see how experimental quantities can be calculated  once the one-body propagator is known, as well as to discuss the basic results of  perturbation theory.




\subsection{Spectral function and relation to experimental observations}
\label{sec:scgf_obs}

Once the one-body Green's function is known, it can be used to calculate the total binding energy and the expectation values of all one-body observables.
The attractive feature of the SCGF approach is that $g(\omega)$ describes the one-body dynamics completely.  
This information can be recast in the particle and hole spectral functions, which contain the separate responses for the attachment and removal of a nucleon.
They can be obtained directly from Eq.~\eqref{eq:g1Leh}, as follows:
 \begin{eqnarray}
\nonumber 
S^{p}_{\alpha \beta}(\omega)&=& -\frac 1 \pi \rm{Im} \; g_{\alpha \beta}(\omega) = \sum_{n} \left({\pazocal X}^n_\alpha \right)^*  {\pazocal X}^n_\beta ~ \delta\Big(\omega-(E^{A+1}_n - E^A_0)\Big) \; ,  \qquad \rm{for ~ }\omega \geq \varepsilon^+_0 \; ,
\\
 \label{eq:SpSh}
S^{h}_{\alpha \beta}(\omega)&=& ~~ \frac 1 \pi \rm{Im } \;  g_{\alpha \beta}(\omega) = \sum_{k}  {\pazocal Y}^k_\alpha  ({\pazocal Y}^k_\beta )^* ~\delta\Big(\omega-(E^A_0 - E^{A-1}_k)\Big) \; ,  \qquad \rm{for ~ }\omega \leq \varepsilon^-_0 \;  .
\end{eqnarray} 
The diagonal parts of Eqs.~\eqref{eq:SpSh}, have a straightforward physical interpretation~\cite{ch11_FetterWalecka,ch11_Dickhoff2008}. 
The particle part, $S^p_{\alpha \alpha}(\omega)$, is the joint probability of adding a nucleon with quantum numbers $\alpha$ to the A-body ground state, $|\Psi^A_0\rangle$, and then to find the system in a final state with energy $E^{A+1}=E^A_0 + \omega$. Likewise,  $S^h_{\alpha \alpha}(\omega)$ gives the probability of removing a particle from state $\alpha$ while leaving the nucleus in an eigenstate of energy $E^{A-1}=E^A_0 - \omega$.  These are demonstrated in coordinate space in Fig.~\ref{fig:ScptFnctN56} for neutrons around $^{56}$Ni.  Below the Fermi energy, \hbox{$E_F\equiv \frac 1 2 (\varepsilon^+_0 + \varepsilon^-_0)$} one can see one single dominant quasihole peak, corresponding to the $f_{7/2}$ orbit, and the states from the $sd$ shell, which are instead very fragmented. Just above $E_F$, one has sharp quasiparticles corresponding to the attachment of a neutron in the remaining $pf$ orbits. Finally, for $\omega>0$, one has neutron-$^{56}$Ni elastic scattering states. Remarkably, one can see that dominant quasiparticle peaks persist around the Fermi surface, which  confirms the existence of a shell structure outside the $^{40}$Ca core.



\begin{figure}[t]
\begin{center}
\includegraphics[height=0.38\textwidth]{Chapter11-figures/Fig_Ni56_SctFn.pdf}
\caption{
Calculated single-particle spectral function for the addition and removal of a neutron to and form $^{56}$Ni, from Ref.~\cite{ch11_Barbieri2009Ni}. The diagonal part, $S_{r,r}(\omega)$, is
shown in coordinate space. Energies below the Fermi level, $E_F$ correspond to the  one-hole spectral function $S^h_{r,r}(\omega)$  which describes
the distribution of nucleons in energy and coordinate space. Integrating over all the quasihole energies yields the matter  density distribution, Eq.~\eqref{eq:rho_r}.
Energies above $E_F$ are for the one-particle spectral function $S^p_{r,r}(\omega)$.}
\label{fig:ScptFnctN56}
\end{center}
\end{figure}

The existence of isolated dominant peaks  as those shown in Fig.~\ref{fig:ScptFnctN56} indicates that the eigenstates $|\Psi^{A+1}_n\rangle$ and $|\Psi^{A-1}_k\rangle$ are to a very good approximation constructed of a nucleon or a hole independently orbiting the ground state $|\Psi^A_0\rangle$. This is the basic hypothesis at the origin of the nuclear shell-model.
 How much a real nucleus deviates from this assumption can be gauged by the deviations in the values of their {\em spectroscopic factors}. These are defined
 as the normalization overlap of the spectroscopic amplitudes for the attachment or removal of a particle:
\begin{eqnarray}
  SF_n^+=   \sum_\alpha  |{\pazocal X}^n_\alpha|^2 \; ,
 \qquad \qquad && \qquad \qquad 
  SF_k^- =  \sum_\alpha  |{\pazocal Y}^k_\alpha|^2 \; .
\label{eq:SFdef}
\end{eqnarray}
The energy distribution of spectroscopic factors is given by
\begin{eqnarray}
  S(\omega) &=& ~ \sum_\alpha S^p_{\alpha \alpha}(\omega) \quad + \quad \sum_\alpha S^h_{\alpha \alpha}(\omega)  \nonumber \\
          &=& ~ \sum_n  SF_n^+ \, \delta( \omega - E^{A+1}_n + E^A_0) 
          ~+~  \sum_n  SF_k^- \, \delta( \omega - E^A_0 + E^{A-1}_k ) \; ,
\label{eq:SFvsE}
\end{eqnarray}
where each $\delta$-peak corresponds to eigenstate of a neighboring isotope with $A\pm1$ particles. these quasiparticle energy are directly observed in nucleon addition and removal experiments.
%
Note that the total strength  seen in such experiment results from a convolution of the spectroscopic amplitudes and the dynamics of the reaction mechanisms. Hence, while the quasiparticle energies appearing in the poles of Eq.~\eqref{eq:g1Leh} are strictly observed, the magnitude of the spectral strength $S(\omega)$ only gives a semi-quantitative description of the observed cross sections.



Any one-body observable can be calculated via the one-body density matrix $\rho_{\alpha\beta}$, which is obtained from $g_{\alpha\beta}(\omega)$ as follows:
\begin{equation}
 \rho_{\alpha \beta} ~\equiv ~ \langle\Psi_0^A\vert a^{\dag}_{\beta}a_{\alpha} \vert\Psi_0^A\rangle
  ~=~  \int_{-\infty}^{\varepsilon^-_0}S^h_{\alpha\beta}(\omega)~d\omega=\sum_k ({\pazocal Y}^k_{\beta})^*{\pazocal Y}^k_{\alpha} .
 \label{eq:OBDM}
\end{equation}
The expectation value of a one-body operator, ${\widehat O}^{1B}$, can then be written in terms of the ${\pazocal Y}$ amplitudes as:
\begin{equation}
\label{eq:den_one}
\langle {\widehat O}^{1B}\rangle  =\sum_{\alpha\beta}  O^{1B}_{\alpha \beta}\,\,\rho_{\beta\alpha}=\sum_k \sum_{\alpha\beta}~ ({\pazocal Y}^k_{\alpha})^*~ O^{1B}_{\alpha\beta} ~  {\pazocal Y}^k_{\beta} \; .
\end{equation}
However, evaluating two- and many-nucleon observables requires the knowledge of many-body propagators. 
%
Eq.~\eqref{eq:OBDM} also implies that the density profile of the system can be obtained by integrating over the hole  spectral function in coordinate space (cf. Fig.~\ref{fig:ScptFnctN56}):
\begin{equation}
  \label{eq:rho_r}
   \rho({\bf r})  =  \int_{-\infty}^{\varepsilon^-_0}S^h_{{\bf r},{\bf r}}(\omega)~d\omega \; .
\end{equation}
Likewise, a second sum (or integration) over the coordinate space yields the total number of particles,
\begin{equation}
  \int d\,{\bf r} \,  \int_{-\infty}^{\varepsilon^-_0} d\omega \; S^h_{{\bf r},{\bf r}}(\omega) ~=~
   \sum_\alpha \int_{-\infty}^{\varepsilon^-_0} \, S^h_{\alpha\alpha}(\omega)~d\omega ~=~ A \; .
\end{equation}

A very special case is the Koltun sum-rule that allows calculating the total energy of the system by means of the exact one-body propagator alone, $g(\omega)$~\cite{ch11_Galitskii1958KSR,ch11_Koltun1974KSR}. 
This relation is exact for any Hamiltonian containing at most one- and two body interactions.  When many-particle interactions are present, it is necessary 
to correct for the over countings that arise from these additional terms~\cite{ch11_Carbone2013Nov}. For the specific case in which a three-body interaction $\widehat{W}$ in included, the exact relation for the ground state energy is given by the following modified Koltun rule:
\begin{equation}
  \label{eq:Koltun_hW}
  E^A_0 = \sum_{\alpha\beta} \frac{1}{2} 
            \int_{-\infty}^{\varepsilon^-_0}  [\,T_{\alpha\beta}+\omega\,\delta_{\alpha\beta}\, ]
            \, S^h_{\beta\alpha}(\omega) \; d\omega
            ~-~  \frac{1}{2} \langle \widehat W\rangle \, .
\end{equation}
This still relies on the use of a one-body propagator but
 it requires  the additional evaluation of the expectation value of the three-body interaction,~$\langle \widehat W \rangle$ (which in principle requires the knowledge of more complex Green's functions).
%
Thankfully, in most cases the total strength of $\widehat{W}$ is much smaller than other terms in the Hamiltonian. Thus, one can safely approximate its expectation value at lowest order, in terms of three  non-interacting density matrices, as
\begin{equation}
   \label{eq:Wddd}
    \langle \widehat W\rangle\simeq\frac{1}{6} \, \sum_{\alpha\beta\mu\gamma\delta\nu} W_{\alpha\beta\mu,\gamma\delta\nu}~\rho_{\gamma\alpha}~\rho_{\delta\beta}~\rho_{\nu\mu} \; .
\end{equation}
%
As a typical example in finite nuclei, the error from this approximation has been estimated not to exceed 250~keV for the total binding energies for $^{16}$O and $^{24}$O~\cite{ch11_Cipollone2013prl}. However, the accuracy of Eq.~\eqref{eq:Wddd} is not guaranteed and needs to be verified case by case.


\subsection{Perturbation expansion of the Green 's function}
\label{sec:pertexp}

In order to understand the following sections and to devise appropriate approximations to the self-energy $\Sigma^{\star}(\omega)$ it is necessary to understand the basic elements of perturbation theory.  These will be also fundamental to derive all-order summation schemes leading to non perturbative solutions and to discuss the concept of self-consistency. We summarize here the material needed to understand the following sections, while the full set of Feynman rules is reviewed in  Appendix 1.

We work with a system of $A$ non-relativistic fermions interacting by means of two-body and three-body interactions.
We divide the Hamiltonian into two parts, $\widehat H = \widehat H_0 + \widehat H_1$. 
The unperturbed term, $\widehat H_0 = \widehat T + \widehat U$, is given by the sum of the kinetic term and an auxiliary one-body operator~$\widehat U$.
Its choice defines the reference state, $\vert\Phi_0^A\rangle$, and the corresponding
unperturbed propagator $g^{(0)}(\omega)$ that are the starting point for the perturbative expansion\footnote{A typical choice in nuclear physics would be a Slater determinant such as the solution of the Hartree-Fock problem or a set of single-particle harmonic oscillator wave functions.}.  
The perturbative term is then  
$\widehat H_1 = -\widehat U + \widehat V + \widehat W$, where $\widehat V$ denotes the two-body interaction operator and $\widehat W$  is the three-body interaction.
In a second-quantized framework, the full Hamiltonian reads:
\begin{equation}
\label{eq:H}
\widehat H = \sum_{\alpha} \varepsilon^0_\alpha\, a^\dag_\alpha a_\alpha - \sum_{\alpha\beta}U_{\alpha \beta}\, a^\dag_\alpha a_{\beta}
%\\\nonumber &+&
+\frac{1}{4} \sum_{\substack{\alpha\gamma\\\beta\delta}}V_{\alpha\gamma,\beta\delta}\, a_\alpha^\dag a_\gamma^\dag a_{\delta} a_{\beta}
%\\\nonumber &+& 
+\frac{1}{36}\sum_{\substack{\alpha\gamma\epsilon \\ \beta\delta\eta}} W_{\alpha\gamma\epsilon,\beta\delta\eta}\,
a_\alpha^\dag a_\gamma^\dag a_\epsilon^\dag a_{\eta} a_{\delta} a_{\beta} \, .
\end{equation}
In Eq.~\eqref{eq:H} we continue to use greek indices $\alpha$,$\beta$,$\gamma$,\ldots to label the single particle basis that defines the models space. But we make the additional assumption that these are the same states which diagonalize the unperturbed Hamiltonian, $\widehat H_0$, with eigenvalues $\varepsilon_\alpha^0$.    This choice is made in most  applications of  perturbation theory but it is not strictly necessary here and it will not affect our discussion in the following sections. 
The matrix elements of the one-body operator $\widehat U$ are given by $U_{\alpha \beta}$. And we  work with
properly antisymmetrized matrix elements of the two-body and three-body forces, $V_{\alpha\gamma,\beta\delta}$ and $W_{\alpha\gamma\epsilon,\beta\delta\eta}$.

In time representation, the many-body Green's functions are defined as the expectation value of  time-ordered products of  annihilation and creation operators  in the Heisenberg picture. This is shown by Eq.~\eqref{eq:g1Time} for the single particle propagator.
%Upon  Fourier transformation this leads to the Lehmann representation given in Eq.~(\ref{eq:g1Leh}).
Each of them can be expanded in a perturbation series in powers of $\widehat{H}_1$.
For the one-body propagator this  reads \cite{ch11_Mattuck1992,ch11_Dickhoff2008}:
\begin{equation}
\label{gpert}
g_{\alpha\beta}(t_\alpha-t_\beta) = (- i) \sum_{n=0}^\infty \left(-  i\right)^n\frac{1}{n!}\int \hspace{-1mm} {\rm d} t_1 \; \ldots \int \hspace{-1mm} {\rm d} t_n \langle\Phi_0^A\vert {\pazocal T} [\widehat H^I_1(t_1) \ldots \widehat H^I_1(t_n)a^I_\alpha(t_\alpha){a_{\beta}^I}^\dag(t_\beta)]\vert\Phi_0^A\rangle_\text{conn} \; ,
\end{equation}
where $\widehat H^I_1(t)$, $a^I_\alpha(t)$ and ${a_{\beta}^I}^\dag(t)$ are now intended as operators in the interaction picture with respect to $H_0$. 
The subscript ``conn'' implies that only \emph{connected} diagrams have to be considered when performing
the Wick contractions of the time-ordered product ${\pazocal T}$.  Each Wick contraction generates an uncorrelated single particle propagator, $g^{(0)}(\omega)$,  which is associated with the system governed by the Hamiltonian~$H_0$.
At order $n=0$, the expansion of Eq.~(\ref{gpert}) simply gives $g^{(0)}(\omega)$.
$H_1$ contains contributions from one-body, two-body and three-body interactions that come from the last three terms on the right hand side of Eq.~\eqref{eq:H}. Thus, for $n\geq1$ the expansion involves terms with individual contributions of each force, or combinations of them, that are linked by uncorrelated propagators.  To each term in the expansion there corresponds a Feynman diagram that gives an intuitive picture of the 
physical process accounted by its contribution. The full set of Feynman diagrammatic rules that stems out of Eq.~(\ref{gpert})
in the presence of three-body interactions is given in detail in Appendix 1.

A first reorganization of the contributions generated by Eq.~(\ref{gpert}) is obtained by considering 
\emph{one-particle reducible} diagrams, that is diagrams that can be disconnected by cutting a single fermionic line. 
In general, the reducible diagrams  generated by expansion~(\ref{gpert}) will always have separate structures that are linked together by only one $g^{(0)}(\omega)$ line. These are the same class of diagrams that are created implicitly in the all-orders resummation of the Dyson equation~(\ref{eq:Dyson}). 
%At higher orders, the fermion lines of $g^{(0)}(\omega)$ can enter connection several instances of $\Sigma^\star(\omega)$. 
 Thus, the irreducible self-energy $\Sigma^\star(\omega)$ is defined as the kernel that collects all the \emph{one-particle irreducible} (1PI) diagrams.
%
As already discussed above,  $\Sigma^\star(\omega)$ plays the role of an effective  potential that  is seen by a nucleon inside the system. It splits in static  and frequency dependent terms:
\begin{equation}
  \Sigma^\star_{\alpha \beta}(\omega) ~=~   - U_{\alpha \beta}  ~+~  \Sigma^{(\infty)}_{\alpha \beta}
      ~+~  \widetilde\Sigma_{\alpha \beta}(\omega)  \; ,
\label{eq:SigSplit}
\end{equation}
where we have separated the field $\widehat{U}$ since this is auxiliary defined and it eventually cancels out when solving the Dyson equation. The term $\Sigma^{(\infty)}_{\alpha \beta}$ plays the role of the static mean-field that a nucleon feels due to the average  interactions will all
other particles in the system.
The frequency-dependent part, $ \widetilde\Sigma_{\alpha \beta}(\omega)$, describes the effect due to dynamical excitations of the many-body state which are induced by the nucleon itself. In general, this means the  propagation of (complex) intermediate excitations and therefore it must have  
a Lehmann representation analogous to that of Eq.~\eqref{eq:g1Leh}.
For very large energies ($\omega \rightarrow \pm\infty$) the poles of such Lehmann representation become vanishingly  small and one is left with just $\Sigma^{(\infty)}$ and the auxiliary potential~$\widehat{U}$.


A further level of simplification in the self-energy expansion 
can be obtained if unperturbed propagators, $g^{(0)}(\omega)$, in the internal fermionic lines are replaced by dressed Green's functions, $g(\omega)$. 
This choice further restricts the set of diagrams to the so-called \emph{skeleton} diagrams \cite{ch11_Dickhoff2008}, which are
defined as 1PI diagrams that do not contain  any portion that can be disconnected by cutting any two fermion lines at different points. 
These portions would correspond to self-energy insertions, which are already re-summed into the dressed propagator $g(\omega)$ by Eq.~(\ref{eq:Dyson}).
The SCGF approach is precisely based on expressing the irreducible self-energy in terms of such skeleton diagrams 
with dressed propagators.
%
The SCGF framework offers great advantages. First, it is intrinsically nonperturbative and completely
independent from any choice of the reference state and auxiliary one-body potential. This is so because~$\widehat{U}$
automatically drops out of the Dyson equation (see Eq.~\eqref{eq:DysSchrod} below) and  $\Sigma^\star(\omega)$ no longer
depends on~$g^{(0)}(\omega)$.
Second, many-body correlations are expanded directly in terms of single particle excitations of the true propagator, which are closer
to the exact solution than those associated with the unperturbed state, $\vert\Phi_0^A\rangle$. 
Third, when a full SCGF calculation is possible it automatically satisfies the basic conservation laws~\cite{ch11_Baym1961,ch11_Baym1962,ch11_Dickhoff2008}. 
%
Finally, the number of diagrams to be considered is vastly reduced to 1PI skeletons one. However, this is not always a simplification since a dressed propagator contains a very large number of poles, which can be much more difficult to deal with than for  the corresponding uncorrelated~$g^{(0)}(\omega)$.

\begin{figure}[t]
\begin{center}
\includegraphics[width=0.95\textwidth]{Chapter11-figures/fig11_2_a.pdf}   
\vskip  0.4cm
\includegraphics[width=0.65\textwidth]{Chapter11-figures/fig11_2_b.pdf}  \hspace{0.30\textwidth} 
\caption{Graphical representation of the effective one-body interaction of Eq.~\eqref{eq:U_eff}, top row, and the effective two-body interaction~\eqref{eq:V_eff}, bottom row . Dashed lines represent the
one-, two,- and three-body interactions entering Eq.~\eqref{eq:H} and wavy lines are the effective operators $\widetilde U$ and  $\widetilde V$.}
\label{fig:EffOps}
\end{center}
\end{figure}


If three- or many-body forces are included in the Hamiltonian, the number of Feynman diagrams that need to be considered at a given order increases very rapidly. In this case it becomes very useful and instructive to restrict the attention to an even smaller class on diagrams that are  {\em interaction-irreducible}~\cite{ch11_Carbone2013Nov}. An interaction vertex is said to be reducible if the whole diagram can be disconnected in two parts by cutting the vertex itself. In general, this happens for an $m$-body interaction when there is a smaller number of $n$ lines ($n < m$) that leave the interaction, may interact only among themselves, and eventually all return to it. The net outcome is that one is left with a ($m-n$)-body operator that results from the average interactions with other $n$-spectator nucleons. This plays the role of a system dependent effective force that is irreducible. Fig.~\ref{fig:EffOps} shows diagrammatically how  $\widehat{V}$ and $\widehat{W}$  can be reduced to  one- and two-body {\em effective interactions} in this way.
%
Hence, for a system with up to three-body forces, we define an effective Hamiltonian
\begin{equation}
\widetilde H_1= {\widetilde U} + {\widetilde  V} + \widehat W \; ,
\label{eq:Heff}
\end{equation}
where $\widetilde U$ and  $\widetilde V$ are the effective interaction operators. 
The diagrammatic expansion arising from  Eq.~(\ref{gpert}) with the effective Hamiltonian $\widetilde H_1$ is
formed only of (1PI, skeleton) interaction-irreducible diagrams. Note that the three-body interaction, $\widehat W$, remains the same as in Eq.~\eqref{eq:H} but enters only diagrams as an  interaction-irreducible three-body force.  
%As long as {\em no} interaction-reducible diagrams are included it can be proved that the Hamiltonian~\eqref{eq:Heff}  does not generate any double counting and all symmetry factor are correct~\cite{ch11_Carbone2013Nov}.
%
The explicit expressions for the one-body and two-body effective interaction operators can be obtained form the Feynman diagrams of Fig.~\ref{fig:EffOps} and they are given by:
\begin{subequations}
\label{eq:UV_eff}
\begin{eqnarray}
  \label{eq:U_eff}
  \widetilde{U}_{\alpha\beta}&=& ~ ~ -U_{\alpha \beta}  ~ ~ +  ~\sum_{\delta\gamma} V_{\alpha\gamma,\beta\delta}~\rho_{\delta\gamma}
                        ~ +  ~ \frac{1}{4} \sum_{\mu \nu \gamma \delta}W_{\alpha\mu\nu,\beta\gamma\delta}
               %~\rho_{\gamma\mu}~\rho_{\delta\nu}  \; ,
                   ~\Gamma_{\gamma \delta, \mu \nu}  \; ,
  \\\label{eq:V_eff} 
  \widetilde{V}_{\alpha\beta,\gamma\delta} &=& ~ V_{\alpha\beta,\gamma\delta}  ~+ ~ 
                          \sum_{\mu\nu}W_{\alpha\beta \mu ,\gamma \delta \nu}~\rho_{\nu\mu}  \; .
\end{eqnarray}
\end{subequations}
where we used the reduced two-body density matrix $\Gamma$, which can be computed from the exact  two-body Green's function:
\begin{equation}
 \Gamma_{\gamma \delta, \mu \nu} = \lim_{\tau \rightarrow 0^-}  -i  \, G^{II}_{\gamma \delta, \mu \nu}(\tau) = 
     \langle\Psi^A_0\vert  a^\dagger_\nu a^\dagger_\mu  a_\gamma  a_\delta  \vert\Psi^A_0\rangle \; .
   \label{eq:2BDM}
\end{equation}


The effective Hamiltonian of Eq.~(\ref{eq:Heff})  not only regroups Feynman diagrams in 
a more efficient way, but also defines the effective one-body and two-body terms from 
higher order interactions. As long as interaction-irreducible diagrams are used together with the 
effective Hamiltonian, $\widetilde{H}_1$, this approach provides a systematic
way to incorporate many-body forces in the calculations and to 
generate effective in-medium interactions. More importantly, the formalism is such that all
symmetry factors are guaranteed to be correct and no diagram is over-counted~\cite{ch11_Carbone2013Nov}.
%
 Eqs.~\eqref{eq:UV_eff} can be seen as a generalization of the normal
ordering of the Hamiltonian with respect to the reference state $\vert\Phi_0^A\rangle$, which is discussed in chapter~8. However, these contractions go beyond normal ordering
because they are performed with respect to the exact
correlated density matrices. To some extent, one can intuitively think of 
the effective Hamiltonian $\widetilde{H}_1$  as being reordered with respect
to the interacting many-body ground-state $|\Psi_0^A\rangle$, rather than 
the non-interacting  $\vert\Phi_0^A\rangle$. 



\begin{figure}[t]
\begin{center}
\includegraphics[height=0.28\textwidth]{Chapter11-figures/fig11_3_a.pdf}   \hspace{0.06\textwidth} 
\includegraphics[height=0.28\textwidth]{Chapter11-figures/fig11_3_b.pdf}
\caption{Second order interaction-irreducible contributions to the self-energy arising from both two- and three-nucleon forces. The diagram depending on the effective two-body interactions (left) also shows the indices and labels that are used for calculating its contribution in {\bf Example 11.2}. }
\label{fig:2ndOrd}
\end{center}
\end{figure}

Since the static self-energy does not propagate any intermediate excitations, it can only receive contribution 
when the incoming and outgoing lines of a Feynman diagram are attached to the same interaction vertex.
Thus, by definition, $\Sigma^{(\infty)}$ must include  the one body term in $\widehat{H}_1$ plus any higher order 
interaction that are reduced to effective one-body interactions, hence:
\begin{equation}
   \widetilde{U}_{\alpha\beta} ~=~ - U_{\alpha \beta}  ~+~  \Sigma^{(\infty)}_{\alpha \beta} \; ,
   \label{eq:UeffSig}
\end{equation}
which defines $\Sigma^{(\infty)}$ by comparison with Eq.~\eqref{eq:U_eff}.
The two terms that contribute to $\Sigma^{(\infty)}$ represent extensions of the Hartree-Fock potentials to correlated 
ground states.  The correlated Hartree-Fock potential from ${\widehat V}$ is the only effective operator
when just two-body forces are present. In this case there is very little gain in using the concept of the effective Hamiltonian~\eqref{eq:Heff}.
However, with three-body interactions, additional effective interaction terms appear in both $\widetilde U$ and $\widetilde V$. 
%
From Eq.~\eqref{eq:UeffSig} we see that the perturbative SCGF expansion of the $\widetilde{H}_1$ Hamiltonian has only 
one (1PI, skeleton and interaction-irreducible)  term at first order.  The first contributions to $\widetilde\Sigma(\omega)$  appear at second order with the two diagrams in Fig.~\ref{fig:2ndOrd}. Expanding with respect to $\widehat{H}_1$, there would have been five diagrams instead of only the two interaction-irreducible ones shown in  Fig.~\ref{fig:2ndOrd}. These diagrams indeed have a proper Lehman representation (see Example 11.2 and Exercise 11.2) and propagate {\em intermediate state configurations} (ISCs) of type 2-particle 1-hole (2p1h), 2h1p, 3p2h, etc...
At third order, $\widetilde{H}_1$ generates 17 SCGF diagrams two of which contain only  two-body interactions. The simplest of these, that involve at most 2p1h and 2h1p ISCs, are shown in Fig.~\ref{fig:3rdOrd}.
%
All interaction-irreducible contributions to the proper self-energy up to third order in perturbation theory are discussed in details in Ref.~\cite{ch11_Carbone2013Nov}. 

\vskip .3 cm
\noindent
{\bf Example 11.1} Calculate the Feynman-Galitskii propagator, $G^{II,f}(\tau)$, that corresponds to the propagation of two particles or two holes that do not interact with each other.

\vskip 0.2 cm
This is the lowest order approximation to the two-times and two-body propagator
which evolves two particle from states $\alpha$ and $\beta$ to  states  $\gamma$ and  $\delta$ after a time $\tau >0$, or two holes  from $\gamma$ and  $\delta$ to $\alpha$ and $\beta$ when $\tau < 0$.  By applying the perturbative expansion equivalent to Eq.~\eqref{gpert} at order $n=0$, we find:
\begin{eqnarray}
 G^{II\, (0)}_{\alpha \beta, \gamma \delta}(\tau) &=& -i \langle\Phi_0^A\vert {\pazocal T} [
       \, a^I_\beta(\tau)  \, a^I_\alpha(\tau) \, {a_{\gamma}^I}^\dag(0) \, {a_{\delta}^I}^\dag(0) \, ]  \vert\Phi_0^A\rangle
\nonumber \\
 &=& i g^{(0)}_{\alpha\gamma}(\tau) \; \, g^{(0)}_{\beta\delta}(\tau)  ~-~ i  g^{(0)}_{\alpha\delta}(\tau) \; \, g^{(0)}_{\beta\gamma}(\tau)
 ~ \equiv ~  G^{II\, (0),f}_{\alpha \beta, \gamma \delta}(\tau)   - G^{II\, (0),f}_{\alpha \delta, \gamma \beta}(\tau) \; .
\label{eq:FeynGalvsG2}
\end{eqnarray}
The Feynman-Galitskii propagator is precisely defined as the non antisymmetrized part of Eq.~\eqref{eq:FeynGalvsG2}.  We  now transform this to frequency space and apply the Feynman rules of Appendix~1 %~\ref{app:Feyn_rules}
to calculate the  $G^{II,f}$ for the more general case of two {\em dressed} propagator lines:
\begin{eqnarray}
G^{II,f}_{\alpha \beta, \gamma \delta}(\omega) &=& \int d\tau \; e^{i\omega \tau} \, G^{II,f}_{\alpha \beta, \gamma \delta}(\tau)
~=~ (-i) \int \frac {d \omega_1}{2\pi} 
  \; i \,  g_{\alpha\gamma}(\omega-\omega_1) \; i \, g_{\beta\delta}(\omega_1)
\label{eq:GIIf_int} \\ \nonumber
&=& - \int \frac {d  \omega_1}{2\pi i}  
  \left\{ \frac{(\pazocal{X}^{n_1}_\alpha)^*  \pazocal{X}^{n_1}_\gamma}{\omega - \omega_1  - \varepsilon^+_{n_1} + i\eta} 
        + \frac{\pazocal{Y}^{k_1}_\alpha  (\pazocal{Y}^{k_1}_\gamma)^*}{\omega - \omega_1  - \varepsilon^-_{k_1} - i\eta}  \right\}
          \left\{ \frac{(\pazocal{X}^{n_2}_\beta)^*  \pazocal{X}^{n_2}_\delta}{\omega_1  - \varepsilon^+_{n_2} + i\eta} 
        + \frac{\pazocal{Y}^{k_2}_\beta  (\pazocal{Y}^{k_2}_\delta)^*}{\omega_1  - \varepsilon^-_{k_2} - i\eta}  \right\} \, ,
\end{eqnarray}
where we have  used the convention that repeated indices are summed over. The integral in the above equation can be performed with the Cauchy theorem by closing an arch on either the positive or the negative imaginary half planes. Hence, contributions where all the poles are  on the same side of the real axis cancel out. Extracting the residues of the other contributions leads to the following result:
\begin{equation}
G^{II,f}_{\alpha \beta, \gamma \delta}(\omega) =
\sum_{n_1, \, n_2} \frac{(\pazocal{X}^{n_1}_\alpha \pazocal{X}^{n_2}_\beta)^*  \pazocal{X}^{n_1}_\gamma \pazocal{X}^{n_2}_\delta}
                      {\omega  - (\varepsilon^+_{n_1}  + \varepsilon^+_{n_2}) + i\eta} 
~-~ \sum_{k_1, \, k_2} \frac{\pazocal{Y}^{k_1}_\alpha \pazocal{Y}^{k_2}_\beta \, (\pazocal{Y}^{k_1}_\gamma \pazocal{Y}^{k_2}_\delta)^*}
                     {\omega  - (\varepsilon^-_{k_1} + \varepsilon^-_{k_2}) - i\eta}   \; .
\label{eq:GIIf}
\end{equation}




\vskip 0.3 cm
\noindent
{\bf Exercise 11.1.} Calculate the contribution of the three-body force $\widehat W$ to the  effective one body potential, in the approximation of 
two dressed but non interacting spectator nucleons.

\vskip 0.2 cm
\noindent
{\bf Solution.}  This is the last term in Fig.~\ref{fig:EffOps}a) and Eq.~\eqref{eq:U_eff} but with $G^{II}(\tau)$ approximated by two independent fermion lines, as for the dressed Feynman-Galitskii propagator. Using Eq.~\eqref{eq:2BDM} and re-expressing the second line of~\eqref{eq:FeynGalvsG2} in terms of $g(\tau)$, we arrive at:
\begin{equation}
\label{eq:ueff_3b_first}
 \widetilde{U}^{(W)}_{\alpha\beta} ~=~ \frac{1}{2} \sum_{\mu \nu \gamma \delta}W_{\alpha\mu\nu,\beta\gamma\delta}
               ~\rho_{\gamma\mu}~\rho_{\delta\nu}  \; .
\end{equation}


\vskip 0.3 cm
\noindent
{\bf Example 11.2} Calculate the expression for the second order contribution to $\Sigma^\star(\omega)$ from two-nucleon interactions only.

\vskip 0.2 cm
This is the  diagram of Fig.~\ref{fig:2ndOrd}a). By applying the Feynman rules of Appendix 1 we have:
\begin{eqnarray}
 \Sigma^{(2,2N)}_{\alpha\beta}(\omega) &=&
  - \frac{(i)^2}{2} \int \frac {d \omega_1}{2\pi} \frac {d \omega_2}{2\pi}  V_{\alpha\sigma,\gamma\delta} 
         \,  g_{\gamma\mu}(\omega+\omega_2-\omega_1) \, g_{\delta\nu}(\omega_1) \, g_{\lambda\sigma}(\omega_2) 
           \, V_{\mu \nu, \beta \lambda}
\nonumber \\
   &=& + \frac{1}{2} \int \frac {d \omega_2}{2\pi i} V_{\alpha\sigma,\gamma\delta} \; 
     G^{II,f}_{\gamma \delta, \mu \nu}(\omega+\omega_2)   \, g_{\lambda\sigma}(\omega_2) 
           \, V_{\mu \nu, \beta \lambda}
\nonumber \\
  &=& \frac{1}{2}   \int \frac {d \omega_2}{2\pi i} \, V_{\alpha\sigma,\gamma\delta} \,
  \left\{
    \frac{(\pazocal{X}^{n_1}_\gamma \pazocal{X}^{n_2}_\delta)^*  \pazocal{X}^{n_1}_\mu \pazocal{X}^{n_2}_\nu}
                      {\omega  + \omega_2  - (\varepsilon^+_{n_1}  + \varepsilon^+_{n_2}) + i\eta} 
 -  \frac{ \pazocal{Y}^{k_1}_\gamma \pazocal{Y}^{k_2}_\delta \, (\pazocal{Y}^{k_1}_\mu \pazocal{Y}^{k_2}_\nu)^*}
                     {\omega  + \omega_2  - (\varepsilon^-_{k_1} + \varepsilon^-_{k_2}) - i\eta}
  \right\}
\nonumber \\
&& \qquad \qquad \quad \times  \left\{ \frac{(\pazocal{X}^{n_3}_\lambda )^* \pazocal{X}^{n_3}_\sigma}{\omega_2  - \varepsilon^+_{n_3} + i\eta} 
        + \frac{\pazocal{Y}^{k_3}_\lambda  (\pazocal{Y}^{k_3}_\sigma)^*}{\omega_2  - \varepsilon^-_{k_3} - i\eta}  \right\} 
          \, V_{\mu \nu, \beta \lambda}
 \; ,
\end{eqnarray}
where we have used the two-body interaction $\widehat V$, but it could have been equally calculated with the two-body effective interaction $\widetilde V$. 
Note that the  integration over $d \omega_1$ is exactly the same as in Eq.~\eqref{eq:GIIf_int}. Thus, we can directly substitute the expression for the Feynman-Galitskii propagator~\eqref{eq:GIIf}  in the last two lines above. By performing the last Cauchy integral we find that only two out of four possible terms survive. The final result for the second order irreducible self-energy is: 
\begin{equation}
\Sigma^{(2,2N)}_{\alpha\beta}(\omega) = \frac{1}{2}  V_{\alpha\sigma,\gamma\delta} \left\{
   \sum_{\substack{n_1, \, n_2 \\  k_3}} \frac{(\pazocal{X}^{n_1}_\gamma   \pazocal{X}^{n_2}_\delta    \pazocal{Y}^{k_3}_\sigma)^*
                                                      \pazocal{X}^{n_1}_\mu \pazocal{X}^{n_2}_\nu  \pazocal{Y}^{k_3}_\lambda}
                      {\omega  - (\varepsilon^+_{n_1}  + \varepsilon^+_{n_2}- \varepsilon^-_{k_3}) + i\eta} 
+ \sum_{\substack{k_1, \, k_2 \\ n_3}} \frac{\pazocal{Y}^{k_1}_\gamma     \pazocal{Y}^{k_2}_\delta \pazocal{X}^{n_3}_\sigma \, 
                                                       (\pazocal{Y}^{k_1}_\mu \pazocal{Y}^{k_2}_\nu \pazocal{X}^{n_3}_\lambda )^*}
                     {\omega  - (\varepsilon^-_{k_1} + \varepsilon^-_{k_2}- \varepsilon^+_{n_3}) - i\eta}  
       \right\}  V_{\mu \nu, \beta \lambda} \; ,
\label{eq:Sig_2nd}
\end{equation}
where repeated greek indices are summed over implicitly but we show the explicit summation over the poles corresponding to 2p1h and 2h1p ISCs.



\vskip 0.3 cm
\noindent
{\bf Exercise 11.2.} Calculate the expression for the  other second order contribution to $\Sigma^\star(\omega)$ arising from three-nucleon interactions (diagram of Fig.~\ref{fig:2ndOrd}b). Show that this contains ISCs of 3p2h and 3h2p.

\vskip 0.2 cm
\noindent
{\bf Solution.} Upon performing the four frequency integrals, one obtains:
\begin{eqnarray}
\Sigma^{(2,3N)}_{\alpha\beta}(\omega) &=& \frac{1}{12}  W_{\alpha\gamma\delta, \mu\nu\lambda} \left\{
   \sum_{\substack{n_1, \, n_2 ,\, n_3 \\  k_4,\, k_5}} 
       \frac{(\pazocal{X}^{n_1}_\mu   \pazocal{X}^{n_2}_\nu   \pazocal{X}^{n_3}_\lambda    \pazocal{Y}^{k_4}_\gamma  \pazocal{Y}^{k_5}_\delta)^*
                 \pazocal{X}^{n_1}_{\mu'} \pazocal{X}^{n_2}_{\nu'}  \pazocal{X}^{n_3}_{\lambda'}    \pazocal{Y}^{k_4}_{\gamma'} \pazocal{Y}^{k_5}_{\delta'}}
                      {\omega  - (\varepsilon^+_{n_1}  + \varepsilon^+_{n_2} + \varepsilon^+_{n_3}- \varepsilon^-_{k_4}- \varepsilon^-_{k_5}) + i\eta} 
 \right. \nonumber \\
 &&\qquad \qquad \quad  \left. +
\sum_{\substack{k_1, \, k_2 ,\, k_3 \\ n_4 ,\, n_5}} 
         \frac{\pazocal{Y}^{k_1}_\mu     \pazocal{Y}^{k_2}_\nu    \pazocal{Y}^{k_3}_\lambda \pazocal{X}^{n_4}_\gamma \pazocal{X}^{n_5}_\delta \, 
               (\pazocal{Y}^{k_1}_{\mu'} \pazocal{Y}^{k_2}_{\nu'}     \pazocal{Y}^{k_3}_{\lambda'} \pazocal{X}^{n_4}_{\gamma'} \pazocal{X}^{n_5}_{\delta'} )^*}
                     {\omega  - (\varepsilon^-_{k_1} + \varepsilon^-_{k_2}+ \varepsilon^-_{k_3}- \varepsilon^+_{n_4}- \varepsilon^+_{n_5}) - i\eta}  
       \right\}  W_{\mu' \nu' \lambda' , \beta \gamma' \delta'} \, . \qquad \qquad
\label{eq:Sig_2nd_3b}
\end{eqnarray}



\section{The Algebraic Diagrammatic Construction method}
\label{sec:scgf_adc}

The most general form of the irreducible self-energy is given by Eq.~\eqref{eq:SigSplit}. The $\Sigma^{(\infty)}$  is defined by the mean-field diagrams of Fig.~\ref{fig:EffOps}a) and Eq.~\eqref{eq:U_eff}, while $\widetilde\Sigma(\omega)$ has a Lehmann representation as seen in the examples of Eqs.~\eqref{eq:Sig_2nd} and~\eqref{eq:Sig_2nd_3b}. Similarly to the case of a propagator, the  pole structure of the energy-dependent part is dictated by the principle of causality with the correct boundary conditions coded by the  $\pm i\eta$ terms at the denominators.  This implies a dispersion relation that  can link the real and imaginary parts of the self-energy~\cite{ch11_MahauxSartor91,ch11_Dickhoff2008}.  Correspondingly, the direct coupling of single particle orbits to  ISCs (of 2p1h and 2h1p character or more complex) imposes  the separable structure of the residues. In this section we consider the case of a  finite system, for which it is useful to use a discretized single particle basis $\{\alpha\}$ as the model space. 
%
From now on we will use the Einstein convention that repeated indices ($n$, $k$, $\alpha$...) are summed over even if not explicitly stated.
%
Thus, the above constraints impose the following  analytical form the the self-energy operator:
\begin{equation}
\Sigma^{(\star)}_{\alpha\beta}(\omega) =  - U_{\alpha \beta}  ~+~ \Sigma^{(\infty)}_{\alpha\beta} ~+~ M^\dagger_{\alpha, r}\frac1{\omega - [E^> +C]_{r,r'} + i \eta}M_{r', \beta} 
       ~+~ N_{\alpha, s}\frac1{\omega - [E^< +D]_{s,s'} - i \eta}N^\dagger_{s', \beta}  \; ,
\label{eq:ADC_SE_form}
\end{equation}
where the $E^>$ and $E^<$ are the unperturbed energies for the forward and backward ISCs and $r$ and $s$ are collective indices that label sets of configurations beyond single particle structure. Specifically, $r$ is for particle addition and will label 2p1h, 3p2h, 4p3h, ... states, in the general case. Likewise, $s$ is for particle removal and we will use it to label 2h1p states (or higher configurations). However, for the approximations presented in this chapter and for our discussion below we will only be limited to 2p1h and 2h1p ISCs.

The expansion of the self-energy at second order in perturbation theory trivially satisfies Eq.~\eqref{eq:ADC_SE_form}. In the results  of Eq.~\eqref{eq:Sig_2nd}, the sums over $r$ and $s$ can be taken to run over ordered configurations $r\equiv\{n_1 < n_2,k_3\}$ and  $s\equiv\{k_1 < k_2, n_3\}$. Because of the Pauli principle, the residues of each pole are antisymmetric with respect to exchanging two quasiparticle or two quasihole indices. Therefore the constraints  $n_1 < n_2$ and  $k_1 < k_2$  can be imposed to avoid counting the same configurations twice. Thus, we can identify the expressions for the residues and poles as follows:
\begin{subequations}
 \label{eq:ADC2_MEC}
\begin{align}
 M_{\alpha,r} ={}& %\sum_{\mu \nu \lambda}
     \pazocal{X}^{n_1}_\mu \pazocal{X}^{n_2}_\nu  \pazocal{Y}^{k_3}_\lambda  V_{\mu \nu, \beta \lambda}
  \label{eq:ADC2_M}  \\
 E^>_{r,r'} ={}& diag \left\{ \,\varepsilon^+_{n_1} + \varepsilon^+_{n_2} - \varepsilon^-_{k_3}  \, \right\}
 \label{eq:ADC2_Efw}  \\
 C_{r,r'} ={}& 0
 \label{eq:ADC2_C}
 \end{align}
\end{subequations}
and 
\begin{subequations}
\label{eq:ADC2_NED}
\begin{align}
 N_{\alpha,s}  ={}& \pazocal{Y}^{k_1}_\mu \pazocal{Y}^{k_2}_\nu \pazocal{X}^{n_3}_\lambda  V_{\mu \nu, \beta \lambda}
   \label{eq:ADC2_N}  \\
 E^<_{s,s'}  ={}& diag \left\{ \, \varepsilon^-_{k_1} + \varepsilon^-_{k_2} - \varepsilon^+_{n_3} \, \right\} 
   \label{eq:ADC2_Ebk}  \\
 D_{s,s'}  ={}& 0  \; ,
  \label{eq:ADC2_D}
\end{align}
\end{subequations}
where the factor $1/2$ from Eq.~\eqref{eq:Sig_2nd} disappears since we restrict the sums to triplets of indices where  $n_1<n_2$ and $k_1<k_2$.
As we will discuss in the next section, Eqs.~\eqref{eq:ADC2_MEC} and \eqref{eq:ADC2_NED} define the algebraic diagrammatic method at second order [ADC(2)].


\begin{figure}[t]
\begin{center}
\includegraphics[height=0.18\textheight]{Chapter11-figures/fig11_4_a.pdf}   \hspace{0.03\textwidth} 
\includegraphics[height=0.18\textheight]{Chapter11-figures/fig11_4_b.pdf}   \hspace{0.03\textwidth} 
\includegraphics[height=0.18\textheight]{Chapter11-figures/fig11_4_c.pdf}  
\caption{The three simplest skeleton and interaction-irreducible diagrams contributing to the self-energy at third order.  All these terms involve intermediate state configurations of at most 2p1h and 2h1p. The first two contain only two-nucleon interactions and represent the first contribution to the  resummation of ladders [diagram a)] and rings [diagram b)]. The diagram c) is the first term containing an irreducible three-nucleon interaction. The remaining 14 diagrams at third order all require explicit three-body interactions and ISCs of 3p2h and 3h2p~\cite{ch11_Carbone2013Nov,ch11_Raimondi_inprep}.  }
\label{fig:3rdOrd}
\end{center}
\end{figure}

Unfortunately, $\Sigma^\star(\omega)$ loses its analytical form of Eq.~\eqref{eq:ADC_SE_form} as soon as one moves to higher orders in perturbation theory. To demonstrate this, let us calculate the contribution of the third order 'ladder' diagram of Fig.~\ref{fig:3rdOrd}a). By exploiting Feynman rules
and Eq.~\eqref{eq:GIIf_int} we obtain
\begin{align}
  \Sigma^{(3,ld)}_{\alpha\beta}(\omega)={}& - \frac{i^3}{4} \int \frac{d\omega_1}{2\pi} \int \frac{d\omega_2}{2\pi} \int \frac{d\omega_3}{2\pi} 
  V_{\alpha\sigma,\gamma\delta} 
         \,  g_{\gamma\gamma'}(\omega+\omega_3-\omega_1) \, g_{\delta\delta'}(\omega_1) 
           \, V_{\gamma' \delta', \mu' \nu'}
  \nonumber \\ && \qquad \qquad \qquad \qquad \qquad \qquad \qquad \qquad \times
    \,  g_{\mu'\mu}(\omega+\omega_3-\omega_2) \, g_{\nu'\nu}(\omega_2) \, V_{\mu \nu, \beta \lambda} \,  g_{\lambda\sigma}(\omega_3) 
\nonumber \\
  ={}&  \frac{1}{4}  \int \frac{d\omega_3}{2\pi i} 
  V_{\alpha\sigma,\gamma\delta} 
         \,  G^{II,f}_{\gamma\delta, \gamma' \delta'}(\omega+\omega_3) \,  
           \, V_{\gamma' \delta', \mu' \nu'}     \,  G^{II,f}_{\mu' \nu', \mu \nu}(\omega+\omega_3) \, V_{\mu \nu, \beta \lambda}  \, g_{\lambda\sigma}(\omega_3) 
\nonumber \\
  ={}&  \frac{1}{4}  \int \frac{d\omega_3}{2\pi i} 
    V_{\alpha\sigma,\gamma\delta} \,
  \left\{
    \frac{(\pazocal{X}^{n_1}_\gamma \pazocal{X}^{n_2}_\delta)^*  \pazocal{X}^{n_1}_{\gamma'} \pazocal{X}^{n_2}_{\delta'}}
                      {\omega+\omega_3  - (\varepsilon^+_{n_1}  + \varepsilon^+_{n_2}) + i\eta} 
 -  \frac{ \pazocal{Y}^{k_1}_\gamma \pazocal{Y}^{k_2}_\delta \, ( \pazocal{Y}^{k_1}_{\gamma'} \pazocal{Y}^{k_2}_{\delta'})^*}
                     {\omega+\omega_3  - (\varepsilon^-_{k_1} + \varepsilon^-_{k_2}) - i\eta}
  \right\}
\nonumber \\
  & \qquad \qquad \times    \, V_{\gamma' \delta', \mu' \nu'} \,  \left\{
    \frac{( \pazocal{X}^{n_4}_{\mu'} \pazocal{X}^{n_5}_{\nu'})^*  \pazocal{X}^{n_4}_\mu \pazocal{X}^{n_5}_\nu}
                      {\omega+\omega_3  - (\varepsilon^+_{n_4}  + \varepsilon^+_{n_5}) + i\eta} 
 -  \frac{ \pazocal{Y}^{k_4}_{\mu'} \pazocal{Y}^{k_5}_{\nu'} \, (\pazocal{Y}^{k_4}_\mu \pazocal{Y}^{k_5}_\nu)^*}
                     {\omega+\omega_3  - (\varepsilon^-_{k_4} + \varepsilon^-_{k_5}) - i\eta}
  \right\}
 \nonumber \\
  & \qquad \qquad \qquad \qquad \times   V_{\mu \nu, \beta \lambda}
   \left\{ \frac{(\pazocal{X}^{n_3}_\lambda )^* \pazocal{X}^{n_3}_\sigma}{\omega_3  - \varepsilon^+_{n_3} + i\eta} 
           + \frac{\pazocal{Y}^{k_3}_\lambda  (\pazocal{Y}^{k_3}_\sigma)^*}{\omega_3  - \varepsilon^-_{k_3} - i\eta}  \right\}   \; .
\label{eq:LaddEg1}
 \end{align}
Performing the Cauchy integrals, only six terms out of the eight combinations of poles survive. To simplify the discussion 
we will focus only on the three integrals that contribute to the forward propagation of the self-energy (third term on the r.h.s.
 of \eqref{eq:ADC_SE_form}). This is done by retaining only the poles $(\omega_3  - \varepsilon^-_{k_3} - i\eta)^{-1}$ in the
 last propagator of Eq.~\eqref{eq:LaddEg1}, which lie above the real axis with respect to the integrand $\omega_3$. Thus,
 we have:
 \begin{align}
  \Sigma^{(ld,>)}_{\alpha\beta}(\omega) ={}& 
 \frac{1}{4}  \int \frac{d\omega_3}{2\pi i} 
    V_{\alpha\sigma,\gamma\delta} \,
  \left\{
 -  \frac{ \pazocal{Y}^{k_1}_\gamma \pazocal{Y}^{k_2}_\delta \, ( \pazocal{Y}^{k_1}_{\gamma'} \pazocal{Y}^{k_2}_{\delta'})^*}
                     {\omega+\omega_3  - (\varepsilon^-_{k_1} + \varepsilon^-_{k_2}) - i\eta}
  \right\}
\nonumber \\
 & \qquad \qquad \times
   \, V_{\gamma' \delta', \mu' \nu'} \,  \left\{
    \frac{( \pazocal{X}^{n_4}_{\mu'} \pazocal{X}^{n_5}_{\nu'})^*  \pazocal{X}^{n_4}_\mu \pazocal{X}^{n_5}_\nu}
                      {\omega+\omega_3  - (\varepsilon^+_{n_4}  + \varepsilon^+_{n_5}) + i\eta} 
  \right\}
  V_{\mu \nu, \beta \lambda}
   \left\{  \frac{\pazocal{Y}^{k_3}_\lambda  (\pazocal{Y}^{k_3}_\sigma)^*}{\omega_3  - \varepsilon^-_{k_3} - i\eta}  \right\}
     \nonumber \\
     &+
 \frac{1}{4}  \int \frac{d\omega_3}{2\pi i} 
    V_{\alpha\sigma,\gamma\delta} \,
  \left\{
    \frac{(\pazocal{X}^{n_1}_\gamma \pazocal{X}^{n_2}_\delta)^*  \pazocal{X}^{n_1}_{\gamma'} \pazocal{X}^{n_2}_{\delta'}}
                      {\omega+\omega_3  - (\varepsilon^+_{n_1}  + \varepsilon^+_{n_2}) + i\eta} 
  \right\}
\nonumber \\
 & \qquad \qquad \times
   \, V_{\gamma' \delta', \mu' \nu'} \,  \left\{
 -  \frac{ \pazocal{Y}^{k_4}_{\mu'} \pazocal{Y}^{k_5}_{\nu'} \, (\pazocal{Y}^{k_4}_\mu \pazocal{Y}^{k_5}_\nu)^*}
                     {\omega+\omega_3  - (\varepsilon^-_{k_4} + \varepsilon^-_{k_5}) - i\eta}
  \right\}
  V_{\mu \nu, \beta \lambda}
   \left\{  \frac{\pazocal{Y}^{k_3}_\lambda  (\pazocal{Y}^{k_3}_\sigma)^*}{\omega_3  - \varepsilon^-_{k_3} - i\eta}  \right\}
     \nonumber \\
 &+
 \frac{1}{4}  \int \frac{d\omega_3}{2\pi i} 
    V_{\alpha\sigma,\gamma\delta} \,
  \left\{
    \frac{(\pazocal{X}^{n_1}_\gamma \pazocal{X}^{n_2}_\delta)^*  \pazocal{X}^{n_1}_{\gamma'} \pazocal{X}^{n_2}_{\delta'}}
                      {\omega+\omega_3  - (\varepsilon^+_{n_1}  + \varepsilon^+_{n_2}) + i\eta} 
  \right\}
\nonumber \\
 & \qquad \qquad \times
   \, V_{\gamma' \delta', \mu' \nu'} \,  \left\{
    \frac{( \pazocal{X}^{n_4}_{\mu'} \pazocal{X}^{n_5}_{\nu'})^*  \pazocal{X}^{n_4}_\mu \pazocal{X}^{n_5}_\nu}
                      {\omega+\omega_3  - (\varepsilon^+_{n_4}  + \varepsilon^+_{n_5}) + i\eta} 
  \right\}
  V_{\mu \nu, \beta \lambda}
   \left\{  \frac{\pazocal{Y}^{k_3}_\lambda  (\pazocal{Y}^{k_3}_\sigma)^*}{\omega_3  - \varepsilon^-_{k_3} - i\eta}  \right\}
\nonumber \\
%
%
%
\nonumber \\
  ={}&   
   \frac{  \frac{1}{2} V_{\alpha\sigma,\gamma\delta} \,  \pazocal{Y}^{k_1}_\gamma \pazocal{Y}^{k_2}_\delta 
     ~  ( \pazocal{Y}^{k_1}_{\gamma'} \pazocal{Y}^{k_2}_{\delta'})^* \;  V_{\gamma' \delta', \mu' \nu'}  \;
        ( \pazocal{X}^{n_4}_{\mu'} \pazocal{X}^{n_5}_{\nu'} \pazocal{Y}^{k_3}_\sigma)^* }
                     {[\varepsilon^-_{k_1} + \varepsilon^-_{k_2} -\varepsilon^+_{n_4}  - \varepsilon^+_{n_5} ]}
    \frac{1}{2}
    \frac{  \pazocal{X}^{n_4}_\mu \pazocal{X}^{n_5}_\nu \pazocal{Y}^{k_3}_\lambda}
                      {\omega  - (\varepsilon^+_{n_4}  + \varepsilon^+_{n_5}  - \varepsilon^-_{k_3}) + i\eta} 
  V_{\mu \nu, \beta \lambda}
      \nonumber \\
 &+
 V_{\alpha\sigma,\gamma\delta} \,
    \frac{(\pazocal{X}^{n_1}_\gamma \pazocal{X}^{n_2}_\delta \pazocal{Y}^{k_3}_\sigma)^* }
                      {\omega  - (\varepsilon^+_{n_1}  + \varepsilon^+_{n_2}  - \varepsilon^-_{k_3}) + i\eta} 
                 \frac{1}{2} 
 \frac{   \pazocal{Y}^{k_3}_\lambda \pazocal{X}^{n_1}_{\gamma'} \pazocal{X}^{n_2}_{\delta'} \;  V_{\gamma' \delta', \mu' \nu'} \;
 \pazocal{Y}^{k_4}_{\mu'} \pazocal{Y}^{k_5}_{\nu'} \, (\pazocal{Y}^{k_4}_\mu \pazocal{Y}^{k_5}_\nu)^*  \frac{1}{2} V_{\mu \nu, \beta \lambda} }
                  {[\varepsilon^-_{k_4} + \varepsilon^-_{k_5} - \varepsilon^+_{n_1}  - \varepsilon^+_{n_2}]}
\nonumber \\
  &+
    \frac{     V_{\alpha\sigma,\gamma\delta}  ~ (\pazocal{X}^{n_1}_\gamma \pazocal{X}^{n_2}_\delta \pazocal{Y}^{k_3}_\sigma)^* }
                      {\omega  - (\varepsilon^+_{n_1}  + \varepsilon^+_{n_2} - \varepsilon^-_{k_3}) + i\eta} 
 \frac{1}{2}  
   \pazocal{X}^{n_1}_{\gamma'} \pazocal{X}^{n_2}_{\delta'}  \, V_{\gamma' \delta', \mu' \nu'} \, ( \pazocal{X}^{n_4}_{\mu'} \pazocal{X}^{n_5}_{\nu'})^*
 \frac{1}{2}  
    \frac{  \pazocal{X}^{n_4}_\mu \pazocal{X}^{n_5}_\nu \pazocal{Y}^{k_3}_\lambda  ~ V_{\mu \nu, \beta \lambda}}
                      {\omega  - (\varepsilon^+_{n_4}  + \varepsilon^+_{n_5} - \varepsilon^-_{k_3}) + i\eta} 
 \nonumber \\
%
%
%
\nonumber \\
  \equiv{}& \quad M^{(2,ld) \, \dagger}\frac1{\omega - E^> + i \eta}M^{(1)}
  \nonumber \\
    &+~  M^{(1) \, \dagger}\frac1{\omega - E^> + i \eta}M^{(2,ld)}
  \nonumber \\
    & +~  M^{(1) \, \dagger}\frac1{\omega - E^> + i \eta}C^{(ld)}\frac1{\omega - E^> + i \eta}M^{(1)}  \; ,
\label{eq:LaddEg2}
 \end{align}
where $M^{(1)}$ and $E^>$ are the same as in Eqs.~\eqref{eq:ADC2_MEC}. The 2p1h ladder interaction $C^{(ld)}$ is at first order in $V$, while the coupling matrix $ M^{(2,ld)}$ is at second order. These can be read from the previous lines of Eq.~\eqref{eq:LaddEg2} and turn out to be (showing all summations explicitly):
\begin{align}
M^{(2,ld)}_{r,\alpha} ={}&  \sum_{k_4, \, k_5} \quad \sum_{\substack{ \alpha ,\, \beta ,\, \gamma, \,  \delta \\ \mu ,\, \nu , \, \lambda}} 
  \frac{   \pazocal{X}^{n_1}_{\gamma} \pazocal{X}^{n_2}_{\delta} \;  V_{\gamma \delta, \sigma \zeta} \;
 \pazocal{Y}^{k_4}_{\sigma} \pazocal{Y}^{k_5}_{\zeta} \, (\pazocal{Y}^{k_4}_\mu \pazocal{Y}^{k_5}_\nu)^*  \pazocal{Y}^{k_3}_\lambda  }
                  {[\varepsilon^-_{k_4} + \varepsilon^-_{k_5} - \varepsilon^+_{n_1}  - \varepsilon^+_{n_2}]} \frac{1}{2} V_{\mu \nu, \alpha \lambda}
  \nonumber \\
  C^{(ld)}_{r,r'} ={}&  \sum_{ \alpha ,\, \beta ,\, \gamma ,\, \delta} \pazocal{X}^{n_1}_{\alpha} \pazocal{X}^{n_2}_{\beta}  \, V_{\alpha \beta, \gamma \delta} \, ( \pazocal{X}^{n_1'}_{\gamma} \pazocal{X}^{n_2'}_{\delta})^* \; \delta_{k_3, k_3'} \; .
\end{align}

Eq.~\eqref{eq:LaddEg2}  clearly breaks the known Lehmann representation for the self-energy and would even lead to inconsistent 
results unless its contribution is very small compared to the second order contribution of Eq.~\eqref{eq:Sig_2nd}. This means that Eq.~\eqref{eq:LaddEg2} would
invalidate the perturbative expansion unless $V$ is small.
Therefore, we need to identify proper corrections that allow to retain these third order contributions but at the same time let
us recover the correct analytical form~\eqref{eq:ADC_SE_form}.  For the first two terms on the right hand side of Eq.~\eqref{eq:LaddEg2}, this
issue can be easily solved by remembering that the corresponding diagram from $\Sigma^{(2)}(\omega)$ (see Eq.~\eqref{eq:Sig_2nd}) is to be included.
If then one adds an extra term that is quadratic in  $M^{(2,ld)}$, this leads to:
\begin{equation}
  \Sigma^{(2)}(\omega)  + \Sigma^{(3, ld)}(\omega) + M^{(2,ld) \, \dagger}\frac 1 {\omega  - E^> + i\eta} M^{(2,ld)}
 % \nonumber \\
 \longrightarrow  \left[ M^{(1)} +  M^{(2,ld)}\right]^\dagger \frac1{\omega  - E^> + i\eta}\left[ M^{(1)} + M^{(2,ld)}\right]  \, ,
\end{equation}
which resolves the issue of obtaining the residues in separable form. Note that this new correction is just one specific Goldstone diagram among
the many that contribute to the self-energy at {\em fourth order}. On the other hand, adding all of the fourth order diagrams would lead to 
new terms that break the Lehmann representation themselves and that in turn would call for the inclusions of selected Goldstone terms at even higher orders.  In other words, we have achieved to recover the structure of Eq.~\eqref{eq:ADC_SE_form}  but at the price of giving up a systematic 
perturbative expansion that is complete at each order in $\widetilde V$. Given that the Lehmann representation is dictated by physical properties, this is 
a more satisfactory rearrangement of the perturbation series.

The last term in Eq.~\eqref{eq:LaddEg2} is more tricky to correct since it contains second order poles as $(\omega  - E - i\eta)^{-2}$, which cannot be 
canceled by single contributions at higher order. Instead, we are forced to perform a non-perturbative resummation of Goldstone diagrams to all orders that results in a geometric series. This is done by considering the relation
\begin{equation}
 \frac1{A-B}  ~=~  \frac1 A ~+~ \frac1 A  \, B \,  \frac1{A-B}  ~=~  \frac1 A  ~+~  \frac1 A  \, B \,  \frac1 A  ~+~  \frac1 A  \, B \,   \frac1 A   \, B \,   \frac1 A
             ~+~ \frac1 A  \, B \,  \frac1 A  \, B \,  \frac1 A  \, B \,  \frac1 A  ~+ \ldots
  \label{eq:1overAB}
\end{equation}
for two operators $A$ and $B$. If we chose  $A\equiv{\omega  - E^> + i\eta}$ and $B\equiv C^{(ld)}$,  the first and second term on the right hand 
side can then be  identified respectively with the contribution from $\Sigma^{(2)}(\omega)$ and the  last term of Eq.~\eqref{eq:LaddEg2}. 
Also in this case, all perturbative terms up to third order have been kept unchanged  but we are forced to select a series of Goldstone diagrams up to infinite order.   
%

If then one adds an extra term that is quadratic in  $M^{(2,ld)}$, this lead to:
\begin{equation}
  \Sigma^{(2)}(\omega)  ~+~ \Sigma^{(3, ld)}(\omega) ~+~ \begin{array}{c} \hbox{terms beyond} \\ \hbox{ 3$^{rd}$ order } \end{array} 
 \longrightarrow  \left[ M^{(1)} +  M^{(2,ld)}\right]^\dagger \frac1{\omega  - E^> - C^{(ld)} + i\eta}\left[ M^{(1)} + M^{(2,ld)}\right]  \ ,
%  \nonumber 
 \label{eq:pp_ladder}
\end{equation}
which now contains {\em all}  the perturbation theory terms at second~\eqref{eq:Sig_2nd} and third order~\eqref{eq:LaddEg2} while preserving the expected analytical form for~$\widetilde\Sigma(\omega)$ of Eq.~\eqref{eq:ADC_SE_form}.

It can be shown that the summation implicit in Eq.~\eqref{eq:pp_ladder} is equivalent to a full resummation of two-particle ladder diagrams in the Tamn-Dancoff approximation (TDA)~\cite{ch11_RingSchuck}. In this sum the remaining quasi hole state appearing in the 2p1h ISC remains uncoupled from the ladder series, as it can be seen from the first term in the series, in Fig.~\ref{fig:3rdOrd}a).  Likewise, one would find that the remaining backward-going terms in Eq.~\eqref{eq:LaddEg1} would lead to resumming the two-hole TDA ladders within the 2h1p configurations. Instead, diagram in Fig.~\ref{fig:3rdOrd}b) involves a resummation of ph ring diagrams. Extensions of these series to random-phase approximation (RPA) is also possible, this would introduce a larger set of high-order Goldstone diagrams but it would not be required to enforce consistency with perturbation theory at third-order.

\vskip 0.3 cm
\noindent
{\bf Exercise 11.3.}  Complete  the calculation of Eq.~\eqref{eq:LaddEg1} and derive the remaining corrections to the 2h1p interaction $D^{(ld)}$  and the 1h-2h1p coupling term  $N^{(2,ld)}$.


\subsection{The ADC($n$) approach and working equations at third-order}

 The  procedure discussed above to devise reliable approximations for the self-energy is at the heart of the
 ADC method, originally introduced by J.~Schirmer and collaborators~\cite{ch11_Schirmer1982ADC2,ch11_Schirmer1983ADCn}.
 This approach  generates  a hierarchy of approximations of increasing accuracy such that,
 at a given order $n$, the ADC($n$) equations will maintain the analytic form of Eq.~\eqref{eq:ADC_SE_form} and
 also contain the full Feynman expansion for $\Sigma^\star(\omega)$ up to order~$n$.
 %
To do this, we expand the Lehmann representation of $\widetilde\Sigma(\omega)$ in powers of the perturbation interaction $\widehat{H}_1$ (or, equivalently, $\widetilde{H}_1$). The interaction matrices 
$C$ and $D$ appearing in the denominators of Eq.~\eqref{eq:ADC_SE_form} can only be of first order in either $\widehat{U}$,  $\widehat{V}$ or $\widehat{W}$. However, the coupling matrices can contain terms of any order:
\begin{align}
  M =& M^{(1)} +  M^{(2)} +  M^{(3)} +  \ldots
   \nonumber  \\
  N =&  N^{(1)} +  N^{(2)} +  N^{(3)} + \ldots  
 \label{eq:MNexp}
\end{align}
Using  Eqs.~\eqref{eq:1overAB}  and~\eqref{eq:MNexp} one finds the following expansion
for Eq.~\eqref{eq:ADC_SE_form}:
\begin{align}
  \Sigma^{\star}(\omega) ={}&  - \widehat{U}  ~+~   \Sigma^{(\infty)} 
  \nonumber \\
   +~& M^{(1) \, \dagger}\frac1{\omega - E^> -C + i \eta}M^{(1)} +N^{(1)}\frac1{\omega - E^< -D - i \eta}N^{(1) \, \dagger}
    \nonumber \\
  +~& M^{(2) \, \dagger}\frac1{\omega - E^> + i \eta}M^{(1)}  +   M^{(1) \, \dagger}\frac1{\omega - E^> + i \eta}M^{(2)}
%  \nonumber \\
%   &&
 +{}  M^{(1) \, \dagger}\frac1{\omega - E^> + i \eta} C \frac1{\omega - E^> + i \eta}M^{(1)}  
    \nonumber \\
  +~& N^{(2)}\frac1{\omega - E^< - i \eta}N^{(1) \, \dagger} ~ +   N^{(1)}\frac1{\omega - E^< - i \eta}N^{(2) \, \dagger}
%  \nonumber \\
%   &&
 ~+{}  N^{(1)}\frac1{\omega - E^< - i \eta} D \frac1{\omega - E^< - i \eta}N^{(1) \, \dagger}  
 \nonumber \\
 +~&   {\pazocal O}({ \widehat{H}_1}^4)  \; ,
 \label{eq:ADC_SE_form_exp}
\end{align}
where all terms up to third order in $\widehat H_1$ are shown explicitly.
The ADC procedure is then to simply calculate all possible diagrams up
to order $n$. By comparing these results to Eq.~\eqref{eq:ADC_SE_form_exp}, one then reads
the minimum expressions for the coupling and interaction matrices, $M$, $N$, $C$ and $D$ that are needed to
include all the $n$-order diagrams. Correspondingly, the energy-independent self-energy $\Sigma^{(\infty)}$
needs to be computed at least up to order $n$ as well.
 Note that the dynamic part of the self-energy, which propagates ISCs, appears only starting from the
 second order. This is so because any such diagram needs one perturbing interaction $V$ to generate an ISC
 and a second one to annihilate it back to a single particle state. In general, if the Hamiltonian contains
 up to $m$-body forces and $i$ is an integer, then the ADC($2i$) and ADC($2i+1$) will require
 ISCs up to  ($k$+1)-particle--$k$-hole and ($k$+1)-hole--$k$-particle, with $k$=($m$-1)*$i$.
 Thus, with two-nucleon forces ADC(2) and ADC(3) include  2p1h and 2h1p states,  ADC(4) and ADC(5) need up to 
 3p2h and 2h3p states, and so on. However, the full ADC(2/3) sets with three-nucleon forces already
 includes 3p2h and 3h2p configurations~\cite{ch11_Raimondi_inprep}.

At first order, ADC(1) requires to only calculate diagram(s) that contribute to $\widetilde{U}=-\widehat{U}+\Sigma^{(\infty)}$,
see Fig.~\ref{fig:EffOps}a), and thus the scheme reduces to Hartree-Fock theory.
At second order and with at most two-body interactions, there is only one diagram contributing to $\widetilde\Sigma(\omega)$
which is already in the proper Lehmann form. Hence, Eqs.~\eqref{eq:Sig_2nd},~\eqref{eq:ADC2_MEC} and~\eqref{eq:ADC2_NED}
fully define the ADC(2) approximation. In this case, $\Sigma^{(\infty)}$ also requires a second order non-skeleton term.

Higher order cases are more complicated. For a two-body Hamiltonian, the only skeleton diagrams at third order
are the ladder and ring diagrams shown in Figs.~\ref{fig:3rdOrd}a) and ~\ref{fig:3rdOrd}b).
  As long as one works with an Hartree-Fock reference state or a fully self-consistent (dressed)
propagators, no other diagram is needed because the additional  non-skeleton terms either vanish or
must not be included (see {\bf Exercise 11.5}).
In these cases, one obtains the following working expressions the for the ADC(3) approximation:
\begin{subequations}
\label{eq:ADC3_MEC}
\begin{align}
M_{r,\alpha} ={}&  {\pazocal X}^{n_1}_{\mu} {\pazocal X}^{n_2}_{\nu} {\pazocal Y}^{k_3}_{\lambda} \,
V_{\mu \nu, \alpha \lambda} ~+~
  \frac{   \pazocal{X}^{n_1}_{\gamma} \pazocal{X}^{n_2}_{\delta} \;  V_{\gamma \delta, \sigma \zeta} \;
 \pazocal{Y}^{k_4}_{\sigma} \pazocal{Y}^{k_5}_{\zeta} \, (\pazocal{Y}^{k_4}_\mu \pazocal{Y}^{k_5}_\nu)^*  \pazocal{Y}^{k_3}_\lambda  }
                  {2 \, [\varepsilon^-_{k_4} + \varepsilon^-_{k_5} - \varepsilon^+_{n_1}  - \varepsilon^+_{n_2}]} \, V_{\mu \nu, \alpha \lambda}
 \label{eq:ADC3_M} \\
 &{}+ \frac{
{\pazocal Y}^{k}_{\sigma} {\pazocal X}^{n_2}_{\rho}
v_{\rho \delta, \sigma \gamma}
{\pazocal Y}^{k_5}_{\gamma} {\pazocal X}^{n_6}_{\delta}
}{(\varepsilon^-_{k} - \varepsilon^+_{n_2} + \varepsilon^-_{k_5} - \varepsilon^+_{n_6})}
 {\pazocal X}^{n_1}_{\mu} ({\pazocal Y}^{k_5}_{\nu} {\pazocal X}^{n_6}_{\lambda})^* \,
v_{\mu \nu, \alpha \lambda}
 ~-~
\frac{
{\pazocal Y}^{k}_{\sigma} {\pazocal X}^{n_1}_{\rho}
v_{\rho \delta, \sigma \gamma}
{\pazocal Y}^{k_5}_{\gamma} {\pazocal X}^{n_6}_{\delta}
}{(\varepsilon^-_{k} - \varepsilon^+_{n_1} + \varepsilon^-_{k_5} - \varepsilon^+_{n_6})}
 {\pazocal X}^{n_2}_{\mu} ({\pazocal Y}^{k_5}_{\nu} {\pazocal X}^{n_6}_{\lambda})^* \,
v_{\mu \nu, \alpha \lambda}
 \nonumber \\
 \nonumber \\
  E^>_{r,r'} ={}& diag \left\{ \,\varepsilon^+_{n_1} + \varepsilon^+_{n_2} - \varepsilon^-_{k_3}  \, \right\}
  \label{eq:ADC3_Efw}  \\
 \nonumber \\
C_{r,r'} ={}&  \pazocal{X}^{n_1}_{\alpha} \pazocal{X}^{n_2}_{\beta}  \, V_{\alpha \beta, \gamma \delta} \, ( \pazocal{X}^{n_1'}_{\gamma} \pazocal{X}^{n_2'}_{\delta})^* \; \delta_{k_3, k_3'}
 \nonumber \\
 &{} +  {\pazocal X}^{n_1}_\alpha {\pazocal Y}^{k_3}_\beta \, V_{\alpha \delta, \beta \gamma} \,
 ({\pazocal X}^{n_1'}_\gamma {\pazocal Y}^{k_3'}_\delta )^*  \; \delta_{n_2, n_2'}
  -  {\pazocal X}^{n_2}_\alpha {\pazocal Y}^{k_3}_\beta \, V_{\alpha \delta, \beta \gamma} \,
 ({\pazocal X}^{n_1'}_\gamma {\pazocal Y}^{k_3'}_\delta )^*  \; \delta_{n_1, n_2'}
  \label{eq:ADC3_C} \\
&{}  -  {\pazocal X}^{n_1}_\alpha {\pazocal Y}^{k_3}_\beta \, V_{\alpha \delta, \beta \gamma} \,
 ({\pazocal X}^{n_2'}_\gamma {\pazocal Y}^{k_3'}_\delta )^*  \; \delta_{n_2, n_1'}
  +  {\pazocal X}^{n_2}_\alpha {\pazocal Y}^{k_3}_\beta \, V_{\alpha \delta, \beta \gamma} \,
 ({\pazocal X}^{n_2'}_\gamma {\pazocal Y}^{k_3'}_\delta )^*  \; \delta_{n_1, n_1'}
\nonumber
\end{align}
\end{subequations}
and
\begin{subequations}
\label{eq:ADC3_NED}
\begin{align}
 N_{\alpha,s} ={}& V_{\alpha \lambda, \mu \nu}  {\pazocal Y}^{k_1}_{\mu} {\pazocal Y}^{k_2}_{\nu} {\pazocal X}^{n_3}_{\lambda} \,
~+~ \frac{
({\pazocal Y}^{k_1}_{\sigma} {\pazocal Y}^{k_2}_{\rho})^*
v_{\sigma \rho,\gamma \delta}
 ({\pazocal X}^{n_7}_{\gamma} {\pazocal X}^{n_8}_{\delta})^*
}{2 \, [\varepsilon^-_{k_1} + \varepsilon^-_{k_2} - \varepsilon^+_{n_7} - \varepsilon^+_{n_8}]}
{\pazocal X}^{n_7}_{\mu} {\pazocal X}^{n_8}_{\nu} {\pazocal X}^{n}_{\lambda} \,
v_{\mu \nu, \alpha \lambda}
  \label{eq:ADC3_N} \\
&{} +
V_{\alpha \lambda, \mu \nu}
({\pazocal Y}^{k_1}_{\mu})^* {\pazocal X}^{n_5}_{\nu} {\pazocal Y}^{k_6}_{\lambda} \,
\frac{
{\pazocal X}^{n_5}_{\gamma} {\pazocal Y}^{k_6}_{\delta} V_{\rho \gamma, \sigma \delta }
{\pazocal Y}^{k_2}_{\sigma} {\pazocal X}^{n_3}_{\rho}
}{[\varepsilon^-_{k_2} - \varepsilon^+_{n_3} + \varepsilon^-_{k_6} - \varepsilon^+_{n_5}]}
-
({\pazocal Y}^{k_2}_{\mu})^* {\pazocal X}^{n_5}_{\nu} {\pazocal Y}^{k_6}_{\lambda} \,
V_{ \alpha \lambda, \mu \nu}
\frac{
{\pazocal X}^{n_5}_{\gamma} {\pazocal Y}^{k_6}_{\delta} V_{\rho \gamma, \sigma \delta  }
{\pazocal Y}^{k_1}_{\sigma} {\pazocal X}^{n_3}_{\rho}
}{[\varepsilon^-_{k_1} - \varepsilon^+_{n_3} + \varepsilon^-_{k_6} - \varepsilon^+_{n_5}]}
 \nonumber \\
 \nonumber \\
 E^<_{s,s'} ={}& diag \left\{ \, \varepsilon^-_{k_1} + \varepsilon^-_{k_2} - \varepsilon^+_{n_3} \, \right\}
   \label{eq:ADC3_Ebk}  \\
 \nonumber \\
 D_{s,s'}={}&  - (\pazocal{Y}^{k_1}_{\alpha} \pazocal{Y}^{k_2}_{\beta})^*  \, V_{\alpha \beta, \gamma \delta} \, \pazocal{Y}^{k_1'}_{\gamma} \pazocal{Y}^{k_2'}_{\delta} \; \delta_{n_3, n_3'}
\nonumber \\
&{} - ({\pazocal Y}^{k_1}_\alpha    {\pazocal X}^{n_3}_\beta )^* V_{\alpha \delta, \beta \gamma} \, {\pazocal Y}^{k_1'}_\gamma    {\pazocal X}^{n_3'}_\delta   \; \delta_{k_2, k_2'}
 + ({\pazocal Y}^{k_2}_\alpha    {\pazocal X}^{n_3}_\beta )^* V_{\alpha \delta, \beta \gamma} \, {\pazocal Y}^{k_1'}_\gamma    {\pazocal X}^{n_3'}_\delta  \; \delta_{k_1, k_2'}
\label{eq:ADC3_D} \\
&{} + ({\pazocal Y}^{k_1}_\alpha    {\pazocal X}^{n_3}_\beta )^* V_{\alpha \delta, \beta \gamma} \, {\pazocal Y}^{k_2'}_\gamma    {\pazocal X}^{n_3'}_\delta   \; \delta_{k_2, k_1'}
 - ({\pazocal Y}^{k_2}_\alpha    {\pazocal X}^{n_3}_\beta )^* V_{\alpha \delta, \beta \gamma} \, {\pazocal Y}^{k_2'}_\gamma    {\pazocal X}^{n_3'}_\delta  \; \delta_{k_1, k_1'}  \; ,
  \nonumber
\end{align}
\end{subequations}
where  only  ordered configurations $r$=$\{n_1 < n_2, k_3 \}$ and  $s$=$\{k_1 < k_2, n_3 \}$ need to be considered, in accordance
with the Pauli principle.  Note that  these equations apply to the case of  two-body interactions  but they remain unchanged for  an effective operator $\widetilde{V}$ that is derived from three-body forces.  However the full inclusion of $\widehat{W}$ would require the inclusion of the diagram of Fig.~\ref{fig:2ndOrd}b) at the ADC(2) level and several other interaction-irreducible diagrams for ADC(3).  The non-skeleton contributions to $\widetilde\Sigma(\omega)$ that arise at third order when the reference propagator in not dressed are shown in Fig.~\ref{fig:SEins_3ndOrd}. The case of three-nucleon forces is discussed in full detail in Ref.~\cite{ch11_Raimondi_inprep}.


\begin{figure}[t]
\begin{center}
\includegraphics[height=0.22\textwidth]{Chapter11-figures/fig11_5_a.pdf}   \hspace{0.25\textwidth}
\includegraphics[height=0.22\textwidth]{Chapter11-figures/fig11_5_b.pdf}   \hspace{0.20\textwidth}
\vskip 0.8 cm
%\hspace{0.05\textwidth}
\includegraphics[height=0.21\textwidth]{Chapter11-figures/fig11_5_c.pdf}   \hspace{0.1\textwidth}
\includegraphics[height=0.21\textwidth]{Chapter11-figures/fig11_5_d.pdf}
\caption{Self-energy insertion diagrams that appear, up to third order, in the perturbative expansion for $\widetilde\Sigma(\omega)$ with two- and three-nucleon interactions. These non-skeleton diagrams need to be considered  when the reference propagators are not self-consistent.  Diagrams a) and b) involve only one- and two-body interactions and results from self-energy insertion into the diagram of Fig.~\ref{fig:2ndOrd}a).  With the inclusion of three-nucleon interactions, the diagrams c) and d) arise from the one of Fig.~\ref{fig:2ndOrd}b).  When an Hartree-Fock reference state is used all these contributions cancel out (see {\bf Exercise 11.5}). }
\label{fig:SEins_3ndOrd}
\end{center}
\end{figure}

To remain consistent with the ADC(n) formulation,  the static self-energy $\Sigma^{(\infty)}$ must also be computed at least to the same order $n$. However, this involves a large number of non-skeleton diagrams when self-consistency is not implemented. In practice, it is relatively inexpensive to compute it directly from dressed propagators, as given by Eqs.~\eqref{eq:SigSplit} and~\eqref{eq:U_eff} and therefore if can be iterated to self-consistency.
This prescription, in which $\widetilde\Sigma(\omega)$ is calculated from an unperturbed reference state $g^{0)}(\omega)$ but $\Sigma^{(\infty)}$
is obtained self-consistently, is often used in nuclear physics applications and we refer to it as the {\em sc0} approximation~\cite{ch11_Soma2014Lanc}.
When dealing with the Coulomb force in molecular systems, the dynamic self-energy can be simply calculated in terms of a Hartree-Fock reference state.  In nuclear physics, an Hartree-Fock reference state is adequate only if the chosen Hamiltonian is particularly soft.  Otherwise, it is necessary to optimize the reference state by choosing a $\widehat{H}_0$ and $g^{0)}(\omega)$  that better represent the correlated single particle energies in the dressed propagator. In all cases, at least the {\em sc0} approach to $\Sigma^{(\infty)}$ is always required when computing finite nuclei and infinite nucleonic matter.

The standard ADC($n$) prescription is to identify the {\em minimal} matrices  $M$, $N$, $C$ and $D$   that make self-energy  complete in
perturbation theory up to order $n$. However, other intermediate approximations are also possible and have been exploited in the past.
%
The so-called \hbox{\em 2p1h-TDA} method is an extension of the second order scheme of Eqs.~\eqref{eq:ADC2_MEC} and~\eqref{eq:ADC2_NED}  where the matrices $C$ and $D$ are  calculated at first order instead, as given by  Eqs.~\eqref{eq:ADC3_C} and~\eqref{eq:ADC3_D}. As a rule of thumb, the ADC(2)  approximation yields roughly 90\% of the total correlation energy in most applications, while the ADC(3) can account for about 99\% of it---hence, with a~1\% error in binding energies. The 2p1h-TDA  contains the ADC(2) in full but it further resums the full set of two-particle~(pp), two-holes~(hh) and particle-hole~(ph) diagrams. This can result in a sensible improvement in the accuracy of binding energies but without the price of computing corrections to the $M$ and $N$ coupling matrices. Nevertheless, the 2p1h-TDA misses the second order terms in Eqs.~\eqref{eq:MNexp} that are known to contribute strongly to quasiparticle energies. As a consequence the one nucleon addition and separation energies (or, equivalently, ionization potentials and electron affinities in molecules) would be predicted poorly in 2p1h-TDA and in general they require full ADC(3) calculations~\cite{ch11_VonNiessen1984ConPhysRep}.
%
%
In nuclear physics applications, the description of collective excitations often requires that  particle-hole configurations are diagonalized at least in the random phase approximation (RPA) scheme. While this is similar to the TDA all-order summations included in 2p1h-TDA and in ADC(3), extra ground state correlations effect from the RPA series are deemed important to reproduce collective modes typical of nuclear systems~\cite{ch11_RingSchuck}.
%
To account for these effects on needs to separate the partial summations in the pp, hh and ph channels, substitute them with equivalent RPA series and  recouple these through a Faddeev-like expansion, in order to eventually reconstruct the self-energy~\cite{ch11_Danielewicz1994OMP,ch11_Barbieri2001frpa}.
The Faddeev-RPA (FRPA)  method contains the ADC(3) in full but it also generates additional ground state correlation terms that are induced by the RPA summation and are at fourth and higher order in the perturbative expansion of the self-energy. A working implementation of the FRPA approach has been formulated in Refs.~\cite{ch11_Barbieri2001frpa,ch11_Barbieri2007Atoms,ch11_Degroote2011frpa}.


\vskip .3 cm
\noindent
{\bf Exercise 11.4.}  Calculate the ladder and ring diagrams in Fig.~\ref{fig:3rdOrd}  and prove Eqs.~\eqref{eq:ADC3_MEC} and~\eqref{eq:ADC3_NED} in full.
[Hint: for the ring diagrams it is simpler to first perform integrations for the free polarization propagator,
$\Pi^f_{\alpha\beta,\gamma\delta}(\omega)=\int \frac{d \, \omega_1}{2\pi i} g_{\alpha\gamma}(\omega+\omega_1)g_{\delta\beta}(\omega_1)$,
which describes  non interacting particle-hole states.]

\vskip .3 cm
\noindent
{\bf Exercise 11.5.}  In case of a reference propagator that is not fully self-consistent, it is necessary to also include non-skeleton diagrams. For
$\widetilde\Sigma(\omega)$ these first appear at third order with the two diagrams shown in Fig.~\ref{fig:SEins_3ndOrd}a) and~\ref{fig:SEins_3ndOrd}b).
Calculate the expressions for these diagrams and:
\begin{itemize}
\item Deduct the corresponding corrections to Eqs.~\eqref{eq:ADC3_MEC} and~\eqref{eq:ADC3_NED}. These will be the complete ADC(3) working equations.
\item Show that they cancel out exactly if the reference propagator is of Hartree-Fock type. Hence these corrections do not need to
be taken into account even tough the Hartree-Fock reference state is {\em not} a dressed---and fully self-consistent---input in this case.
\end{itemize}
[Hint: In Hartee-Fock theory, the static self-energy $\Sigma^{(\infty)}(\omega)$ reduces to the Hartree-Fock potential. The reference state in this case is given by $\widehat{H}_0=\widehat{T}+\widehat{U}^{HF}\equiv\widehat{H}^{HF}$, which is also the Hartree-Fock Hamiltonian. Additionally, in the notation of Eqs.~\eqref{eq:Z_ampl} below,
the (orthogonal) single particle wave functions are the solutions of $\{T + \Sigma^{HF}\} \pazocal{Z}^i = \varepsilon^i \pazocal{Z}^i$.]





\subsection{Solving the Dyson equation}

 Once we have  a suitable approximation to the self-energy, it is necessary to solve the Dyson equations~\eqref{eq:Dyson}
 to obtain the single particle propagator, the  associated  observables and the  spectral function. The latter will also yield spectroscopic amplitudes and their spectroscopic factor for the addition and removal of a nucleon form the correlated state $|\Psi^A_0\rangle$.  In doing this, Eqs.~\eqref{eq:Dyson} takes the form of a one-body Schr\"odinger equation for the scattering of a particle or a hole inside the medium. Given that all the Cauchy integrals associated with Feynman diagrams have been carried out, we can safely take the limit $\pm i \eta \rightarrow 0$ in all denominators for simplicity. The same equation applies to states both above and below the Fermi surface.
 Thus, it is convenient to take a general index $i$ and using $\varepsilon_i$ and $\pazocal Z^i$ to label energies and spectroscopic amplitudes for all quasiparticle and quasihole states. Specifically,
 \begin{equation}
\varepsilon_i \longrightarrow \left\{
\begin{array}{lcl}
\varepsilon_n^+ & \quad & \hbox{for $i$=$n$, particle,}  \\ ~ \\
\varepsilon_k^- &  & \hbox{for $i$=$k$ hole,}
\end{array} \right.
\qquad \hbox{and} \qquad
\pazocal Z^i_\alpha  \longrightarrow \left\{
\begin{array}{lcl}
\pazocal X^n_\alpha & \quad & \hbox{for $i$=$n$, particle,} \\ ~ \\
\pazocal Y^k_\alpha &  & \hbox{for $i$=$k$, hole.}
\end{array} \right.
\label{eq:Z_ampl}
\end{equation}

In order to extract the solution for the pole $i$ in the Lehmann representation, we extract the corresponding residue on both the left and right hand side of Eq.~\eqref{eq:Dyson_a}:
 \begin{equation}
   \lim _{\omega \rightarrow \varepsilon_i}
   \left\{
     g_{\alpha\beta}(\omega) = g^{(0)}_{\alpha\beta}(\omega) + g^{(0)}_{\alpha\gamma}(\omega)  \Sigma^\star_{\gamma\delta}(\omega) g_{\delta\beta}(\omega)
   \right\} \; ,
 \end{equation}
which gives
 \begin{equation}
    \pazocal Z^i_\alpha (\pazocal Z^i_\beta)^*  =  \left. g^{(0)}_{\alpha\gamma}(\omega)  \Sigma^\star_{\gamma\delta}(\omega) \pazocal Z^i_\delta (\pazocal Z^i_\beta)^*
    \right|_{\omega = \varepsilon_i}
 \; .
 \end{equation}
By dividing out $\pazocal Z^i_\beta$ and using the fact that $[g^{(0)}(\omega)]^{-1} = \omega - \widehat{H}_0$ % = \omega - (\widehat{T} + \widehat{U})$
we finally obtain the  eigenvalue equation
 \begin{eqnarray}
   \varepsilon_i   \pazocal Z^i_\alpha &=&   \left. \left\{ \widehat{T} + \widehat{U} +   \Sigma^\star(\omega)   \right\}_{\alpha \, \delta}
  \pazocal Z^i_\delta  \right|_{\omega = \varepsilon_i}
  \nonumber \\
  &=&  \left. \left\{  \widehat{T}  +  \Sigma^{(\infty)} +M^\dagger\frac1{\omega - E^> -C + i \eta}M +N\frac1{\omega - E^< -D - i \eta}N^\dagger     \right\}_{\alpha \, \delta}
  \pazocal Z^i_\delta  \right|_{\omega = \varepsilon_i}  \; ,
\label{eq:DysSchrod}
 \end{eqnarray}
where the potential $\widehat{U}$ defining the unperturbed state completely cancels out. From here we see that the true irreducible self-energy $\Sigma^{(\infty)}+\widetilde\Sigma(\omega)$
acts as the non-local and energy dependent potential that accounts for the motion of both particles and holes inside the system and for their
coupling intermediate excitations.
At positive energies ($\omega > 0$) this equation describes the elastic scattering of a nucleon off the $|\Psi^A_0\rangle$ ground
states and the self-energy can be identified with a fully microscopic optical potential~\cite{ch11_Capuzzi1996,ch11_Cederbaum2001,ch11_Barbieri2005}.
In this case the spectroscopic amplitudes $\pazocal Z^i$ correspond to scattering wave functions with the usual asymptotic
normalization.
Instead, at $\omega < 0$, Eq.~\eqref{eq:DysSchrod}  describes the transition to states of $|\Psi^{A\pm1}_i\rangle$ with bound amplitudes.
The norm of each $\pazocal Z^i$ gives the corresponding spectroscopic factor and it is obtained as
\begin{equation}
 S^i = \sum_\alpha |\pazocal Z^i_\alpha|^2 =  \frac 1 {1 - (\overline{\pazocal Z}^i_\beta)^*
   \left. \frac{d \, \Sigma^\star_{\beta \gamma}(\omega)}{d \, \omega} \right|_{\omega = \varepsilon_i}
    \overline{\pazocal Z}^i_\gamma}   \; ,
\label{eq:SFnorm1}
\end{equation}
where $\overline{\pazocal Z}^i \equiv {\pazocal Z}^i / \sqrt{S^i}$ is the spectroscopic amplitude normalized to 1.

Equations~\eqref{eq:DysSchrod} and~\eqref{eq:SFnorm1} are the central equations of the Green's function formalism and show how the
single-particle propagator is the solution of an effective one-body Schr\"odinger equation for a nucleon or a
hole propagating inside the correlated system. The energy dependence of  $\Sigma^\star(\omega)$  and its non-locality are a consequence of the underlying  many-body dynamics. Eq.~\eqref{eq:SFnorm1} also shows that the reduction of spectral strength commonly observed in correlated systems arises from the dispersion
properties of the self-energy.

 In spite of its beauty, Eq.~\eqref{eq:DysSchrod} is also the worst starting point to solve the Dyson equation in a discretized finite basis. Unless one is interested in just a few solutions near the Fermi surface or the model space is extremely small, this approach will require high computational times due to the large amounts of diagonalizations required to extract the correct eigenvalues. The reason is that root-finding algorithms are needed to match the eigenvalues $\varepsilon_i$ with the argument of  $\Sigma^\star(\varepsilon_i)$ but simple searching algorithms may miss a large amount of solutions. The consequences of missing a large portion of spectral strength are that wrong results would be obtained for the ground state observables computed as in Sec.~\ref{sec:scgf_obs}. This can also deteriorate the self-consistency already at the level of the static self-energy, $\Sigma^{(\infty)}=\widetilde{U}$.  If Eq.~\eqref{eq:DysSchrod} must be used, it is possible to gather all the necessary solutions by starting from extremely fine energy meshes to be sure that all eigenvalues are bracketed first. However, this easily becomes suicidal in terms of the increase of computing time.
 %
We discuss here a different approach that is not affected by these problems and that will also give some further insight into the physics content of the Dyson equation.

First, for each solution of the Dyson equation we define two new vectors $\pazocal{W}^i$ and  $\pazocal{V}^i$ which live in the ISCs space as follows:
\begin{align}
 %\sum_{r'}   \;
 [\omega - E -C ]_{r,r'}  \, \pazocal{W}^i_{r'}  ~\equiv{}&~
  % \sum_\delta \;
M_{r,\delta}   Z^i_\delta  \; ,
 %\label{eq:defW} \\
 \nonumber \\
 %\sum_{s'}  \;
  [\omega - E -D ]_{s,s'} \, \pazocal{V}^i_{s'} ~\equiv{}&~
 % \sum_\delta \;
   N^\dagger_{s,\delta}    Z^i_\delta \; ,
 \label{eq:defWV}
\end{align}
where we have let $i\eta\rightarrow 0$ as this is no longer needed in a finite and discretized basis. With these definitions, Eq.~\eqref{eq:DysSchrod}  is easily rearranged into a single eigenvalue problem of larger dimensions but where the corresponding matrix is energy independent:
\begin{equation}
\left( \begin{array}{ccccc}
 \widehat{T} + \Sigma^{(\infty)}  &~&   M^\dagger   &~&  N~  \\
&&\\
    M   &&  C  && \\
    &&\\
    N^\dagger    &&      &&  D
\end{array} \right)
\left( \begin{array}{c}
\pazocal{Z}^i \\ ~ \\ \pazocal{W}^i \\~ \\ \pazocal{V}^i
\end{array} \right)
=\left( \begin{array}{c}
  \pazocal{Z}^i \\ ~\\ \pazocal{W}^i \\~\\ \pazocal{V}^i
\end{array} \right)
  \varepsilon_i
\label{eq:DysMtx}
\end{equation}
and the normalization condition~\eqref{eq:SFnorm1} becomes
\begin{equation}
\sum_\alpha  |\pazocal Z^i_\alpha|^2 + \sum_r  |\pazocal W^i_r|^2 + \sum_s  |\pazocal V^i_s|^2 = 1   \; .
\label{eq:SFnorm2}
\end{equation}

The advantage of Eq.~\eqref{eq:DysMtx} is that it linearizes the Dyson equation and yields all solutions in one single diagonalization. Although the dimension of the Dyson equation is much larger than a one-body Schr\"odinger problem and that it requires substantial amount of memory storage, it typically provides the full spectral strength 100 times faster than using Eq.~\eqref{eq:DysSchrod} directly. Furthermore, it is possible to reduce the dimensionality of the eigenvalue problem by projecting matrices $C$ and $D$ (separately!) onto smaller Lanczos/Krylov subspaces~\cite{ch11_Schirmer1989BlkLanc,ch11_Soma2014Lanc}. In this way one reduces the number of poles of $g(\omega)$ far away from the Fermi surface---where only their averages is physically meaningful---but conserves the overall strength needed to compute ground state observables.

Eq.~\eqref{eq:DysMtx} also puts in evidence how the Dyson equation is very closely related to a configuration interaction (CI) approach. For solutions ($\varepsilon^+_n$,$\pazocal{X}^n$) in the single particle spectrum,  the eigenstates of $|\Psi^{A+1}_n\rangle$ are expanded in terms of 1p configurations (from the $\widehat{T} + \Sigma^{(\infty)}$ sector) and 2p1h or larger configurations, which is evident from the matrix $C$, in Eq.~\eqref{eq:ADC3_C}. However, additional 2h1p configurations are included through matrix $D$. This is in spirit very similar to how ground state correlations are included in the random phase approximation approach~\cite{ch11_RingSchuck}. Furthermore,  the  matrices that couples these subspaces are the same as in CI only at first order ($M^{(1)}$ and $N^{(1)}$). The eigenstates of Eq.~\eqref{eq:DysMtx} will approach the exact solution as the approximation of the self-energy is systematically improved. Similarly, the propagation of hole states that correspond to the eigenstates of $|\Psi^{A-1}_k\rangle$ are obtained in a CI fashion.
Eq.~\eqref{eq:SFnorm2} is then the natural normalization condition for the CI expansion and shows that the spectroscopic amplitudes are the projection of more complex many-body wave functions onto a single-particle space.

\vskip .3 cm
\noindent
{\bf Exercise 11.6.}  Perform a Taylor expansion of the propagator $g(\omega)$ at zero-th order around a given pole $\varepsilon_i^\pm$. Then, use this and the
conjugate Dyson equation~\eqref{eq:Dyson_b} to obtain the normalization condition for spectroscopic factors given in  Eq.~\eqref{eq:SFnorm1}.


\vskip .3 cm
\noindent
{\bf Exercise 11.7.}  Based on the definitions of vectors $\pazocal{W}^i$ and $\pazocal{V}^i$, Eqs.~\eqref{eq:defWV}, show that~\eqref{eq:SFnorm1} and~\eqref{eq:SFnorm2} are equivalent.



\subsection{A simple pairing model}

\begin{figure}[ht]
\begin{center}
\includegraphics[width=0.8\textwidth]{Chapter11-figures/Pairing_LNP_scgf.pdf}
\caption{Correlation energy for the pairing Hamiltonian of Eq.~\eqref{eq:H_pair} obtained for different  ADC($n$) approximations
to the self-energy in a $sc0$ scheme. The purple line shows the exact results from full CI theory.  }
\label{fig:pairng_adc}
\end{center}
\end{figure}

\begin{figure}[ht]
\begin{center}
\includegraphics[width=0.8\textwidth]{Chapter11-figures/Pairing_LNP_ALL.pdf}
\caption{Correlation energy for the pairing Hamiltonian of Eq.~\eqref{eq:H_pair} obtained for different
many-body methods discussed in this book. The purple line is the exact results from full CI theory. Various CC approximations (as discussed in
chapter 8), the  second order perturbation theory (MBPT2)  and ADC(3) results are shown.}
\label{fig:pairng_all}
\end{center}
\end{figure}

As a first application of the ADC formalism, we consider here the paring Hamiltonian discussed in chapter~8. This is a system of  four spin-1/2 fermions in a 4-level model space that interact through a pairing force:
  \begin{equation}
   \widehat{H} = \widehat{H}_0 +  \widehat{V} = \delta \sum_{p=1}^4  \sum_{\sigma=+, -} (p-1) a^{\dagger}_{p \sigma} a_{p \sigma}
 ~-~ \frac{g}{2} \sum_{p, q=1}^4 a^{\dagger}_{p+}a^{\dagger}_{p-}  a_{q-}a_{q+}
\label{eq:H_pair}
\end{equation}


In spite of its simplicity,  the model of Eq.~\eqref{eq:H_pair} is a particularly difficult test for many-body approximations based on ISRs because the Hamiltonian does not contain leading order contributions that connect single particle excitation with 2p1h/2h1p configurations in the ground state. This means that the dominant many-body effect accounted in ADC(2,3) or CCSD calculations are suppressed and  higher configurations of 2p2h dominate.




We perform calculations at different levels of approximation in the ADC($n$) scheme. First, we implement the ADC(2) equations~\eqref{eq:ADC2_MEC} and~\eqref{eq:ADC2_NED} and a hybrid scheme in which the matrices $M$ and $N$ remain the same as in ADC(2) but with the use of the third order  expressions
for $C$ and $D$. The latter approximation in commonly referred to as `extended-ADC(2)' and includes the all order summations of ladder and ring diagrams. However, it misses third order contributions to the coupling matrices that are important to obtain correct separation energies  for addition and removal of a particle.  The results for the correlation energies are compared to the exact full-configuration interaction (FCI) results in Fig.~\ref{fig:pairng_adc}, where it is seen that the extended-ADC(2)
gives a small improvement for a repulsive  interaction ($g<0$), but it deteriorates for attractive pairing.
The complete ADC(3) approximation corrects the issues of the extended-ADC(2) and becomes very close the the exact result for a repulsive pairing. However, for $g>0$, it does not improve upon the simple ADC(2).  This is mostly because of the lack of direct terms in Hamiltonian~\eqref{eq:H_pair} that connects single particle states to 2p2h and 4p4h configurations...



A comparison with the CC approach and MBPT is presented in Fig.~\ref{fig:pairng_all}.
The ADC(3) and CCD methods perform similarly at $g<0$, where they are both close to the exact solution. They deviate by similar amounts for attractive pairing but in opposite directions.

The FORTRAN code that generated these results is available online at  {\sloppy  \url{https://github.com/ManyBodyPhysics/LectureNotesPhysics/blob/master/doc/src/Chapter11-programs/Pair_Model}}. We will not discuss this code here, but we will give a detailed discussion of how to structure a similar ADC($n$) code for infinite matter computations in the next section.


\section{A computational project in infinite matter}
\label{sec:scgf_comp}


In this section we discuss how to approach  ADC(3) calculations for infinite matter. We will do this
using the C++ programming language and will refer to the numerical code provided within this chapter at the {\sloppy
URL:  \url{https://github.com/ManyBodyPhysics/LectureNotesPhysics/blob/master/doc/src/Chapter11-programs/Inf_Matter}.}

\iffalse
The first fundamental step  to set up a SCGF computation is the choice of the model space. For
infinite matter, translational invariance imposes that the Dyson equation is diagonal in momentum  and
therefore it becomes much easier to solve the problem in momentum space. However, there remain two possible
choices for how to encode single particle degrees of freedom.
The first one is to subdivide the infinite space in  boxes of finite length  and to impose periodic boundary conditions
(see also chapter 8). In this way, the number of fermions included in each box is finite and determined by the particle
density of the system. The resulting model space is naturally expressed by a set of discretized single particle
states and equations in the form of Eqs.~\eqref{eq:ADC3_MEC}, \eqref{eq:ADC3_NED} and \eqref{eq:DysMtx}, and it
can be solve directly as ti would be done for a finte system in a box.  Numerical results then need to be converged
with respect to the truncation of the k-space inside each box.   We will follow this approach for the present
computational project.
The other approach is to retain the full momentum space and write the SCGF equations already in the full
thermodynamic limit. This approach is more suitable to solve the Dyson equation at finite temperatures and
in a full SCGF fashion and will be discussed further in Sections~\ref{sec:scgf_finiteT}. % and~\ref{sec:scgf_comp_finiteT}.


\vskip 0.5 cm
{\em Construction of the model space.}


\lstset{language=c++}
\begin{lstlisting}
    i_count = 0;
    for (int ix=-imax; ix<=imax; ++ix)
      for (int iy=-imax; iy<=imax; ++iy)
        for (int iz=-imax; iz<=imax; ++iz) {
          itest = ix*ix + iy*iy + iz*iz;
          if ((nsq_mn <= itest) && (itest <= nsq_mx)) ++i_count;
        }
\end{lstlisting}
Once we know how many single particle $\vec k$ states we have, we can allocate
arrays in memory to store the relevant quantum numbers of each of them:
\begin{lstlisting}
  int i1 = this->Count_sp_basis(0, nsq_mx, imax);

  SpNAlloc = (ch_max - ch_min + 1) * 2 * i1 + 10; // +10 for safety

  cout << "\n allocating space for "<< SpNAlloc << " sp states... \n";

  nx    = new int[SpNAlloc];
  ny    = new int[SpNAlloc];
  nz    = new int[SpNAlloc];
  nsq   = new int[SpNAlloc];
  spin  = new int[SpNAlloc];
  chrg  = new int[SpNAlloc];
  k     = new double[SpNAlloc];
  e_kin = new double[SpNAlloc];
  e_HF  = new double[SpNAlloc];
  e_sp  = new double[SpNAlloc];
  group = new int[SpNAlloc];

  int ich, is;

  double xk = 0.0, xek = 0.0;

  i1 = 0;
  for (int isq=0; isq<=nsq_mx; ++isq) {
    for (int ix=-imax; ix<=imax; ++ix) {
      for (int iy=-imax; iy<=imax; ++iy) {
        for (int iz=-imax; iz<=imax; ++iz) {
          if ((ix*ix + iy*iy + iz*iz) != isq) continue;

          xek = double(isq);
          xk  = sqrt(xek) * 2.0 * PI / Lbox;
          xek = xk * xk * MeVfm * MeVfm / 2.0 / NUCLEONmass;  //nuc_mass_ave;
          cout << i1 << "  " << ix << "  " << iy << "  " << iz << "  ";
          cout << isq << "  " << xk << "  " << xek << endl;

          for (ich=ch_min; ich<=ch_max; ++ich)
            for (is=-1; is<2; is+=2) {
              nx[i1] = ix;
              ny[i1] = iy;
              nz[i1] = iz;
             nsq[i1] = isq;
            spin[i1] = is;
            chrg[i1] = ich;
               k[i1] = xk;
           e_kin[i1] = xek;
            e_HF[i1] = 0.0;
            e_sp[i1] = 0.0;
           group[i1] = -100;
            ++i1;
            }

        }
      }
    }
  }
  SpNmax = i1;
\end{lstlisting}

\vskip 0.5 cm
{\em Construction of ISCs.}

\vskip 0.5 cm
{\em Two body interaction: Minnesota potential}

\vskip 0.5 cm
{\em Hartree-fock theory}

\vskip 0.5 cm
{\em Constructing the Dyson matrix}

\vskip 0.5 cm
{\em Numerical solutions at ASDC(20 and ADC(3) level}



\begin{figure}[t]
\begin{center}
\includegraphics[width=0.45\textwidth]{Chapter11-figures/cimcccd.pdf}
\includegraphics[width=0.45\textwidth]{Chapter11-figures/cimcccd.pdf}
\caption{Equation of state for pure neutron matter (left) and symmetric nuclear matter (right) as predicted by for the Minnesota two-nucleon interaction. Different curves show results for different ADC($n$) approximations. The ADC(2) (dashed lines), 2p1h-TDA (long dashed line) and full ADC(3) (don-dashed lines) are calculated with an Hartree-Fock reference state and single particle energies. The full line gives results for ADC(3) with a self-consistent unperturbed single particle energies.}
\label{fig:minn_adc_eos}
\end{center}
\end{figure}


\begin{figure}[ht]
\begin{center}
\includegraphics[width=0.45\textwidth]{Chapter11-figures/cimcccd.pdf}
\includegraphics[width=0.45\textwidth]{Chapter11-figures/cimcccd.pdf}
\caption{Spectral function of pure neutron matter (left) and symmetric nuclear matter (right) at nominal saturation density ($\rho=0.16$~fm$^{-3}$) from ADC(3)---3D plot. }
\label{fig:minn_adc_sfnct}
\end{center}
\end{figure}


%We show here the result form pure neutron matter calculated with the Minnesota interactions.

Fig.~\ref{fig:minn_SE} shows the self-energy for momentum $k=2 K_F/3$  for nominal saturation density...

\begin{figure}[ht]
\begin{center}
\includegraphics[width=0.45\textwidth]{Chapter11-figures/cimcccd.pdf}
\includegraphics[width=0.45\textwidth]{Chapter11-figures/cimcccd.pdf}
\caption{Nucleon self-energy, $\Sigma(k,\omega)$ of pure neutron matter (left) and symmetric nuclear matter (right) at nominal saturation density ($\rho=0.16$~fm$^{-3}$) from ADC(3).  Values for $k=0$,$K_F/2$,$3K_F/2$. }
\label{fig:minn_SE}
\end{center}
\end{figure}


We finally compare results for different methods discussed in this book in Fig.~\ref{fig:minn_all}...

\begin{figure}[ht]
\begin{center}
\includegraphics[width=0.8\textwidth]{Chapter11-figures/cimcccd.pdf}
\caption{Total correlation energy for pure nuclear matter obtained from different methods presented in this book. This would include CCD, IM-SRG and ADC(3) in
a discretized cartesian basis and CIMC calculations.}
\label{fig:minn_all}
\end{center}
\end{figure}

\fi


\section{Self-consistent Green's functions at finite temperature in the thermodynamic limit}
\label{sec:scgf_finiteT}

In the following we want to concentrate on the study of an infinite system at finite temperature; we will then set ourselves in the thermodynamic limit, i.e. number of particles $N$ and volume $V$ tending at infinity with density $\rho=N/V$ kept constant. The may-body SCGF approach at finite temperature is particularly suited for this kind of study because it is thermodynamically consistent, meaning that a quantity calculated from the microscopic point of view yields the same results as the thermodynamical macroscopic quantity~\cite{ch11_Baym1962}. This consistency is strictly related to the fact that a fully dressed propagator, obtained via iterative solution of Dyson's equation Eqs.~(\ref{eq:Dyson}), is used in the calculation of the partition function in the Luttinger-Ward formalism~\cite{ch11_Luttinger1960}, from which one extracts the thermodynamical properties of the system. Furthermore, it can be demonstrated that this method fulfills the Hugenholtz van-Hove theorem~\cite{ch11_Hugenholtz1958}, and this once again relates to the fact that the conservation laws of particle number, momentum and energy are preserved in this kind of approximation~\cite{ch11_Baym1961,ch11_Baym1962}.

We will show in this section how to calculate the self-consistent propagator in the ladder approximation, a specific approximation for the self-energy $\Sigma^\star(\omega)$ where particle-particle and hole-hole intermediate scattering states are resummed to all orders in the so called in-medium T-matrix. We will be working with the effective Hamiltonian of Eq.~\eqref{eq:Heff}, considering the two-body averaged three-body force that enters $\widetilde U$ as given in Eq.~\eqref{eq:ueff_3b_first}, and disregarding all irreducible three-body terms. The Koltun sum rule of Eq.~\eqref{eq:Koltun_hW} will then be used to obtain the total energy of the many-body system. The great advantage of working at finite temperature is that the appearance of pairing when considering hole-hole intermediate states is washed out by the thermal effects~\cite{ch11_Alm1996}. Recently an improved treatment of pairing in the SCGF method when going to zero temperature has been presented by the Authors of Ref.~\cite{ch11_ding2016}. Within the Luttinger-Ward formalism, the entropy can then be calculated via the knowledge of the self-consistent propagator, and from the entropy all other thermodynamical quantities are accessible. We will not treat here the calculation of the entropy, for a detailed description we refer the reader to Chapter 3 of Ref.~\cite{ch11_Rios2007PhD}.

In the next section, we will give a few hints on the theoretical formalism and then sketch in the following section the working equations necessary to perform the numerical implementation. The full self-consistent numerical calculation considering the complete off-shell properties of the system and considering fully microscopic potentials was performed by the T\"ubingen/Barcelona group~\cite{ch11_Frick2003,ch11_Frick2004PhD,ch11_Frick2005,ch11_Rios2006C74,ch11_Rios2008,ch11_Rios2009} and from the Cracow group~\cite{ch11_Soma2006,ch11_Soma2008,ch11_Soma2009,ch11_Soma2009phd}.

\subsection{Finite-temperature Green's function formalism}
In a similar way done in Sec.~\ref{sec:scgf_defs}, we start by defining the one-body Green's function, however this time as a statistical average in the grand-canonical ensemble:
\begin{equation}
\label{eq:thermalG}
ig({\bf x}t, {\bf x'}t')= {\rm Tr}\{\widehat{\rho}{\pazocal T}[\widehat\psi({\bf x}t) \widehat\psi^\dagger({\bf x'}t')]\}\,;
\end{equation}
here ${\pazocal T}$ describes the Wick time-ordered product of the quantum field operators in the Heisenberg picture of creation, $\widehat\psi^\dagger({\bf x'}t')$, and destruction, $\widehat\psi({\bf x}t)$, of a single-particle state. The field operators are related to the operators of creation and destruction, i.e. $a^\dagger_\alpha$ and $a_\alpha$, via $\widehat\psi^\dagger({\bf x'})=\sum_\alpha\psi_\alpha({\bf x})^\dagger a^\dagger_\alpha$ and $\widehat\psi({\bf x})=\sum_\alpha\psi_\alpha({\bf x}) a_\alpha$, where the coefficients are the single-particle wave functions and the sum is over the complete basis set of single-particle quantum numbers. The statistical factor $\widehat \rho$ is defined by:
\begin{equation}
\widehat \rho=\frac{1}{Z}e^{-\beta(\widehat H -\mu\widehat N)}\,,
\end{equation}
where $\beta=1/T$, and $Z$ is the grand-partition function
\begin{equation}
\label{eq:part_fun}
Z={\rm Tr}\,e^{-\beta(\widehat H -\mu\widehat N)}\,,
\end{equation}
with $\widehat H$ the Hamiltonian given in Eq.~(\ref{eq:H}), and $\widehat N$ the particle number operator. The trace in Eq.~(\ref{eq:part_fun}) is to be taken over a full set of energy and particle number eigenstates of the system. The two possible time-ordering products in Eq.~(\ref{eq:thermalG}) are given by:
\begin{equation}
\label{eq:Tproduct}
{\pazocal T}[\widehat\psi({\bf x}t) \widehat\psi^\dagger({\bf x'}t')]=
 \Bigg\{
  \begin{tabular}{c}
  \,\,\,\,$\widehat\psi({\bf x}t) \widehat\psi^\dagger({\bf x'}t'), \quad t>t'$  \\
  $-\widehat\psi^\dagger({\bf x'}t') \widehat\psi({\bf x}t), \quad t'>t$
  \end{tabular}
\end{equation}
The first time-ordered product in Eq.~(\ref{eq:Tproduct}) describes the creation of a particle state at time $t'$ with position ${\bf x'}$, and the destruction of the propagated particle state at time $t$ with position ${\bf x}$. Analogously, the second time-ordered product in Eq~(\ref{eq:Tproduct}) describes the destruction of a particle state, or creation of a hole state, at time $t$ with position ${\bf x}$, and the destruction of the propagated hole state at time $t'$ with position ${\bf x'}$. From Eq.~(\ref{eq:Tproduct}) one can define the correlation functions: 
\begin{eqnarray}
\label{eq:corr_creat}
ig^>({\bf x}t, {\bf x'}t')&=& ~ ~ {\rm Tr}\{\widehat{\rho}[\widehat\psi({\bf x}t) \widehat\psi^\dagger({\bf x'}t')]\} \\
ig^<({\bf x}t, {\bf x'}t')&=& -{\rm Tr}\{\widehat{\rho}[\widehat\psi^\dagger({\bf x'}t')\widehat\psi({\bf x}t)]\}\,.
\label{eq:corr_destr}
\end{eqnarray}
Depending on the specific time ordering, the Green's function defined in Eq.~(\ref{eq:thermalG}) corresponds to one correlation function or the other, i.e. either to Eq.~(\ref{eq:corr_creat}) or to Eq.~(\ref{eq:corr_destr}). It is also useful to define the retarded propagator; this is that part of the one-body Green's functions which is related only to the causal  propagation of events, i.e. forward in time:
\begin{equation}
\label{eq:retar_prop}
g^R({\bf x}t, {\bf x'}t')=\theta(t-t')[g^>({\bf x}t, {\bf x'}t')-g^<({\bf x}t, {\bf x'}t')]\,.
\end{equation}

If we now look at the quantum field operators of creation and destruction defined in the Heisenberg picture
\begin{equation}
\label{eq:heis_field}
\widehat\psi({\bf x}t)=e^{i\widehat Ht}\widehat\psi^\dagger({\bf x}0)e^{-i\widehat Ht}\,
\end{equation}
it is then possible to observe a resemblance between the thermal weight factor $e^{\beta\widehat H}$ and the time evolution operator $e^{i\widehat Ht}$ when considering the imaginary time domain $t=-i\beta$. If one includes the expression (\ref{eq:heis_field}) in the definition of the correlation functions Eqs.(\ref{eq:corr_creat}) and (\ref{eq:corr_destr}), it can then be checked that for a certain imaginary time domain there is absolute convergence of the two expressions, specifically in the intervals $-i\beta<t-t'<0$ for $g^>$ and $0<t-t'<i\beta$ for $g^<$. Furthermore, it can be shown that the two correlation functions are related to one another at one of their imaginary time boundaries, providing the important relation: 
\begin{equation}
\label{eq:qprelation}
g^<({\bf x},t=0;{\bf x'}, t')=e^{\beta\mu}g^>({\bf x},t=-i\beta;{\bf x'},t')\,.
\end{equation}
Thanks to the translational invariance under space and time of an infinite system, the Green's function only depends on the differences ${\bf r}={\bf x'}-{\bf x}$ and $\tau=t-t'$. Consequently, exploiting the quasi-periodicity relation of the Green's function along the imaginary time axis given in Eq.~(\ref{eq:qprelation}), one can write a discrete Fourier representation for the one-body Green's function in the frequency domain:
\begin{equation}
g({\bf r},\tau)=\int \frac{{\rm d}^3p}{(2\pi)^3}e^{i{\bf p}{\bf r}}\frac{1}{-i\beta}\sum_\nu e^{-iz_\nu\tau}g({\bf p},z_\nu)\,,
\end{equation}
where $z_\nu=\frac{\pi\nu}{-i\beta}+\mu$ are the Matsubara frequencies. The Fourier coefficients are then given by the inverse transformation:
\begin{equation}
\label{eq:Fouriercoeff}
g({\bf p},z_\nu)=\int {\rm d}^3r\int_0^{-i\beta}{\rm d}\tau\,e^{-i{\bf p}{\bf r}+iz_\nu\tau}g({\bf r},\tau)\,.
\end{equation}
These coefficients are evaluated for an infinite set of complex frequencies $z_\nu$, corresponding to the imaginary time domain, however one would like to understand the properties of the physical propagator, i.e. in the real time and frequencies domain. To do so let's go back to the expressions of the correlation functions, Eqs.~(\ref{eq:corr_creat}) and (\ref{eq:corr_destr}), and write down their Fourier transform:
\begin{eqnarray}
\label{eq:FTpart}
g^>({\bf p},\omega) &=& i\int{\rm d}^3r\int_{-\infty}^{+\infty}{\rm d}\tau\,e^{-i{\bf pr}+i\omega t}g^>({\bf r},\tau)\,,\\
\label{eq:FThole}
g^<({\bf p},\omega) &=& -i\int{\rm d}^3r\int_{-\infty}^{+\infty}{\rm d}\tau\,e^{-i{\bf pr}+i\omega t}g^<({\bf r},\tau)\,.
\end{eqnarray}
These two quantities now define the spectral probability to attach or remove a particle with an energy $\omega$ from the momentum state {\bf p} of the many-body system (we now omit for simplicity the spin/isospin quantum numbers). The sum of these two functions defines a positive quantity which we call the spectral function:
\begin{equation}
\label{eq:spec_fun}
A({\bf p},\omega)=g^>({\bf p},\omega)+g^<({\bf p},\omega)\,.
\end{equation}
An important feature of the spectral function is that it fulfills the sum rule
\begin{equation}
\int_{-\infty}^{+\infty}\frac{{\rm d}\omega}{2\pi}A({\bf p},\omega)=1\,,
\end{equation} 
and this is also a reason why one can speak about probabilities when referring to the spectral function. We will show below how this quantity relates to Eqs.~\eqref{eq:SpSh}.

Inserting Eqs.(\ref{eq:FTpart}) and (\ref{eq:FThole}) in Eq.~(\ref{eq:qprelation}), we can write the Fourier transform of the periodicity condition
\begin{equation}
g^>({\bf p},\omega)=e^{\beta(\omega-\mu)}g^<({\bf p},\omega)\,,
\end{equation}
and considering the definition of the spectral function, we can write the correlation functions in momentum/frequency:
\begin{eqnarray}
\label{eq:FTpart_spec}
g^<({\bf p},\omega) &=& f(\omega)A({\bf p},\omega)\,,\\
\label{eq:FThole_spec}
g^>({\bf p},\omega) &=&[1-f(\omega)]A({\bf p},\omega)\,,
\end{eqnarray}
where $f(\omega)=\frac{1}{1+e^{\beta(\mu-\omega)}}$ is the Fermi-Dirac distribution function. These expression show that, once the spectral function is known, it is easy to access the correlation functions. A similar relation can be found between the spectral function and the Fourier coefficients in Eq.(\ref{eq:Fouriercoeff}):
 \begin{equation}
\label{eq:FT_fullprop}
g({\bf p},z_\nu)=\int^{+\infty}_{-\infty} \frac{{\rm d}\omega'}{2\pi} \frac{A({\bf p},\omega')}{z_\nu-\omega'}\,.
\end{equation}
The previous expression is performed for an infinite set of frequencies in the imaginary time domain. However we would like to extend this to the entire complex plane, especially close to the real-time domain, to which we are most interested. It can be demonstrated that this analytical continuation is possible and one can safely replace $z_\nu=z$~\cite{ch11_Baym1961}. This property that relates the Green's function $g({\bf p},z)$ in the complex plane to the spectral function $A({\bf p},\omega)$ is called the spectral decomposition of the single-particle propagator. A similar Fourier transform can be written for the retarded propagator defined in Eq.~(\ref{eq:retar_prop}):
 \begin{equation}
\label{eq:FT_retarprop}
g^R({\bf p},\omega)=\int^{+\infty}_{-\infty} \frac{{\rm d}\omega'}{2\pi} \frac{A({\bf p},\omega')}{\omega_+-\omega'}\,,
\end{equation}
with $\omega_+=\omega+i\eta$. At this point, exploiting the Plemelj identity,
\begin{equation}
\frac{1}{\omega\pm i\eta}=\frac{\pazocal P}{\omega}\mp i\pi\delta(\omega)\,,
\end{equation}
one can separate the real and imaginary part of the retarded propagator, and it can be checked that the imaginary part of the retarded propagator coincides with the spectral function up to a factor:
\begin{equation}
\label{eq:spec_img}
A({\bf p},\omega)=-2{\rm Im}g({\bf p},\omega_+)\,.
\end{equation}
The Dyson's equation given in Eq.~\eqref{eq:Dyson} can be rewritten in an algebraic form as follows:
\begin{equation}
g({\bf p},\omega_+)=\frac{1}{[g^{(0)}({\bf p},\omega_+)]^{-1}-\Sigma^\star({\bf p},\omega_+)}\,,
\label{eq:G_algebraic}
\end{equation}
and combining Eq.~(\ref{eq:G_algebraic}) with Eq.~(\ref{eq:spec_img}),
one can express the spectral function as:
\begin{equation}
A({\bf p},\omega)=\frac{-2\mathrm{Im}\Sigma^\star({\bf p},\omega_+)}
{[\omega-\frac{p^2}{2m}-\mathrm{Re}\Sigma^\star({\bf p},\omega)]^2+[\rm{Im}\Sigma^\star({\bf p},\omega_+)]^2} \,.
\label{eq:spectral_fun}
\end{equation}
The numerical self-consistent solution that one has to perform is the one given by Eq.~(\ref{eq:spectral_fun}). Self-consistency is achieved once the spectral function inserted in the calculation of the irreducible self-energy is equal to the one obtained by solving Eq.~(\ref{eq:spectral_fun}).

Before going on, it is interesting to see that in the limit of zero temperature, the spectral decomposition of the one-body propagator given in Eq.~(\ref{eq:FT_fullprop}) can be separated into two pieces:
\begin{equation}
g({\bf p},\omega)=\int_{\varepsilon_\textrm F}^\infty\mathrm d\omega'\frac{S^p({\bf p},\omega')}{\omega-\omega'+i\eta}
+\int_{-\infty}^{\varepsilon_\textrm F}\mathrm d\omega'\frac{S^h({\bf p},\omega')}{\omega-\omega'-i\eta}\,,
\label{eq:Lehm_infty}
\end{equation}
The $S^p({\bf p},\omega)$ and $S^h({\bf p},\omega)$ correspond to the particle and hole spectral functions, which were already introduced in Eqs.~(\ref{eq:SpSh}). Notice however that, unlike in Eqs.~(\ref{eq:SpSh}), we have introduced one single Fermi energy $\varepsilon_\textrm F$ ($\varepsilon_\textrm F$ = $\varepsilon_0^+$ = $\varepsilon_0^-$) in the integrals domain. This is a consequence of using a plane-wave basis in momentum space. In an uncorrelated system, this energy defines the last filled level and hence corresponds to the energy needed to remove a particle from the many-body ground state. In the case of an interacting system, not in the superfluid nor in the superconducting phase, $\varepsilon_\textrm F$ equals the chemical potential $\mu$, and corresponds to the minimum energy necessary to add or remove a particle to/from the many-body system. Consequently, the expression for the spectral function given in Eq.~(\ref{eq:spectral_fun}) can be divided into two parts:
\begin{eqnarray}
S^p({\bf p},\omega)&=&-\frac{1}{\pi}\frac{\mathrm{Im}\Sigma^\star({\bf p},\omega)}
{(\omega-\frac{p^2}{2m}-\mathrm{Re}\Sigma^\star({\bf p},\omega))^2+(\rm{Im}\Sigma^\star({\bf p},\omega))^2} \quad \omega>\varepsilon_\textrm F\,,\qquad\,\,\,
\label{eq:Sp_self}
\\ 
S^h({\bf p},\omega)&=&\frac{1}{\pi}\frac{\mathrm{Im}\Sigma^\star({\bf p},\omega)}
{(\omega-\frac{p^2}{2m}-\mathrm{Re}\Sigma^\star({\bf p},\omega))^2+(\rm{Im}\Sigma^\star({\bf p},\omega))^2} \quad\,\,\,\, \omega<\varepsilon_\textrm F\,,\quad
\label{eq:Sh_self}
\end{eqnarray}
resembling the structure of Eqs.~(\ref{eq:SpSh}).

In the next subsection we will briefly sketch the steps that have to be taken to perform the numerical implementation of Eq.~(\ref{eq:spectral_fun}) at finite temperature.

\subsection{Numerical implementation: details and tips}
We show in Fig.~\ref{fig:num_impl} a schematic representation of how the numerical code works when considering both two-body and three-body forces. %Depending on the starting point of the iteration procedure, $\sim15$ iterations are needed to obtain converged results. These could be less for low densities and high temperatures, and if the starting point is a converged iteration for another density, temperature state of the system. The number of iterations for convergence increase for lower temperatures and high densities. 
The fundamental quantities that one has to calculate are the first-order two-body Green's function, the in-medium $T$-matrix and the irreducible self-energy, which are depicted in the the three blue boxes in Fig.~\ref{fig:num_impl}, with their respective Feynman diagrams (rules for writing Feynman diagrams are given in Appendix 1). Diagrams are a direct way to write down the complicated mathematical expressions that one has to solve numerically. The one-body and two-body effective nuclear potentials are depicted in the center orange boxes. These are similar to the contributions which where given in Fig.~\ref{fig:EffOps}, except for the first term in the one-body effective potential, which is zero in infinite matter, and the last term which in our case is only the first-order two-body averaged three-nucleon force, with the correct multiplying factor, as in Eq.~\eqref{eq:ueff_3b_first}~\cite{ch11_Carbone2013NM3nf,ch11_Carbone2014}. As can be seen from the graph in Fig.~\ref{fig:num_impl}, all quantities, in blue or orange boxes, are fed with the spectral function, the left red box, which is then iteratively calculated solving Dyson's equation, Eq.~(\ref{eq:spectral_fun}). The self-consistency reached is checked by comparing the chemical potential from the previous iteration with the new one calculated after solving Eq.~(\ref{eq:spectral_fun}).

For clarity, we will distinguish between the wording \emph{calculation} and \emph{iteration}: we will refer to \emph{calculation} as the set of several \emph{iterations} necessary to get to a converged result for the spectral function, so an \emph{iteration} is exactly what is depicted in Fig.~~\ref{fig:num_impl}. For a more in depth explanation of the numerical details the reader can refer to Refs.~\cite{ch11_Frick2004PhD,ch11_Rios2007PhD}.\\


Each calculation is performed at a specific density $\rho$ and temperature $T$ of the system. One starts the first iteration with a guess of the spectral function, which is given in terms of the imaginary, Im$\Sigma^\star({\bf p},\omega)$, and real part, Re$\Sigma^\star({\bf p},\omega)$, of the irreducible self-energy. One usually starts with a converged solution for these quantities at a different point in $\rho, T$ of the system.

 
\begin{figure}[t]
\begin{center}
\includegraphics[width=1.0\textwidth]{Chapter11-figures/numerical_scgf_diagrams.pdf}
\caption{The structure of a ladder SCGF calculation including both two-body and three-body forces through the definition of effective interactions (see text for details). Each quantity is also represented via the corresponding Feynman diagram.}
\label{fig:num_impl}
\end{center}
\end{figure}

\begin{itemize}
\item {\bf Numerical tips for the $({\bf p},\omega)$ meshes:}
The mesh of the single-particle momentum ${\bf p}$ for the self-energy is adjusted during the first iteration to be more dense around the Fermi momentum $p_{\rm F}$ corresponding to the specific density considered: $N_{\bf p}=70$ mesh points are enough, considering linear meshes at low momentum and around the Fermi momentum, and a logarithmic mesh for the tail all the way up to a value $\sim10p_{\rm F}$. The mesh in the single-particle energy $\omega$ has to be very dense because of the complicated features of the spectral function, especially at the quasi-particle peak. However storing a dense mesh at each iteration is demanding in terms of memory; for this reason one saves separately the imaginary and real part in a sparse mesh, $N_{\omega}=~6000$ linear mesh points from $\approx$[-2000:15000] MeV, and then interpolates these quantities during the iterations to a denser meshes, $N_{\omega}=~30000$ linear mesh points, in order to have a good description of the spectral function in the energy domain. However, as it will be explained later on, the mesh in energy is adjusted in different ways during the iteration according to the specific quantities that one has to calculate (two-body propagator, T-matrix, etc.).
\end{itemize}

We will now enumerate the steps to perform a complete iteration.\\
\noindent {\bf 1.} The density $\rho$, temperature $T$, spectral function $A({\bf p},\omega)$ and single-particle spectrum $\varepsilon({\bf p})$ are the inputs to calculate the first quantity: the chemical potential $\mu$. This is done by exploiting the sum rule for the density:
\begin{equation}
\rho=\nu\int\frac{{\rm d}{\bf p}}{2\pi^3}\int_{-\infty}^{+\infty}\frac{{\rm d}{\omega}}{2\pi}A({\bf p},\omega)f(\omega,\mu)\,.
\label{eq:chempot_micro}
\end{equation}
\begin{itemize}
\item {\bf Numerical tips for the $\mu$ mesh:} One chooses a sample mesh of chemical potentials $\mu$ to insert in the Fermi-Dirac function $f(\omega,\mu)$ and then solves Eq.~(\ref{eq:chempot_micro}). For each mesh point $\mu$ one gets a value of density $\rho$. Parametrizing $\rho$ as a function of $\mu$, one can then encounter the value of $\mu$ which corresponds to the external density. The mesh of $\mu$ can be initially distributed around the value of the single-particle spectrum calculated at $p_{\rm F}$ (in the case of a zero temperature calculation it holds the relation $\varepsilon(p_{\rm F})=\mu$), and then adjust the mesh testing if the limits include the value of the external density. However, it must be noted that the single-particle spectrum $\varepsilon({\bf p})$ comes from a previous calculation, as also the chemical potential which enters the spectral function in Eq.~(\ref{eq:chempot_micro}), so its has to be kept in mind that in the solution of Eq.~(\ref{eq:chempot_micro}) one is considering two different chemical potentials, which will end up coinciding at the end of the convergence loop of iterations.
\end{itemize}

\noindent {\bf 2.} As a second step one solves a self-consistent equation in the energy to find a new single-particle spectrum:
\begin{equation}
\varepsilon({\bf p})=\frac{p^2}{2m}+{\rm Re}\Sigma({\bf p},\varepsilon({\bf p}))\,,
\end{equation}
which will be used throughout the iteration. 
 
\noindent{\bf 3.} At this point the imaginary part of the first-order two-body Green's function can be calculated. The first order approximation of the two-body propagator corresponds to the independent propagation of two fully dressed particles. This includes two terms, a direct and an exchange one, as depicted diagrammatically in Fig.~\ref{fig:num_impl}. The $G^{II,f}_{pphh}$ was already defined in Eq.~\eqref{eq:FeynGalvsG2}, however here we consider dressed single-particle propagators, $g^{(0)}\rightarrow g$.
The imaginary part of this quantity reads:
\begin{equation}
\label{eq:gII_imag}
{\rm Im}G^{II,f}_{pphh}(\Omega_+;{\bf p},{\bf p'}) = -\frac{1}{2}\int_{-\infty}^{+\infty}\frac{{\rm d}{\omega}}{2\pi}A({\bf p},\omega)A({\bf p'},\Omega-\omega)[1-f(\omega)-f(\Omega-\omega)]\,.
\end{equation}
where $\Omega_+$ is the sum of the energies of the two particles close to the real axis. This expression is derived from a sum over Matsubara frequencies of a function with a double pole on the real-energy axis via use of the Cauchy theorem~\cite{ch11_Rios2007PhD}.
\begin{itemize}
\item {\bf Numerical tips for the $\omega$ mesh :} The integrand of Eq.~(\ref{eq:gII_imag}) will be particularly hard to resolve in the energy range where the two spectral functions are peaked. It can be demonstrated that performing a convenient variable change, one is safe with defining a mesh in energy accurately distributed around two specific regions, e.g. $\tilde\omega=0$ and $\tilde\omega=\tilde\Omega$ (for details see Ref.~\cite{ch11_Rios2007PhD}). To obtain the spectral function in this specific mesh one interpolates the imaginary and real self-energies to this mesh and then solves Eq.~(\ref{eq:spectral_fun}).
\end{itemize}

\noindent{\bf 4.} From the imaginary part it is then possible to obtain the real-part of the first-order two-body Green's function via a dispersion relation:
\begin{equation}
{\rm Re}G^{II,f}_{pphh}(\Omega_+;{\bf p},{\bf p'}) = -{\pazocal P}\int_{-\infty}^{+\infty}\frac{{\rm d}{\Omega'}}{\pi}\frac{{\rm Im}G^{II,f}_{pphh}(\Omega_+;{\bf p},{\bf p'})}{\Omega-\Omega'}\,.
\end{equation}

\noindent{\bf 5.} An angle average is then necessary to calculate $G^{II,f}_{pphh}$. This average is necessary to circumvent the coupling of partial waves with different total angular momentum $J$ which appear in $G^{II,f}_{pphh}$. The average is performed over the angle formed by the center of mass momentum ${\bf P}={\bf p}+{\bf p'}$ and the relative momentum of the two nucleons ${\bf k}=({\bf p}-{\bf p'})/2$. This strategy will facilitate the solution of a Lippmann-Schwinger type equation to evaluate the effective interaction in the medium, known as the in-medium T-matrix. The average reads:
\begin{equation}
\overline{G^{II,f}_{pp,hh}}(\Omega;{\bf P},{\bf k})=\frac{1}{2}\int_{-1}^{+1}{\rm d\,cos}\theta G^{II,f}_{pp,hh}(\Omega;|{\bf P}/2+{\bf k}|,|{\bf P}/2-{\bf k}|)\,.
\end{equation}

\noindent{\bf 6.} The two-body angle-averaged propagator together with the nuclear potential are then used to obtain the in-medium T-matrix. As explained previously, this is a ladder resummation of particle-particle and hole-hole diagrams, this differs with respect to the Brueckner G-matrix presented in Chapter 8 because it includes hole-hole diagrams and considers the full off-shell description of the spectral function. As seen from Fig.~\ref{fig:num_impl}, the potential to be included is the sum of a bare two-body potential and an averaged three-body one. Details on the numerical solution for the averaged three-body force are given in the next section, while working equations for three-nucleon chiral forces will be given in Appendix 2. The Lippmann-Schwinger type equation to be solved reads:
\begin{equation}
\label{eq:t-matrix}
\langle{\bf k'}|{\rm T}(\Omega_+,{\bf P})|{\bf k}\rangle=\langle{\bf k'}|\widetilde V|{\bf k}\rangle+\int {\rm d}{\bf k_1}\langle{\bf k'}|\widetilde V|{\bf k_1}\rangle\overline{G^{II,f}_{pp,hh}}(\Omega;{\bf P},{\bf k_1})\langle{\bf k_1}|{\rm T}(\Omega_+;{\bf P})|{\bf k}\rangle\,.
\end{equation}
This is a one dimensional integral equation for each allowed combination of $J,\,S,\,T$, and at most two coupled values of $L$, due to the tensor component of the nuclear interaction. It must be noted that the nuclear interaction $\widetilde V$ considered in Eq.~(\ref{eq:t-matrix}) is the effective two-body operator given in Eq.~(\ref{eq:V_eff}). By means of a discretization procedure, the equation for the $T$-matrix is converted into a complex matrix equation which can be solved via standard numerical techniques~\cite{ch11_Rios2007PhD}. A matrix inversion has to be performed to solve this equation. This can be quite demanding if the dimension of the matrix is large. 
\begin{itemize}
\item {\bf Numerical tips for the ${\bf k_1}$ and $\Omega$ mesh:} It is important to sample in a correct manner the number of integration mesh points without loosing physical information. This is achieved by sampling conveniently the region where $G^{II,f}_{pp,hh}$ is maximum in the relative momentum ${\bf k_1}$, so for $\Omega>0$ close to the pole $k_1=\sqrt{m\Omega}$, and the high relative momentum region where, due to correlations, $G^{II,f}_{pp,hh}$ might not be negligible. Furthermore there is a node for $\Omega=2\mu$ present in the $T$-matrix so an accurate mesh for the bosonic energies around this value is needed for the forthcoming calculation of the self-energy.
\end{itemize}
At low temperatures, the appearance of bound states signals the onset of the pairing instability. This would directly appear as a pole in the matrix which has to be inverted to solve the Lippmann-Schwinger equation, for ${\bf P}=0$ and $\Omega=2\mu$~\cite{ch11_Thouless1960}. However, this should be seen only below a critical temperature which is around $T_c\sim4$ MeV. The calculations should not go below this border line in temperature. Especially in the case of symmetric nuclear matter, convergence at this temperature and for increasing density starts to be slow and difficult to achieve. This is due to the neutron-proton pairing in the coupled $^3S_1-^3D_1$ channel. In pure neutron matter, where this channel is not available, convergence is good for higher densities, and even for lower temperatures. 

\noindent{\bf 7.} The remaining step in the SCGF method is the calculation of the self-energy from the T-matrix. The first quantity to be obtained is the imaginary part of the self-energy, Im$\Sigma^\star$:
\begin{equation}
{\rm Im}\Sigma^\star({\bf p},\omega_+)=\int\frac{{\rm d}{\bf p'}}{2\pi^3}\int_{-\infty}^{+\infty}\frac{{\rm d}{\omega'}}{2\pi}\langle{\bf pp'}|{\rm Im}{\rm T}(\omega+\omega'_+,{\bf P})|{\bf pp'}\rangle A({\bf p'},\omega')[f(\omega')+b(\omega+\omega')]\,;
\end{equation}
We recall that this expression is also obtained from a summation over Matsubara frequencies of a function with two poles on the real energy axis~\cite{ch11_Rios2007PhD}. 
\begin{itemize}
\item {\bf Numerical tips for the ${\bf p'}$  and $\omega'$ meshes:} A momenta and energy integrals have to be performed, taking special care for the pole in energy of the Bose function $b(\Omega)$. This pole is canceled by the node we had previously mentioned in the T-matrix, for this reason it comes in hand that we had previously defined a convenient mesh for $\Omega$ around the node. 
\end{itemize}
 
\noindent{\bf 8}. The real part of the self-energy is then obtained by means of a dispersion relation from the imaginary part: 
\begin{equation}
{\rm Re}\Sigma^\star({\bf p},\omega_+)=\Sigma^{(\infty)}({\bf p})-{\pazocal P}\int_{-\infty}^{+\infty}\frac{{\rm d}{\lambda}}{\pi}\frac{{\rm Im}\Sigma^\star({\bf p},\lambda_+)}{\omega-\lambda}\,.
\label{eq:self_en}
\end{equation}
The $\Sigma^{(\infty)}$ is the correlated Hartree-Fock part of the single-particle self-energy as described by Eq.~\eqref{eq:UeffSig}, where we now consider the two-body averaged three-body interaction as given in Eq.~\eqref{eq:ueff_3b_first}. Eq.~(\ref{eq:self_en}) can then be explicitly written as:
\begin{equation}
\label{eq:HF_self}
\Sigma^{(\infty)}({\bf p})=\int\frac{{\rm d}{\bf p'}}{2\pi^3}n({\bf p'})\Big[\langle{\bf pp'}|V^{\rm 2NF}|{\bf pp'}\rangle +\frac{1}{2}\langle{\bf pp'}|\widetilde V^{\rm 3NF}|{\bf pp'}\rangle\Big] \,,
\end{equation}
where $V^{\rm 2NF}$ and $\widetilde V^{\rm 3NF}$ correspond respectively to the first and second terms in Eq.~(\ref{eq:V_eff}). $\widetilde V^{\rm 3NF}$ is a one-body averaged three-nucleon force, detailed description on how to calculate this quantity and the momentum distribution $n({\bf p})$, together with an additional numerical sample code, are given in Sec.~\ref{subsec:average_3bf}. The averaged three-body force is addressed in Appendix 2 in the specific case of chiral three-nucleon forces appearing at leading order in chiral effective field theory.

Finally, via Eq.~(\ref{eq:spectral_fun}) the spectral function can be obtained and the procedure starts again from Eq.~(\ref{eq:chempot_micro}) until a consistent result is achieved for the chemical potential. It must to be kept in mind that, according to the mesh points in which the spectral function is needed, the interpolation is done on the imaginary and real part of the self-energy, and not directly on the spectral function. This is done in order to avoid incorrect samplings of the structure of the spectral function which could induce numerical inaccuracies. We must point out that the energy mesh for the evaluation of the spectral function must be accurate enough to reproduce not only the quasi-particle peak region, but furthermore the low and high-energy tails which characterize the spectral function.   

To calculate the total energy of the system, we make use of the modified Koltun sum rule given in Eq.~(\ref{eq:Koltun_hW}). Consequently we need to evaluate the expectation value of the three-body operator $\langle \widehat W\rangle$. As already stated in Sec.~\ref{sec:scgf_obs}, we approximate this expectation value to its first order term, which in infinite matter corresponds to the integral over three independent but fully correlated momentum distributions $n({\bf p})$. The integral to be evaluated is given by the expression:
\begin{equation}
\langle \widehat W\rangle\simeq\frac{\nu}{\rho}\frac{1}{6}\int \frac{{\rm d}{\bf p}}{(2\pi)^3}\int\frac{{\rm d}{\bf p'}}{(2\pi)^3}n({\bf p})n({\bf p'})\langle{\bf pp'}|\widetilde V^{\rm 3NF}|{\bf pp'}\rangle\,,
\label{3B_exp}
\end{equation}
with $\nu$ the degeneracy of the system. So once the averaged three-body force is known $\widetilde V^{\rm 3NF}$, the total energy per nucleon of the system can be calculated via the modified Koltun sum rule:
\begin{equation}
\frac{E}{A}=\frac{\nu}{\rho}\int\frac{{\rm d}{\bf p}}{(2\pi)^3}\int\frac{{\rm d}\omega}{2\pi}\frac{1}{2}\Big\{\frac{p^2}{2m}+\omega\Big\}A({\bf p},\omega)f(\omega)-\frac{1}{2}\langle \widehat W\rangle\,,
\end{equation}
which is analogous to Eq.~(\ref{eq:Koltun_hW}).


\subsection{Averaged three-body forces: numerical details.}
\label{subsec:average_3bf}
The inclusion of one-body averaged three-nucleon forces $\widetilde V^\mathrm{3NF}$ enters the calculations presented in the previous section, as it is shown in Eqs.~(\ref{eq:t-matrix}) and (\ref{eq:HF_self}). This average corresponds to a trace over the spin/isospin quantum numbers of the averaged particle, in this case the third particle, and by an integration over its momentum ${\bf p_3}$:
\begin{equation}
%\nn &&
\langle {\bf p'_1 p'_2}|\widetilde V^\mathrm{3NF}|{\bf p_1 p_2}\rangle_A =
\mathrm{Tr}_{\sigma_3}\mathrm{Tr}_{\tau_3}
\int \frac{{\mathrm d}{\bf p}_3}{(2\pi)^3}n({\bf p_3})
%\\ && \quad\quad\quad\quad\quad
\langle {\bf p_1' p_2' p_3}|V^\mathrm{3NF}
(1-P_{13}-P_{23})
|{\bf p_1 p_2 p_3}\rangle_{A_{12}}\ \,,
\label{eq:dd3bf_new}
\end{equation}
where we have omitted the spin/isospin indices in the potential matrix elements for a better view. In Eq.~\ref{eq:dd3bf_new} the ket on the right hand side is antisymmetrized only with respect to particles 1 and 2, i.e. ${\rm A}_{12}=(1-{\rm P}_{12})/2$, part which is not affected by the averaging procedure over particle 3; ${\rm P}_{ij}=(1+\boldsymbol \sigma_i\cdot\boldsymbol \sigma_j)(1+\boldsymbol \tau_i\cdot\boldsymbol \tau_j)/4$ is the permutation operator of momentum and spin/isospin quantum numbers of particles i and j. The momentum distribution that appears in Eq.~(\ref{eq:dd3bf_new}) can be obtained directly from the spectral function, via the relation:
\begin{equation}
\label{eq:mom_dist}
n({\bf p})=\int_{-\infty}^{+\infty}\frac{{\rm d}\omega}{2\pi}A({\bf p},\omega)f(\omega)
\end{equation}

Let us give some details on the numerical implementation for the calculation of Eq.~\eqref{eq:dd3bf_new} regarding the mesh of the internal momentum ${\bf p_3}$ and the calculation of the momentum distribution $n({\bf p_3})$ via Eq.~\eqref{eq:mom_dist}:
\begin{itemize}
\item We start with the definition of the mesh necessary to calculate the integral over the internal momenta ${\bf p_3}$. Considering that in the integral we deal with a dressed distribution function $n({\bf p_3})$, which may have populated states at high momentum, we need to cover momenta up to a certain high value in which it is sure that the $n({\bf p_3})$ has reached zero. One can choose an arbitrary number $imesh$ of Gauss-Legendre meshes (in the example shown below, $imesh=4$), with each mesh spanning a region of width $2/3 p_{\rm F}$. This width is chosen to cover accurately the behavior of the distribution function from momenta lower to those higher than $p_\textrm F$. Finally, high momenta are reached through a tangential mesh. We have 100  points in the Gauss-Legendre meshes, and 50 in the tangential one. 
\item We then need to calculate the momentum distribution function via solution of Eq.~(\ref{eq:mom_dist}); in order to do so the spectral function coming from the previous iterative step has to be used. The values of the imaginary and real part of the self-energy are stored at each iterative step for different points in the energy and momentum space: for the momentum mesh we have $N_k=70$, for single-particle momenta going from 0 to 3000 MeV; for the energy mesh we interpolate through a spline the values of the imaginary and real part of the self-energy to a fine linear energy mesh of $N_{\omega,\textrm{spline}}=30000$ in a smaller range of values from $\approx$[-2000:5000] MeV. These values are used to calculate the spectral function (see Eq.~(\ref{eq:spectral_fun})) necessary to evaluate correctly Eq.~(\ref{eq:mom_dist}). The integral in Eq.~(\ref{eq:mom_dist}) is then performed via a trapezoidal integral in the energy range. Finally we perform a linear interpolation of the obtained values of $n({\bf p})$ to the mesh of ${\bf p_3}$ defined for the integration of the quantities in the averaged force, Eq.~\eqref{eq:dd3bf_new}. Extrapolated values of $n({\bf p_3})$ are set to zero.
\end{itemize}

\noindent
Here we show a simple Fortran code to perform the previous two steps (gauss() is a common routine to perform a gauss-legendre mesh; splin() and splin2() are common routines to perform spline interpolations; linint() is a common routine to perform linear interpolation):

\vspace*{0.3cm}
\lstset{alsolanguage=[90]Fortran}
\begin{lstlisting}
    ! ... MOMENTA MESH FOR INTEGRALS OVER MOMENTUM DISTRIBUTION 

    write(*,*) "Correlated distribution function for averaged 3BF integration"

   ! choose number of mesh regions for momenta k3, (imesh-1) gauss set + 1 tangent set for farther points
   imesh = 4 
   
   ! choose number of points for gauss and tangent sets
    Nk1=100   ! gauss
    Nk2=50      !tangent
    Nk3=(imesh-1)*Nk1+Nk2     ! total number of mesh points 
    
    ALLOCATE(xk3(Nk3),wk3(Nk3))
    ALLOCATE(xaux(Nk1),waux(Nk1))   ! always allocate the biggest between Nk1 and Nk2
    
    ! initialize variables
    xk3=0d0
    wk3=0d0
    
    ! first mesh point
    kin = 0d0
     
    do im = 1, imesh  ! loop of regions
    
    	! initialize auxiliary variables at each region
    	xaux=0d0 
    	waux=0d0
    
    	! create different mesh types according to region
   	if(im < imesh) then
    		kfin = im*(2d0/3d0)*kF  ! set final point of mesh region
        
    		! ... gaussian set of points for momenta k3 from kin to kfin
     		call gauss(kin,kfin,Nk1,xaux,waux)
		
		! copy points to final vector for mesh k3
		do ik3=1,Nk1
			xk3(ik3+(im-1)*Nk1)= xaux(ik3)
			wk3(ik3+(im-1)*Nk1)= waux(ik3)
		enddo

	elseif(im == imesh) then
		kfin=	1
	
		! .. create a tangent set
	 	call gauss(0d0,1d0,Nk2,xaux,waux)

     		c=10d0*kF/tan(pi/2.d0*xaux(Nk2))
     		do ik3=1,Nk2
        			xk3(ik3+(im-1)*Nk1)=c*tan(pi/2.d0*xaux(ik3))+kin
        			xxw=cos(pi/2.d0*xaux(ik3))
        			xxw=xxw*xxw
        			wk3(ik3+(im-1)*Nk1)=pi/2.d0*c/xxw*waux(ik3)
     		enddo
		
	endif
	
	kin = kfin ! set last point of previous region to first point of next region
	
	enddo

      ! ... obtainin correlated momentum distribution

     ! ... FINE ENERGY MESH WHERE CALCULATIONS ARE DONE
     ! ... allocate energy mesh for calculation of momentum distribution
     N_fine=30000
     ALLOCATE(xmom(N_fine))
     
     wi=-2000.d0 !MeV  initial energy for spectral function
     wf=5000.d0  !MeV  final energy for spectral function
     dw=(wf-wi)/dble(N_fine-1)
        
     ! ... LOOP OVER KMESH        
     do ik=1,Nk
           
        edk=xkmesh(ik)**2/(2.d0*xmass)  ! kinetic spectrum
           
        do iw=1,Nwac
           auxre(iw)=xreal_sigma(ik,iw)    ! real part of self-energy
           auxim(iw)=ximag_sigma(ik,iw)  !imaginary part of self-energy
        enddo
           
        ! obtain derivatives of self-energy for later splines
        iiim=1
        iire=1
        call spline(w_actual,auxim,Nwac,yspl,yspl,d2im)  
        call spline(w_actual,auxre,Nwac,yspl,yspl,d2re)   
       
        ! ... LOOP OVER WFINE
        do iif=1,N_fine
           w_fine = wi + dble(iif-1)*dw  
           wfine(iif)=w_fine
           fdfine=fermi(t,xmu,w_fine)  !Fermi-Dirac distribution
              
           ! .. Spline interpolation in fine energy mesh
           call splin2(w_actual,auxim,d2im,Nwac,w_fine,ximsig,iiim)
           call splin2(w_actual,auxre,d2re,Nwac,w_fine,xresig,iire)
              
           ! ... Spectral function
           sf=-ximsig/( (w_fine - edk - xresig)**2 + ximsig**2 )/pi
              
           ! ... momentum distribution
           xmom(iif)=sf*fdfine
           
        enddo ! END LOOP OVER WFINE

		! trapezoidal rule for integration over energy
        ieq=1
        call trapz(w_fine,xmom,N_fine,ieq,mom)
        xmk(ik)=mom

     enddo ! LOOP OVER MOMENTA
     
     ! ... interpolation of momentum distribution to mesh for integrals
     call linint(xkmesh,xmk,Nk,xk3,xnk3,Nk3)

     ! ... set extrapolated values of n(k) to zero
     do ik3=1,Nk3
        xnk0=xk3(ik3)
        if(xnk0.gt.xkmesh(Nk)) xnk3(ik3)=0d0
        if(xnk0.lt.0d0) xnk3(ik3)=0d0
     enddo

     DEALLOCATE(xaux,waux,xmom,xmk)
  
 \end{lstlisting}
 \vspace*{0.3cm}

We note that the chemical potential $\mu$ enters the calculation of the averaged three-body force, via the momentum distribution, for this reason it is best to perform Eq.~\eqref{eq:dd3bf_new} after step one of the iterative procedure presented in the previous section. For futher details on including three-body forces in a SCGF infinite matter calculation we refer the reader to Ref.~\cite{ch11_Carbone2014PhD}.


\section{Concluding remarks}
This chapter concludes an overview of the major methods based on Fock space, which are covered in chapters 8, 10 and 11 of this book. All these approaches have the common feature that their computing requirements scale only polynomially with the increase of particle number. This feature has permitted to push {\em ab-initio} studies of atomic nuclei up to medium-mass isotopes: a progress that would have seemed unthinkable until just a decade ago. 
%

Here, we have focused on many-body Green's function theory, which is arguably the most complex of these formalisms but it has the advantage of providing a unique and global view of the many-particle structure and dynamics.   The spectral function is extracted directly from the physics information contained in the one-body Green's function and gives an intuitive understanding of correlations (beyond shell structure) of the system. Besides, expectation values of observables can be calculated easily, including binding energies.
%
The formalism of SCGF is so vast that even a dedicated monograph would not be able to cover it in full. In this chapter, we have focused on presenting the two most important techniques that are currently used in modern {\em ab-initio} nuclear theory.
 In the first case, the Algebraic Diagrammatic Construction method proves to be particularly suited for the study of finite nuclei, but can as well be applied to infinite matter, as was demonstrated in this chapter. In the second case, we looked at how one can solve the  Dyson's equation directly in momentum space for extended systems.  This is an important aspect since the formalism allows to construct a fully-dressed propagator at finite temperature, which grants the method to be thermodynamically consistent, preserving all the fundamental laws of conservation. 
%
%
For these cases we also discussed the most relevant steps and knowhow necessary for implementing SCGF calculations. Furthermore, we provided working numerical codes that can solve the same toy models used as examples throughout this book: a four-level pairing Hamiltonian and neutron matter with a Minnesota force.   While these applications are simple, the codes we provide already contain the most crucial elements and could be easily extended to real applications (in nuclear physics and other fields too!). We hope this chapter can be the starting point for readers interested in working with many-body Green's functions, starting from the sample codes presented and making use of numerous tips provided for the numerical solutions.


What we did not touch upon, due to lack of space, are the most advanced techniques that have been introduced in recent years or that are still under development. Improving accuracy in calculating open shell isotopes, describing excited spectra,  accessing deformed nuclei and describing pairing and superfluidity at finite temperatures are some among the compelling challenges that are to be addressed and that will be crucial to the study of exotic nuclei at future radioactive beam facilities.
%
Likewise, the methods described in this chapter can be extended to novel applications in nuclear physics,  besides the structure and reactions with unstable nuclei. Examples are: understanding the response to electroweak probes and the interaction of high energy neutrinos with matter; the spectral function (and hence the individual behavior) of hyperons in finite nuclei and neutron star matter; how thermodynamic properties of nuclear matter impact stellar evolution.
%%
With still much room for further development, Fock space methods, and the SCGF approach in particular, are possibly the most promising frontiers for advancing first principle computations on large and complex nuclei. All in all, this is an exciting time not only for computational nuclear physics itself  but also for the quest of an accurate understanding of nuclear structure and related topics. 


%In this chapter we presented two techniques based on the use of many-body Green's functions. The great advantage of working within this frame lies in the access one gains to the complete spectral behaviour of the many-body system. In fact, through solution of Dyson's equation, one constructs a non perturbative self-energy which leads to the definition of a fully-dressed propagator. This quantity brings in for free fundamental informations, such as spectroscopic factors or momentum distributions. Furthermore, for positive energies, the spectral function corresponds to the optical potential in elastic scattering. Finally, thanks to specific sum rules, one is able to calculate the total energy of the many-body ground state without much effort.  We have presented numerical solutions for infinite nuclear matter both with the use of a discrete basis in a box as also with a plane-wave basis for the study of the system in thermodynamical limit. In the first case, the Algebraic Diagrammatic Construction method proves to be particularly suited for the study of nuclei, but can be as well applied to infinite matter, as was demonstrated in this chapter. In the second case, through solution of Dyson's equation, one construct a fully-dressed propagator at finite temperature, which grants the method to be thermodynamically consistent, preserving all the fundamental laws of conservation. 

%We hope this chapter can be the starting point for readers interested in working with many-body Green's functions, starting from the sample codes presented and making use of numerous tips provided for the numerical solutions.


\begin{acknowledgement}
We Thank A. Cipollone, W. H. Dickhoff, T. Duguet, K. Hebeler, M. Hjorth-Jensen, H. Muther, A. Polls, A. Rios, J. Schirmer, A. Schwenk, V. Som\`a and D. Van Neck  for several fruitful collaborations and  enlightening discussions over the years.
%
This work was supported by the Science and Technology Facilities Council (STFC) under Grant No. ST/L005743/1, % 
the  Deutsche Forschungsgemeinschaft through Grant SFB 1245
and by the Alexander von Humboldt Foundation through a Humboldt Research Fellowship for Postdoctoral Researchers.
\end{acknowledgement}
%

\section*{Appendix 1: Feynman rules for the one-body propagator and self-energy}
\addcontentsline{toc}{section}{Appendix 1: Feynman rules for the one-body propagator and self-energy}
\label{app:Feyn_rules}

We present in this appendix the Feynman rules associated with the diagrams arising
in the perturbative expansion of Eq.~(\ref{gpert}). The rules are given both in time and energy formulation, and some specific examples will be considered at the end.
We provide the Feynman diagram rules for a given $p$-body propagator. These arise from a trivial generalization of the perturbative 
expansion of the one-body propagator in Eq.~(\ref{gpert})~\cite{ch11_Carbone2013Nov}. 
At  $k$-th order in perturbation theory, any contribution from the time-ordered product in 
Eq.~(\ref{gpert}), or its generalization, is represented by a diagram with $2p$ external 
lines and $k$ interaction lines (from here on called vertices), 
all connected by means of oriented fermion lines. 
These fermion lines arise from contractions between annihilation and creation operators.
Applying the Wick theorem to any such arbitrary diagram, results in the following Feynman rules. In the following we will explicitly include the $\hbar$ factor.
\begin{description}
\item[\underline{Rule 1}:] Draw all, topologically distinct and connected diagrams with $k$ vertices, and $p$ incoming and $p$ outgoing external lines, using directed arrows. For interaction vertices the external lines are not present.
\item[\underline{Rule 2}:] Each oriented fermion line represents a Wick contraction, leading to the unperturbed propagator  
${\rm i}\hbar g_{\alpha\beta}^{(0)}(t_\alpha-t_\beta)$ [or ${\rm i}\hbar g_{\alpha\beta}^{(0)}(\omega_i)$]. 
In time formulation, the $t_\alpha$ and $t_\beta$ label the times of the vertices at the end and at the beginning of the line. 
In energy formulation, $\omega_i$ denotes the energy carried by the propagator. 
\item[\underline{Rule 3}:] Each fermion line starting from and ending at the \emph{same} vertex is an 
equal-time propagator,  $-{\rm i}\hbar g_{\alpha\beta}^{(0)}(0^-)=\rho_{\alpha\beta}^{(0)}$.
\item[\underline{Rule 4}:] For each one-body, two-body or three-body vertex, write down a factor $\frac{\rm i}{\hbar} U_{\alpha \beta}$, \, $-\frac{\rm i}{\hbar} V_{\alpha\gamma,\beta\delta}$  or  $-\frac{\rm i}{\hbar} W_{\alpha\gamma\xi,\beta\delta\theta}$, respectively. For effective interactions, the factors are $-\frac{\rm i}{\hbar} \widetilde{U}_{\alpha \beta}$, \, $-\frac{\rm i}{\hbar} \widetilde{V}_{\alpha\gamma,\beta\delta}$.
% Note that it is +U  and -\tildeU because U is subtracted in H_1 and in eq. (10)...  [CB]
\end{description}
When propagator renormalization is considered, only skeleton diagrams are used in the 
expansion. In that case, the previous rules apply with the substitution 
${\rm i} \hbar g_{\alpha\beta}^{(0)} \to 
{\rm i} \hbar g_{\alpha\beta}$.
Furthermore, note that  Rule 3 applies to diagrams embedded 
in the one-boy effective interaction 
 and therefore they should not be considered explicitly in
an interaction-irreducible expansion. 
In calculating $\widetilde U$, however, 
one should use the correlated $\rho_{\alpha\beta}$ instead of the unperturbed one. 
\begin{description}
\item[\underline{Rule 5}:] Include a factor $(-1)^{L}$ where $L$ is the number of closed fermion loops. This sign comes from the odd permutation of  operators needed to create a loop
and does not include loops of a single propagator, already accounted for by Rule 3.
\item[\underline{Rule 6}:] For a diagram representing a $2p$-point Green's function, add a factor $(-{\rm i}/\hbar)$, whereas for a $2p$-point interaction vertex without external lines (such as $\Sigma^\star$) add a factor ${\rm i}\hbar$.
\end{description}
The next two rules require a distinction between the time and the energy representation. 
In the time representation:
\begin{description}
\item[\underline{Rule 7}:] Assign a time to each interaction vertex. All the fermion lines connected to the same vertex $i$ share the same time,~$t_i$. 
\item[\underline{Rule 8}:] Sum over all the internal quantum numbers and integrate over all internal times from $-\infty$ to $+\infty$. 
\end{description}
Alternatively, in energy representation:
\begin{description}
\item[\underline{Rule 7'}:]  Label each fermion line with an energy $\omega_i$, 
under the \emph{constraint} that the total incoming energy equals the total outgoing energy at 
each interaction vertex, \hbox{$\sum_i\omega_i^{in}=\sum_i\omega_i^{out}$}.
\item[\underline{Rule 8'}:] Sum over all the internal quantum numbers and integrate over each independent internal energy, with an extra factor $\frac{1}{2\pi}$, i.e. $\int^{+\infty}_{-\infty} \frac{{\rm d}\omega_i}{2\pi}$.
\end{description}
Each diagram is then multiplied by a combinatorial factor S that  originates from the number of 
equivalent Wick contractions that lead to it. This symmetry factor 
represents the order of the symmetry group for one specific diagram or, in other words, 
the order of the permutation group of both open and closed lines, once the vertices are fixed. 
Its structure, assuming only 2BFs and 3BFs, is the following :
\begin{equation}
S=\frac{1}{k!}\frac{1} {[(2!)^2]^{q} [(3!)^2]^{k-q} }\binom{k}{q} \; C
= \prod_i S_i \; .
\label{diagsymfac}
\end{equation}
Here, $k$ represents the order of expansion. 
$q$ ($k-q$) denotes the number of two-body (three-body) vertices in the diagram.
The binomial factor counts the number of terms in the expansion $(V+W)^k$ 
that have $q$ factors of $V$ and $k-q$ factors of $W$.
Finally, $C$ is  the overall number of \emph{distinct} contractions and reflects 
the symmetries of the diagram. Stating general rules to find $C$ is not simple. 
For example, explicit simple rules valid for the well-known $\lambda \phi^4$ scalar  theory are still 
an object of debate~\cite{ch11_Feyn_rules}. 
An explicit calculation for $C$ has to be carried out diagram by diagram 
\cite{ch11_Feyn_rules}. Eq.~(\ref{diagsymfac}) can normally be factorized in a product factors $S_i$,
each due to a particular symmetry of the diagram. In the following, we list a series of specific examples which is,
by all  means, not exhaustive.
\begin{description}
 \item[\underline{Rule 9}:]  For each group of $n$ symmetric lines, or symmetric groups-of-lines as defined below, multiply by a symmetry factor $S_i$=$\frac{1}{n!}$. The overall symmetry factor of the diagram will be $S=\prod_i S_i$.
Possible cases include:
  \end{description}
\begin{itemize}
 \item[(i)]\quad {\em Equivalent lines}.  
 $n$ equally-oriented fermion lines are said to be equivalent if they start from the same initial vertex and end on the same final vertex.
 \item[(ii)]\quad {\em Symmetric and interacting lines}.  
 $n$ equally-oriented fermion lines that start from the same initial vertex and end on the same final 
 vertex, but are linked via an interaction vertex to one or more close fermion line blocks. 
 The factor arises as long as the diagram is {\em invariant} under the permutation of the two blocks.
 \item[(iii)]\quad {\em Equivalent groups of lines}. 
 These are blocks of interacting lines (e.g. series of bubbles) that are equal to each other: 
           they all start from the same initial vertex and end on the same final vertex.
 \end{itemize} 

 Rule 9(i)  is the most well-known case and applies, for instance, to the two second order diagrams 
 of Fig.~\ref{fig:2ndOrd}. Diagram a) in Fig.~\ref{fig:2ndOrd} has 2 upward-going equivalent lines and requires a symmetry factor $S_e$=$\frac1{2!}$. In contrast, diagram b) in Fig.~\ref{fig:2ndOrd} has 3 upward-going equivalent lines and 2 downward-going equivalent lines, that give a factor $S_e$=$\frac1{2! \, 3!}$=$\frac1{12}$. For an extended explanation on how to calculate this combinatorial factor and examples for rules 9(ii) and 9(iii) we refer to Ref.~\cite{ch11_Carbone2013Nov}.
 
 As an example of the application of the above Feynman rules, we give here the formulae for diagram c) in Fig.~\ref{fig:3rdOrd}. There are two sets of upward-going equivalent lines, which contribute to a
symmetry factor $S_e=\frac{1}{2^2}$. Considering the overall factor of Eq.~(\ref{diagsymfac}) and the
presence of one closed fermion loop, one finds:

\begin{eqnarray}
\Sigma^{(c)}_{\alpha \beta}(\omega)=
- \frac{(i \hbar)^{4} }{4}
%
\int\frac{{\rm d}\omega_1}{2\pi} \cdots \int\frac{{\rm d}\omega_4}{2\pi}
\sum_{\substack{ \gamma\delta\nu \mu\epsilon\lambda \\ \xi\eta\theta \sigma\tau\chi}} 
&&
\widetilde{V}_{\alpha\gamma,\delta\nu}G^{(0)}_{\delta\mu}(\omega_1)G^{(0)}_{\nu\epsilon}(\omega_2) 
W_{\mu\epsilon\lambda,\xi\eta\theta}G^{(0)}_{\xi\sigma}(\omega_3)G^{(0)}_{\eta\tau}(\omega_4) \times 
\nonumber\\ &&
G^{(0)}_{\lambda\gamma}(\omega_1+\omega_2-\omega)
\widetilde{V}_{\sigma\tau,\beta\chi}
G^{(0)}_{\chi\theta}(\omega_3+\omega_4-\omega)  \, .
\end{eqnarray}


\section*{Appendix 2: Chiral next-to-next-to-leading order three-nucleon forces}
\addcontentsline{toc}{section}{Appendix 2: Chiral next-to-next-to-leading order three-nucleon forces}
\label{app:scgf_3NF}

In this appendix we perform the average given in Eq.~(\ref{eq:dd3bf_new}) for the specific case of leading order three-nucleon forces (3NFs), i.e. next-to-next-to-leading order (NNLO), in the chiral effective filed theory expansion~\cite{ch11_vKol1994,ch11_Epelbaum2002Dec2}. At NNLO we have a a two-pion exchange (TPE), one-pion exchange (OPE) and contact 3NFs, given respectively by the following expressions:

\begin{equation}
V^\mathrm{3NF}_\mathrm{TPE} =\sum_{i\neq j\neq k}  \frac{g_A^2}{8F_\pi^4}
\frac{(\boldsymbol\sigma_i\cdot{\bf q}_i)(\boldsymbol\sigma_j\cdot{\bf q}_j)}{({\bf q}_i^2 + M_\pi^2)
({\bf q}_j^2 + M_\pi^2)}
F_{ijk}^{\alpha\beta}\tau_i^{\alpha}\tau_j^{\beta}\,,
\label{eq:tpe}
\end{equation}
\begin{equation}
V^\mathrm{3NF}_\mathrm{OPE} = -\sum_{i\neq j\neq k} \frac{c_D g_A}{8F_\pi^4\Lambda_\chi}
\frac{\boldsymbol\sigma_j\cdot{\bf q}_j}{{\bf q}_j^2 + M_\pi^2}(\boldsymbol\tau_i\cdot\boldsymbol\tau_j)
(\boldsymbol\sigma_i\cdot{\bf q}_j)\,;
\label{eq:ope}
\end{equation}
\begin{equation}
V^\mathrm{3NF}_\mathrm{cont} =  \sum_{j\neq k} \frac{c_E}{2F_\pi^4\Lambda_\chi}
\boldsymbol\tau_j \cdot \boldsymbol\tau_k \,.
\label{eq:cont}
\end{equation}
In the TPE contribution of Eq.~(\ref{eq:tpe}), the quantity $F_{ijk}^{\alpha\beta}$ is
\begin{equation}
F_{ijk}^{\alpha\beta}=\delta^{\alpha\beta} [-4M_\pi^2c_1+2 c_3{\bf q}_i\cdot{\bf q}_j]
+\sum_\gamma c_4\epsilon^{\alpha\beta\gamma}\tau^\gamma_k\boldsymbol\sigma_k\cdot[{\bf q}_i\times{\bf q}_j]\,.
\label{eq:tpe_tensor}
\end{equation}
The force is regularized with a function that in Jacobi momenta reads:
\begin{equation}
\label{eq:regulator}
f({\bf p_1},{\bf p_2},{\bf p_3})=f(p,q)=\exp{\left[-\frac{(p^2+3q^2/4)}{\Lambda^2_\textrm{3NF}}\right]^n}\,,
\end{equation}
where ${\bf p}=({\bf p}_1-{\bf p}_2)/2$ and ${\bf q}=2/3({\bf p_3}-({\bf p}_1+{\bf p}_2)/2)$ are identified only in this expression as the Jacobi momenta. $\Lambda_\textrm{3NF}$ defines the cutoff value applied to the 3NF in order to obtain a three-body contribution which dies down similarly to the two-body part one. The regulator function is applied both on incoming $({\bf p},{\bf q})$ and outgoing $({\bf p'},{\bf q'})$ Jacobi momenta. In the numerical calculation, the approximation of ${\bf P}=0$ is used to facilitate the solution of equations. The averaged terms presented in the following are calculated only for equal relative incoming and outgoing momentum, i.e. ${\bf k}={\bf k'}$ with ${\bf k}=|{\bf p_1}-{\bf p_2}|/2$ and ${\bf k'}=|{\bf p'_1}-{\bf p'_2}|/2$; an extrapolation is then applied to obtain the off-diagonal potential matrix elements~\cite{ch11_Carbone2014}. Given these conditions, the regulator on incoming and outgoing momenta can be defined as a function of $f(k,p_3)$.\\

{\bf Symmetric Nuclear Matter.} Let's start with the isospin-symmetric case of nuclear matter. Evaluating Eq.~(\ref{eq:dd3bf_new}) for the TPE term of Eq.~(\ref{eq:tpe}) leads to three contracted in-medium two-body interactions.\\ %These are represented in Figs.~\ref*{TPE-1}-\ref*{TPE-3}. 
{\bf TPE-1:} The first term is an isovector tensor term, this corresponds to a 1$\pi$ exchange contribution with an in-medium pion propagator:
\begin{equation}
\widetilde V_\mathrm{TPE-1}^\mathrm{3NF}=\frac{g_A\,\rho_f}{2 F_\pi^4}
\frac{(\boldsymbol\sigma_1\cdot{\bf q})(\boldsymbol\sigma_2\cdot{\bf q})}{[q^2 + M_\pi^2]^2}
\boldsymbol\tau_1\cdot\boldsymbol\tau_2[2 c_1M_\pi^2+ c_3\,q^2]\,.
\label{eq:tpe_dd_1}
\end{equation}
$\rho_f$ defines the integral of the correlated momentum distribution function weighed by the regulator function $f(k,p_3)$
\begin{equation}
\frac{\rho_f}{\nu}=\int \frac{{\mathrm d}{\bf p}_3}{(2\pi)^3}n({\bf p}_3)f(k,p_3)\,,
\label{eq:rho_f}
\end{equation}
where $\nu$ is the degeneracy of the system, $\nu=4$ in the isospin symmetric case. If the regulator function included in Eq.~(\ref{eq:rho_f}) were not dependent on the internal integrated momentum $p_3$, the integral would reduce to the value of the total density of the system, $\rho$, divided by the degeneracy and multiplied by an external regulator function.\\ 
{\bf TPE-2:} The second term is also a tensor contribution to the in-medium NN interaction. It adds up to the previous term. %and contributes to $V^v_{\sigma q}$ in Eq.~(\ref{on-shell_vnn}). 
This term includes vertex corrections to the 1$\pi$ exchange due to the presence of the nuclear medium:
\begin{eqnarray}
%&& 
\widetilde V_\mathrm{TPE-2}^\mathrm{3NF}&=& \frac{g_A^2}{8\pi^2F_\pi^4}
\frac{\boldsymbol\sigma_1\cdot{\bf q}\boldsymbol\sigma_2\cdot{\bf q}}{q^2 + M_\pi^2} \boldsymbol\tau_1\cdot\boldsymbol\tau_2
\\ \nonumber && 
\times
\Big\{-4c_1M_\pi^2\left[\Gamma_1(k)+\Gamma_0(k)\right]
%\\\nonumber &&  \qquad\qquad
-(c_3+c_4)\left[q^2(\Gamma_0(k)+2\Gamma_1(k)+\Gamma_3(k))+4\Gamma_2(k)\right]
%\\ && \qquad\qquad
+4c_4{\pazocal I}(k)\Big\}\,.
\label{eq:tpe_dd_2}
\end{eqnarray}
We have introduced the functions $\Gamma_0(k), \Gamma_1(k), \Gamma_2(k), \Gamma_3(k), {\pazocal I}(k)$, which are integrals over a single pion propagator:
\begin{eqnarray}
\label{eq:gamma0}
\frac{\Gamma_0(k)}{(2\pi)^2}&=& \qquad \int\frac{{\mathrm d}{\bf p}_3}{(2\pi)^3}n({\bf p}_3)
\frac{1}{[{\bf k}\pm{\bf p}_3]^2 + M_\pi^2}f(k,p_3)\,;
\\ \label{eq:gamma1}
\frac{\Gamma_1(k)}{(2\pi)^2}&=&\frac{1}{k^2}\int\frac{\d{\bf p}_3}{(2\pi)^3}n({\bf p}_3)
\frac{\pm{\bf k}\cdot{\bf p_3}}{[{\bf k}\pm{\bf p}_3]^2 + M_\pi^2}f(k,p_3)\,;
\\ \label{eq:gamma2}
\frac{\Gamma_2(k)}{(2\pi)^2}&=&\frac{1}{2k^2}\int\frac{\d{\bf p}_3}{(2\pi)^3}n({\bf p}_3)
\frac{p_3^2k^2-({\bf k}\cdot{\bf p_3})^2}{[{\bf k}\pm{\bf p}_3]^2 + M_\pi^2}f(k,p_3)\,;
\\ \label{eq:gamma3}
\frac{\Gamma_3(k)}{(2\pi)^2}&=&\frac{1}{2k^4}\int\frac{\d{\bf p}_3}{(2\pi)^3}n({\bf p}_3)
\frac{3({\bf k}\cdot{\bf p_3})^2-p_3^2k^2}{[{\bf k}\pm{\bf p}_3]^2 + M_\pi^2}f(k,p_3)\,;
\\ \label{eq:i_integral}
\frac{{\pazocal I}(k)}{(2\pi)^2}&=& \qquad  \int \frac{{\mathrm d}{\bf p}_3}{(2\pi)^3}n({\bf p}_3)
\frac{[{\bf p}_3\pm{\bf k}]^2}{[{\bf p}_3\pm{\bf k}]^2 + M_\pi^2}f(k,p_3)\,.
\end{eqnarray}
The $\pm$ in the pion propagator depend on which average term one is calculating in Eq.~\eqref{eq:dd3bf_new} if the direct or the one-exchange terms, $P_{13}$ or $P_{23}$.\\
\noindent
{\bf TPE-3:} The last TPE contracted term includes in-medium effects for a 2$\pi$ exchange two-body term:
\begin{eqnarray}
\nonumber 
\widetilde V_\mathrm{TPE-3}^\mathrm{3NF} = \frac{g_A^2}{16\pi^2F_\pi^4}
  &\Big\{ & -12c_1M_\pi^2\big[2\Gamma_0(k)-G_0(k,q)(2M_\pi^2+q^2)\big]
\\\nonumber &&   %\quad
- \, c_3\big[12\pi^2\rho_f-12(2M_\pi^2+q^2)\Gamma_0(k)
- \, 6q^2\Gamma_1(k)+3(2M_\pi^2+q^2)^2G_0(k,q)\big] 
\\\nonumber &&  %\qquad
+  \, 4c_4 \boldsymbol\tau_1\cdot\boldsymbol\tau_2(\boldsymbol\sigma_1\cdot\boldsymbol\sigma_2\, q^2-\boldsymbol\sigma_1\cdot{\bf q}\boldsymbol\sigma_2\cdot{\bf q})
G_2(k,q)
\\\nonumber &&  %\qquad
-\, (3c_3+c_4\boldsymbol\tau_1\cdot\boldsymbol\tau_2)\,i(\boldsymbol\sigma_1+\boldsymbol\sigma_2)\cdot({\bf q}\times{\bf k})
\\\nonumber &&  \qquad \qquad \qquad \qquad \times
\big[2\Gamma_0(k)+2\Gamma_1(k)-(2M_\pi^2+q^2)G_0(k,q)+2G_1(k,q)\big]
\\\nonumber &&  %\qquad 
- \, 12c_1M_\pi^2\,  i(\boldsymbol\sigma_1+\boldsymbol\sigma_2)\cdot({\bf q}\times{\bf k})
\big[G_0(k,q)+2G_1(k,q)\big]
\\ &&  %\qquad
+ \, 4c_4\boldsymbol\tau_1\cdot\boldsymbol\tau_2\boldsymbol\sigma_1\cdot({\bf q}\times{\bf k})\boldsymbol\sigma_2\cdot({\bf q}\times{\bf k})
%\\ && \qquad\times
\big[G_0(k,q)+4G_1(k,q)+4G_3(k,q)\big]\Big\} \,. \qquad \qquad 
\label{eq:tpe_dd_3}
\end{eqnarray}
Here we have introduced the function $G_0(k,q)$, which is an integral over the product of two different pion propagators:
\begin{equation}
\frac{G_{0,\star,\star\star}}{(2\pi)^2}(k,q)=
\int \frac{{\mathrm d}{\bf p}_3}{(2\pi)^3}n({\bf p}_3)
\frac{\{p_3^0,p_3^2,p_3^4\}}{\big[[{\bf k}+{\bf q}+{\bf p}_3]^2+M_\pi^2\big]\big[[{\bf p}_3+{\bf k}]^2+M_\pi^2\big]}f(k,p_3)\,.
\label{eq:G_0} 
\end{equation}
The functions $G_{\star}(k,q), G_{\star\star}(k,q)$ have been introduced to define the rest of the functions, $G_1(k,q), G_2(k,q)$ and $G_3(k,q)$:
\begin{equation}
\label{eq:G_1}
G_1(k,q)=\frac{\Gamma_0(k)-(M_\pi^2+k^2)G_0(k,q)-G_\star(k,q)}{4k^2-q^2}\,,
\end{equation}
\begin{equation}
\label{eq:G_1star}
G_{1\star}(k,q)=\frac{3\Gamma_2(k)+k^2\Gamma_3(k)-(M_\pi^2+k^2)G_\star(k,q)-G_{\star\star}(k,q)}{4k^2-q^2}\,,
\end{equation}
\begin{equation}
\label{eq:G_2}
G_2(k,q)=(M_\pi^2+k^2)G_1(k,q)+G_\star(k,q)+G_{1\star}(k,q)\,,
\end{equation}
\begin{equation}
\label{eq:G_3}
G_3(k,q)=\frac{\Gamma_1(k)/2-2(M_\pi^2+k^2)G_1(k,q)-2G_{1\star}(k,q)-G_\star(k,q)}{4k^2-q^2}\,.
\end{equation}
Note that $G_{1\star}(k,q)$ is needed only to define $G_2(k,q),\,G_3(k,q)$.\\

Integrating Eq.~(\ref{eq:dd3bf_new}) for the OPE 3NF term, given in Eq.~(\ref{eq:ope}), leads to two contributions.\\
{\bf OPE-1:} The first one is a tensor contribution which defines a vertex correction to a 1$\pi$ exchange nucleon-nucleon term. It is proportional to the quantity $\rho_f$, similar to what was obtained for the TPE 3NF contracted term $\widetilde V_\mathrm{TPE-1}^\mathrm{3NF}$ (see Eq.~\ref{eq:tpe_dd_1}):
\begin{equation}
\widetilde V_\mathrm{OPE-1}^\mathrm{3NF}=-\frac{c_D\,g_A\,\rho_f}{8\,F_\pi^4\,\Lambda_\chi}
\frac{(\boldsymbol\sigma_1\cdot{\bf q})(\boldsymbol\sigma_2\cdot{\bf q})}{q^2 + M_\pi^2}
(\boldsymbol\tau_1\cdot\boldsymbol\tau_2)\,.
\label{eq:ope_dd_1}
\end{equation}
As for the $\widetilde V_\mathrm{TPE-1}^\mathrm{3NF}$ term, $\widetilde V_\mathrm{OPE-1}^\mathrm{3NF}$ it's an isovector tensor term.\\% $V^v_{\sigma q}$ of Eq.~(\ref{on-shell_vnn}).
{\bf OPE-2:} The second term derived from the 3NF OPE defines a vertex correction to the short-range contact NN interaction. It reads:
\begin{eqnarray}
\nonumber &&
\widetilde V_\mathrm{OPE-2}^\mathrm{3NF}=\frac{c_Dg_A}{16\pi^2F_\pi^4\Lambda_\chi}\Big\{
\big(\Gamma_0(k)+2\Gamma_1(k)+\Gamma_3(k)\big)
%\\\nonumber &&
\left[\boldsymbol\sigma_1\cdot\boldsymbol\sigma_2\Big(2k^2-\frac{q^2}{2}\Big)\right.
\\\nonumber && \left.\qquad
+(\boldsymbol\sigma_1\cdot{\bf q}\,\boldsymbol\sigma_2\cdot{\bf q})\Big(1-\frac{2k^2}{q^2}\Big)-\frac{2}{q^2}\boldsymbol\sigma_1\cdot({\bf q}\times{\bf k})
\boldsymbol\sigma_2\cdot({\bf q}\times{\bf k})\frac{1}{q^2}\right]
\\ && \qquad
+2\Gamma_2(k)(\boldsymbol\sigma_1\cdot\boldsymbol\sigma_2)\Big](\boldsymbol\tau_1\cdot\boldsymbol\tau_2)
+6{\pazocal I}(k)\Big\}\,.
\label{eq:ope_dd_2}
\end{eqnarray}
\vskip .3 cm
\noindent
{\bf Exercise 11.8.} Compute Eq.~(\ref{eq:dd3bf_new}) for the contact term given in Eq.~(\ref{eq:cont}). Demonstrate that it yields a scalar central contribution to the in-medium NN interaction proportional to $\rho_f$ with formal expression:
\begin{equation}
\widetilde V_\mathrm{cont}^\mathrm{3NF}=-\frac{3 c_E\rho_f}{2 F_\pi^4\Lambda_\chi}\,.
\label{eq:cont_dd}
\end{equation}
\\


%%%%%%%%%%%% neutron matter %%%%%%%%%%%%%%%%%%%%%%
{\bf Pure Neutron Matter.} In the case of pure neutron matter, the evaluation of Eq.~(\ref{eq:dd3bf_new}) is simplified. In fact,
the trace over isospin is trivial, neutron matter can only be in total isospin $T=1$, i.e. $\boldsymbol\tau_1\cdot\boldsymbol\tau_2=1$. Consequently the exchange operator reduces only to the momentum and spin part, i.e. in spin space it reads:
\begin{equation}
P_{12}=\frac{1+\boldsymbol\sigma_1\cdot\boldsymbol\sigma_2}{2}\,.
\label{eq:perm_op_2}
\end{equation}
Furthermore it can also be proved that for non-local regulator, such as the one in Eq.~\eqref{eq:regulator}, the 3NF terms proportional to $c_4, c_D, c_E$ go to zero \cite{ch11_Tolos2008,ch11_Hebeler2010Jul}. Therefore the only density-dependent contributions, which are non-zero in neutron matter, are those proportional to low-energy constants $c_1$ and $c_3$ in Eq.~(\ref{eq:tpe}). The density-dependent interacting terms obtained in neutron matter will only differ with respect to the symmetric case ones by different pre-factors. This is due to the fact that the only part which changes from the symmetric to the pure neutron matter case is the trace over isospin indices. 

In order to obtain the correct degeneracy for neutron matter, i.e.  $\nu=2$, we need to replace $\rho_f \rightarrow 2\rho_f$ in the $\widetilde V_\mathrm{TPE-1}^\mathrm{3NF}$ contribution of Eq.~(\ref{eq:tpe_dd_1}) and the $\widetilde V_\mathrm{TPE-3}^\mathrm{3NF}$ contribution of Eq.~(\ref{eq:tpe_dd_3}), (see also Eq.~(\ref{eq:rho_f})). The isovector tensor terms $\widetilde V_\mathrm{TPE-1}^\mathrm{3NF}$ and  $\widetilde V_\mathrm{TPE-2}^\mathrm{3NF}$, given in Eqs.~(\ref{eq:tpe_dd_1})-(\ref{eq:tpe_dd_2}) must then change prefactor according to:
\begin{equation}
\widetilde V_\mathrm{TPE-1}^\mathrm{3NF}: \boldsymbol\tau_1\cdot\boldsymbol\tau_2 \rightarrow \frac{1}{2}\boldsymbol\tau_1\cdot\boldsymbol\tau_2\,, 
\label{eq:pnm_tpe_1}
\end{equation}
\begin{equation}
\widetilde V_\mathrm{TPE-2}^\mathrm{3NF}: \boldsymbol\tau_1\cdot\boldsymbol\tau_2 \rightarrow 
\frac{1}{4}(\boldsymbol\tau_1\cdot\boldsymbol\tau_2-2)\,.
\label{eq:pnm_tpe_2}
\end{equation}
The isoscalar part of the density-dependent potential appearing in $\widetilde V_\mathrm{TPE-3}^\mathrm{3NF}$, which contributes to both a central and spin-orbit terms, must change prefactor according to:
\begin{equation}
\widetilde V_\mathrm{TPE-3}^\mathrm{3NF}: 1 \rightarrow \frac{1}{3}\,.
\label{eq:pnm_tpe_3}
\end{equation}




\bibliographystyle{spphys}
\bibliography{lnp}





